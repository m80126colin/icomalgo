\documentclass[12pt,a4paper,oneside]{book}

\usepackage{CJK,CJKnumb}
\usepackage{amsmath,amsfonts,amssymb,amsthm}	% AMS 標準套件
\usepackage{titletoc,titlesec}					% titletoc 目錄修改套件, titlesec 美化章節標題套件
\usepackage{makeidx}
\usepackage{listings}
\makeindex

\linespread{1.24}

%% === 設定頁面格式 ===
%\hoffset				% 水平位移,預設為 0pt
\voffset = -15pt		% 垂直位移,預設為 0pt
\oddsidemargin = 0pt	% 預設為 31pt
%\topmargin = 20pt		% 預設為 20pt
%\headheight = 12pt		% header 的高度,預設為 12pt
%\headsep = 25pt		% header 和 body 的距離,預設為 25pt
\textheight = 620pt		% body 內文部分的高度,預設為 592pt
\textwidth = 450pt		% body 內文部分的寬度,預設為 390pt
%\marginparsep = 10pt	% margin note 和 body 的距離,預設為 10pt
%\marginparwidth = 35pt	% margin note 的寬度,預設為 35pt
%\footskip = 30pt		% footer 高度 + footer 和 body 的距離,預設為 30pt

\begin{document}
\begin{CJK}{UTF8}{bkai}

%% === 常用的指令,替換成中文 ===
\renewcommand\figurename{圖~}
\renewcommand\tablename{表~}
\renewcommand\contentsname{目~錄~}
\renewcommand\listfigurename{插~圖~目~錄}
\renewcommand\listtablename{表~格~目~錄}
\renewcommand{\appendixname}{附~錄}
%\renewcommand{\refname}{參~考~資~料}		% article
\renewcommand{\bibname}{參~考~文~獻}		% book
\renewcommand\indexname{索~引}
\renewcommand{\today}{\number\year~年~\number\month~月~\number\day~日}
%\newcommand{\zhtoday}{\CJKdigits{\the\year}年\CJKnumber{\the\month}月\CJKnumber{\the\day}日}

%% ===
\setcounter{secnumdepth}{3}									% 設定計數到 subsubsection
\renewcommand{\thepart}{第\CJKnumber{\arabic{part}}部分}
\renewcommand{\thechapter}{\arabic{chapter}}
\renewcommand{\thesection}{\arabic{section}}				% 改 section 為 1, 2, 3 非 1.1, 1.2, 1.3
\renewcommand{\thesubsection}{\arabic{subsection}}			% subsection 也改一改
\renewcommand{\thesubsubsection}{\arabic{subsubsection}}	% subsubsection 也改一改

%% === 設定章節標題 (配合 titlesec) ===
\titleformat{\part}[display]
	{\center\huge\bfseries}
	{\thepart}
	{1em}
	{\Huge}
%
\titleformat{\chapter}[display]
	{\bf\Large}
	{第~\CJKnumber{\thechapter}~章}
	{1ex}
	{\Huge}
	[\vspace{2ex}]
%
\titleformat{\section}{\Large\bfseries}{第\CJKnumber{\thesection}節}{1em}{}
%
\titleformat{\subsection}{\large\bfseries}{\CJKnumber{\thesubsection}、}{0.5em}{}
%
\titleformat{\subsubsection}{\bfseries}{(\CJKnumber{\thesubsubsection})}{0.5em}{}
% \titlespacing{\subsubsection}{0pt}{5pt}{-10pt}

%% === 設定目錄標題 (配合 titletoc) ===
\titlecontents{part}
[0em]
{\center\Large}
{}
{}
{}
%
\titlecontents{chapter}
[0em]
{}
{\large\bf{第\CJKnumber{\thecontentslabel}章~~}}
{}{~~\titlerule*{.}\bf\contentspage}
%
\titlecontents{section}
[4em]
{}
{第\CJKnumber{\thecontentslabel}節\quad}
{}{~~\titlerule*{.} \contentspage}
%
\titlecontents{subsection}
[8em]
{}
{\CJKnumber{\thecontentslabel}、}
{}{~~\titlerule*{.} \contentspage}

%% === 定義 ===
\newtheorem{myrule}{\begin{CJK}{UTF8}{bkai}原理\end{CJK}}[section]
\newtheorem{mythm}{\begin{CJK}{UTF8}{bkai}定理\end{CJK}}[section]
\newtheorem{mydef}{\begin{CJK}{UTF8}{bkai}定義\end{CJK}}[section]
\newtheorem*{mydef*}{\begin{CJK}{UTF8}{bkai}定義\end{CJK}}
\newtheorem{mypropo}{\begin{CJK}{UTF8}{bkai}性質\end{CJK}}[section]
\newtheorem{myquest}{\begin{CJK}{UTF8}{bkai}例題\end{CJK}}[section]
\newtheorem{myexe}{\begin{CJK}{UTF8}{bkai}練習題\end{CJK}}[subsection]
\numberwithin{equation}{section}
\renewenvironment{proof}{\begin{CJK}{UTF8}{bkai}\textbf{證明}\end{CJK}}{\qed}
\newenvironment{mysol}{\begin{CJK}{UTF8}{bkai}\textbf{解答}\end{CJK}}{\qed}

\title{Introduction to Competitive Algorithms\\競賽演算法簡介}
\author{許胖}
\maketitle
\tableofcontents

\part{基礎競賽程式技巧}
\chapter{預熱}
\section{算術運算子}

\paragraph{}C++ 的算術運算子有五類:
\paragraph{}五則運算的運算順序和我們自然的觀點相同,都是先乘除餘、後加減。加、減、乘法幾乎與普通的運算無異,除法需特別注意,若資料型態是 int,則除法為整數除法,
\begin{lstlisting}[language=C++]
cout << 8 / 5 << endl;
\end{lstlisting}
\paragraph{}的結果為:
\begin{lstlisting}
1
\end{lstlisting}
\paragraph{}若資料型態為 double,則除法為浮點數除法,
\begin{lstlisting}[language=C++]
cout << 8.0 / 5.0 << endl;
\end{lstlisting}
\paragraph{}的結果則為:
\begin{lstlisting}
1.6
\end{lstlisting}
\paragraph{}除此之外,除法還有除以零的問題,以下四種 cout 請各位讀者自行分別執行,會發生什麼問題、產生什麼現象,有興趣的話可以在網路上多查些資料。
\end{CJK}
\end{document}
\ifx \allfiles \undefined
\documentclass[12pt,a4paper,oneside]{report}

%% === CJK 套件 ===
\usepackage{CJKutf8,CJKnumb}                    % 中文套件
\usepackage[unicode]{hyperref,xcolor}
\hypersetup{
    colorlinks,
    linkcolor={blue!100!black},
    citecolor={blue!75!black},
    urlcolor={blue!50!black}
}
%% === AMS 標準套件 ===
\usepackage{amsmath,amsfonts,amssymb,amsthm}    % 數學符號
%% === 章節內容 ===
\usepackage{enumitem}                           % 修改 enumerate, item
\usepackage{titletoc,titlesec}                  % titletoc 目錄修改套件, titlesec 美化章節標題套件
\usepackage{imakeidx}                           % 索引
%% ===  ===
\usepackage[chapter]{algorithm}                 % 演算法套件
\usepackage[noend]{algpseudocode}               % pseudocode 套件
\usepackage{listings}                           % 程式碼
%% ===  ===
\usepackage{tikz,tkz-graph,tkz-berge}
%% ===  ===
\usepackage{xkeyval,xargs}

\makeindex[name=noun]        % 索引生成
\linespread{1.24}

%% === 設定頁面格式 ===
%\hoffset         = 10pt                      % 水平位移,預設為 0pt
\voffset         = -15pt                     % 垂直位移,預設為 0pt
\oddsidemargin   = 0pt                       % 預設為 31pt
%\topmargin       = 20pt                      % 預設為 20pt
%\headheight      = 12pt                      % header 的高度,預設為 12pt
%\headsep         = 25pt                      % header 和 body 的距離,預設為 25pt
\textheight      = 620pt                     % body 內文部分的高度,預設為 592pt
\textwidth       = 450pt                     % body 內文部分的寬度,預設為 390pt
%\marginparsep    = 10pt                      % margin note 和 body 的距離,預設為 10pt
%\marginparwidth  = 35pt                      % margin note 的寬度,預設為 35pt
%\footskip        = 30pt                      % footer 高度 + footer 和 body 的距離,預設為 30pt

%% === itemize,enumerate 設定 ===
%  使用 enumitem 套件
\setlist[itemize]{itemsep=0pt,parsep=0pt}
\setlist[enumerate]{itemsep=0pt,parsep=0pt}

%% === 設定 C++ 格式 ===
\lstset{
  language=[11]C++,                     % 設定語言
  %% === 空白, tab 相關 ===
  tabsize=2,                            % 設定 tab = 多少空白
  %showspaces=true,                      % 設定是否標示空白
  %showtabs=true,                        % 設定是否標示 tab
  %tab=\rightarrowfill,                  % 設定 tab 樣式
  %% === 行數相關 ===
  numbers=left,                         % 行數標示位置
  stepnumber=1,                         % 每隔幾行標示行數
  numberstyle=\tiny,
  basicstyle=\ttfamily,
  keywordstyle=\color{blue}\ttfamily,
  stringstyle=\color{red}\ttfamily,
  commentstyle=\color{green}\ttfamily,
  morecomment=[l][\color{magenta}]{\#},
  morekeywords={cout,endl}
}


\begin{document}
\begin{CJK}{UTF8}{bkai}

%% === 常用的指令,替換成中文 ===
\renewcommand{\figurename}{圖}
\renewcommand{\tablename}{表}
\renewcommand{\contentsname}{目~錄~}
\renewcommand{\listfigurename}{插~圖~目~錄}
\renewcommand{\listtablename}{表~格~目~錄}
\renewcommand{\appendixname}{附~錄}
%\renewcommand{\refname}{參~考~資~料}    % article
\renewcommand{\bibname}{參~考~文~獻}     % book
\renewcommand{\indexname}{索~引}
\renewcommand{\today}{\number\year~年~\number\month~月~\number\day~日}
%\newcommand{\zhtoday}{\CJKdigits{\the\year}年\CJKnumber{\the\month}月\CJKnumber{\the\day}日}

%% === 
\floatname{algorithm}{演算法}

%% === 定義 ===
\newtheorem{myrule}{\begin{CJK}{UTF8}{bkai}原理\end{CJK}}[section]
\newtheorem{mythm}{\begin{CJK}{UTF8}{bkai}定理\end{CJK}}[section]
\newtheorem{mydef}{\begin{CJK}{UTF8}{bkai}定義\end{CJK}}[section]
\newtheorem*{mydef*}{\begin{CJK}{UTF8}{bkai}定義\end{CJK}}
\newtheorem{mypropo}{\begin{CJK}{UTF8}{bkai}性質\end{CJK}}[section]
\newtheorem{myquest}{\begin{CJK}{UTF8}{bkai}例題\end{CJK}}[section]
\newtheorem{myexe}{\begin{CJK}{UTF8}{bkai}練習題\end{CJK}}[subsection]
\numberwithin{equation}{section}
\renewenvironment{proof}{\begin{CJK}{UTF8}{bkai}\textbf{證明}\end{CJK}}{\qed}
\newenvironment{mysol}{\begin{CJK}{UTF8}{bkai}\textbf{解答}\end{CJK}}{\qed}

\fi

\chapter{預熱}

\section{算術運算子}

\paragraph{}C++ 的算術運算子有五類:
\paragraph{}五則運算的運算順序和我們自然的觀點相同,都是先乘除餘、後加減。加、減、乘法幾乎與普通的運算無異,除法需特別注意,若資料型態是 \lstinline{int},則除法為整數除法,
\begin{lstlisting}
  cout << 8 / 5 << endl;
\end{lstlisting}
\paragraph{}的結果為:
\begin{lstlisting}
  1
\end{lstlisting}
\paragraph{}若資料型態為 double,則除法為浮點數除法,
\begin{lstlisting}
  cout << 8.0 / 5.0 << endl;
\end{lstlisting}
\paragraph{}的結果則為:
\begin{lstlisting}
  1.6
\end{lstlisting}
\paragraph{}除此之外,除法還有除以零的問題,以下四種 cout 請各位讀者自行分別執行,會發生什麼問題、產生什麼現象,有興趣的話可以在網路上多查些資料。

\ifx \allfiles \undefined
\printindex[noun]

\clearpage
\end{CJK}
\end{document}
\fi
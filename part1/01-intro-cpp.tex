\ifx \allfiles \undefined
\documentclass[12pt,a4paper,oneside]{report}

%% === CJK 套件 ===
\usepackage{CJKutf8,CJKnumb}                    % 中文套件
\usepackage[unicode]{hyperref,xcolor}
\hypersetup{
    colorlinks,
    linkcolor={blue!100!black},
    citecolor={blue!75!black},
    urlcolor={blue!50!black}
}
%% === AMS 標準套件 ===
\usepackage{amsmath,amsfonts,amssymb,amsthm}    % 數學符號
%% === 章節內容 ===
\usepackage{enumitem}                           % 修改 enumerate, item
\usepackage{titletoc,titlesec}                  % titletoc 目錄修改套件, titlesec 美化章節標題套件
\usepackage{imakeidx}                           % 索引
%% ===  ===
\usepackage[chapter]{algorithm}                 % 演算法套件
\usepackage[noend]{algpseudocode}               % pseudocode 套件
\usepackage{listings}                           % 程式碼
%% ===  ===
\usepackage{tikz,tkz-graph,tkz-berge}
%% ===  ===
\usepackage{xkeyval,xargs}

\makeindex[name=noun]        % 索引生成
\linespread{1.24}

%% === 設定頁面格式 ===
%\hoffset         = 10pt                      % 水平位移,預設為 0pt
\voffset         = -15pt                     % 垂直位移,預設為 0pt
\oddsidemargin   = 0pt                       % 預設為 31pt
%\topmargin       = 20pt                      % 預設為 20pt
%\headheight      = 12pt                      % header 的高度,預設為 12pt
%\headsep         = 25pt                      % header 和 body 的距離,預設為 25pt
\textheight      = 620pt                     % body 內文部分的高度,預設為 592pt
\textwidth       = 450pt                     % body 內文部分的寬度,預設為 390pt
%\marginparsep    = 10pt                      % margin note 和 body 的距離,預設為 10pt
%\marginparwidth  = 35pt                      % margin note 的寬度,預設為 35pt
%\footskip        = 30pt                      % footer 高度 + footer 和 body 的距離,預設為 30pt

%% === itemize,enumerate 設定 ===
%  使用 enumitem 套件
\setlist[itemize]{itemsep=0pt,parsep=0pt}
\setlist[enumerate]{itemsep=0pt,parsep=0pt}

%% === 設定 C++ 格式 ===
\lstset{
  language=[11]C++,                     % 設定語言
  %% === 空白, tab 相關 ===
  tabsize=2,                            % 設定 tab = 多少空白
  %showspaces=true,                      % 設定是否標示空白
  %showtabs=true,                        % 設定是否標示 tab
  %tab=\rightarrowfill,                  % 設定 tab 樣式
  %% === 行數相關 ===
  numbers=left,                         % 行數標示位置
  stepnumber=3,                         % 每隔幾行標示行數
  numberstyle=\tiny,
  %% === 顏色設定 ===
  basicstyle=\ttfamily,
  keywordstyle=\color{blue}\ttfamily,
  stringstyle=\color{red}\ttfamily,
  commentstyle=\color{green}\ttfamily,
  morecomment=[l][\color{magenta}]{\#},
  morekeywords={}
}

\begin{document}
\begin{CJK}{UTF8}{bkai}

%% === 常用的指令,替換成中文 ===
\renewcommand{\figurename}{圖}
\renewcommand{\tablename}{表}
\renewcommand{\contentsname}{目~錄~}
\renewcommand{\listfigurename}{插~圖~目~錄}
\renewcommand{\listtablename}{表~格~目~錄}
\renewcommand{\appendixname}{附~錄}
%\renewcommand{\refname}{參~考~資~料}    % article
\renewcommand{\bibname}{參~考~文~獻}     % book
\renewcommand{\indexname}{索~引}
\renewcommand{\today}{\number\year~年~\number\month~月~\number\day~日}
%\newcommand{\zhtoday}{\CJKdigits{\the\year}年\CJKnumber{\the\month}月\CJKnumber{\the\day}日}

%% === 
\floatname{algorithm}{演算法}

%% === 定義 ===
\newtheorem{myrule}{\begin{CJK}{UTF8}{bkai}原理\end{CJK}}[section]
\newtheorem{mythm}{\begin{CJK}{UTF8}{bkai}定理\end{CJK}}[section]
\newtheorem{mydef}{\begin{CJK}{UTF8}{bkai}定義\end{CJK}}[section]
\newtheorem*{mydef*}{\begin{CJK}{UTF8}{bkai}定義\end{CJK}}
\newtheorem{mypropo}{\begin{CJK}{UTF8}{bkai}性質\end{CJK}}[section]
\newtheorem{myquest}{\begin{CJK}{UTF8}{bkai}例題\end{CJK}}[section]
\newtheorem{myexe}{\begin{CJK}{UTF8}{bkai}練習題\end{CJK}}[subsection]
\numberwithin{equation}{section}
\renewenvironment{proof}{\begin{CJK}{UTF8}{bkai}\textbf{證明}\end{CJK}}{\qed}
\newenvironment{mysol}{\begin{CJK}{UTF8}{bkai}\textbf{解答}\end{CJK}}{\qed}

\fi

\chapter{預熱}

\section{算術運算子}

\paragraph{}C++ 的算術運算子有五類:
\paragraph{}五則運算的運算順序和我們自然的觀點相同,都是先乘除餘、後加減。加、減、乘法幾乎與普通的運算無異,除法需特別注意,若資料型態是 \lstinline{int},則除法為整數除法,
\begin{lstlisting}
  cout << 8 / 5 << endl;
\end{lstlisting}
\paragraph{}的結果為:
\begin{lstlisting}
  1
\end{lstlisting}
\paragraph{}若資料型態為 \lstinline{double},則除法為浮點數除法,
\begin{lstlisting}
  cout << 8.0 / 5.0 << endl;
\end{lstlisting}
\paragraph{}的結果則為:
\begin{lstlisting}
  1.6
\end{lstlisting}
\paragraph{}除此之外,除法還有除以零的問題,以下四種 \lstinline{cout} 請各位讀者自行分別執行,會發生什麼問題、產生什麼現象,有興趣的話可以在網路上多查些資料。
\begin{enumerate}
\item \lstinline{cout << 1 / 0 << endl;}
\item \lstinline{cout << 0 / 0 << endl;}
\item \lstinline{cout << 1.0 / 0.0 << endl;}
\item \lstinline{cout << 0.0 / 0.0 << endl;}
\end{enumerate}
\paragraph{}取餘數的運算子會和平常直觀也不太一樣,一般來說 mod m 取出來的餘數會介於 0 ~ n - 1 之間,但是在負數下,取出來的餘數也會是負數,例如:
\begin{lstlisting}
  cout << -5 % 3 << endl;
\end{lstlisting}
\paragraph{}結果會是:
\begin{lstlisting}
  -2
\end{lstlisting}
\paragraph{}如果我們要做出我們常見 n mod m 的效果,就利用下列方法:
\begin{lstlisting}
  cout << ((n % m) + m) % m << endl;
\end{lstlisting}
\paragraph{}因為我們不知道 n 是正整數還是負整數,因此我們先取 mod m,保證它如果是正整數時,它的值會在 0 到 m-1 之間;如果是負整數的話,它的值就會在 -(m-1) 到 0 之間。
\paragraph{}接著加完 m 之後再取一次 mod m,對於正整數而言雖然結果沒有影響;但是對於負整數,在加完 m 之後就會變成正的餘數,再下來 mod m 對結果沒有影響,但要是萬一 n 是正整數的話,在加完 m 之後就會大於 m,因此 mod m 這個步驟是針對正整數來做的。
\label{UVa:10071}
\paragraph{}UVa 10071: Back to High School Physics
\paragraph{}這題只要能夠讀懂題意都不難寫。如果不知道怎樣讀取多筆測資請先參考迴圈部分。

\section{比較、邏輯運算子}

\paragraph{}C++ 的比較運算子有六種:
\paragraph{}注意!判斷等於是「雙等號」!若讀者在一開始寫程式與預期不同時,不妨檢查一下等於是否寫錯寫成指定 (單等號)了。比較運算子的回傳值為「真、假」兩種,在 C++ 中以「非零」當做「真」,「零」做為「假」。因此假設我們判斷 a mod b 是否「非零」時,一般的寫法如下:
\begin{lstlisting}
  a % b != 0
\end{lstlisting}
\paragraph{}但是我們可以簡化為:
\begin{lstlisting}
  a % b
\end{lstlisting}
\paragraph{}原因是當 \lstinline{a % b} 非零時,一般 \lstinline{a % b != 0} 的寫法會回傳「真」,但非零會被 C++ 當做「真」來看待;同樣地,當 \lstinline{a % b} 為零時,一般的寫法會回傳「假」,但為零會被 C++ 當做「假」。因此簡化過的寫法,大多時候可以取代原來一般寫法。
\paragraph{}C++ 邏輯運算子有三種:
\paragraph{}如果我們要判斷一個整數 x 是否介於整數 a 和 b 之間,數學上我們會寫成:
\begin{lstlisting}
  a <= x <= b
\end{lstlisting}
\paragraph{}但程式不能這樣子寫--這段程式是無法編譯的!原因是因為比較運算子是二元運算子,二元運算子大多是將運算子兩邊的運算元做運算後,在回傳一個結果,例如下面這個算術運算式:
\begin{lstlisting}
  1 + 2 + 3
\end{lstlisting}
\paragraph{}這個式子中因為 + 法的結合性是從左到右,因此這個運算式會先算 \lstinline{1 + 2},回傳 \lstinline{3},變成:
\begin{lstlisting}
  3 + 3
\end{lstlisting}
\paragraph{}再算 \lstinline{3 + 3},回傳 \lstinline{6}。如果是更一般的四則運算:
\begin{lstlisting}
  1 + 2 * 3
\end{lstlisting}
\paragraph{}這個式子會因為 \lstinline{*} 法的優先順序高於 \lstinline{+} 法,因此會先算 \lstinline{2 * 3},回傳 \lstinline{6}:
\begin{lstlisting}
  1 + 6
\end{lstlisting}
\paragraph{}再算 \lstinline{1 + 6},回傳 \lstinline{7}。同樣的道理,如果我們的比較運算式寫成如下:
\begin{lstlisting}
  a <= x <= b
\end{lstlisting}
\paragraph{}那麼我們會「先算」 \lstinline{a <= x},因為對於我們人類來說,我們可以很容易知道這是一個判斷的式子,但是電腦不能,電腦只會認為這一切都是「計算」--意即抓出一個運算元和數個運算子,做運算之後,再丟回去結果--因此電腦依照「運算順序」和「結合性」來做計算,如果 \lstinline{a <= x} 是對的,就會傳「真」,如果不是就回傳「假」,但無論真還是假都無法和 b 做比較,因此數學上的寫法在電腦上沒辦法直接表達原來我們想要的意思。
\paragraph{}因此我們就將這個判斷式做改變,既然要判斷 x 是否在 a 和 b 之間,我們可以改寫成:\lstinline{a <= x} 而且 \lstinline{x <= b},符合二元運算的特性,而且在直觀上也不偏離原意,這正好是電腦可以接受的寫法:
\begin{lstlisting}
  a <= x && x <= b
\end{lstlisting}
\paragraph{}同樣我們的邏輯運算式有「非」這個選項,因此也有一個小技巧來判斷 a mod b 是否「為零」,原本的寫法如下:
\begin{lstlisting}
  a % b == 0
\end{lstlisting}
\paragraph{}利用 \lstinline{!} 運算子,我們判斷 a mod b 是否非零後,再反向,換句話說就是把非零判斷為「真」的選項反過來變成「假」;「假」反過來變成「真」:


\ifx \allfiles \undefined
\printindex[noun]

\clearpage
\end{CJK}
\end{document}
\fi
\documentclass[utf8]{beamer}

%% === CJK 套件 ===
\usepackage{CJKutf8,CJKnumb}                 % 中文套件
%% === AMS 標準套件 ===
\usepackage{amsmath,amsfonts,amssymb,amsthm} % 數學符號
\usepackage{ulem}
%% ===  ===
%\usepackage[chapter]{algorithm}              % 演算法套件
%\usepackage[noend]{algpseudocode}            % pseudocode 套件
\usepackage{listings}                        % 程式碼
%% === TikZ 套件 ===
\usepackage{tikz,tkz-graph,tkz-berge}        % 繪圖
\usepackage{multicol}
\usepackage{xkeyval,xargs}
\usepackage{xcolor}

\usetheme{Boadilla}
\usecolortheme{whale}

\setbeamertemplate{items}[circle]

%% === 設定 C++ 格式 ===
\lstset{%
  language=C++,                     % 設定語言
  %% === 空白, tab 相關 ===
  tabsize=2,                            % 設定 tab = 多少空白
  %showspaces=true,                      % 設定是否標示空白
  %showtabs=true,                        % 設定是否標示 tab
  %tab=\rightarrowfill,                  % 設定 tab 樣式
  %% === 行數相關 ===
  %numbers=left,                         % 行數標示位置
  %stepnumber=1,                         % 每隔幾行標示行數
  %numberstyle=\tiny,
  %% === 顏色設定 ===
  basicstyle=\ttfamily,
  keywordstyle=\color{blue}\ttfamily,
  stringstyle=\color{red!50!brown}\ttfamily,
  commentstyle=\color{green!50!black}\ttfamily,
  %identifierstyle=\color{black}\ttfamily,
  emphstyle=\color{purple}\ttfamily,
  extendedchars=false,
  texcl=true,
  moredelim=[l][\color{magenta}]{\#}
}

\begin{document}
\begin{CJK}{UTF8}{bkai}

\title{基礎程式設計技巧(三)\\字元與字串}
\author{許胖}
\institute[PCSH]{板燒高中}

\begin{frame}
  \titlepage
\end{frame}
\begin{frame}
  \frametitle{大綱}
  \begin{multicols}{2}
    \tableofcontents
  \end{multicols}
\end{frame}

\section{簡介}
\begin{frame}
  \frametitle{大綱}
  \begin{multicols}{2}
    \tableofcontents[currentsection]
  \end{multicols}
\end{frame}

\section{預處理器}
\begin{frame}
  \frametitle{大綱}
  \begin{multicols}{2}
    \tableofcontents[currentsection]
  \end{multicols}
\end{frame}

\subsection{\#include}

\begin{frame}
  \frametitle{\#include}
\end{frame}

\subsection{\#define}

\section{字元與字串}
\begin{frame}
  \frametitle{大綱}
  \begin{multicols}{2}
    \tableofcontents[currentsection]
  \end{multicols}
\end{frame}

\subsection{字元與 ASCII}

\begin{frame}[fragile]
  \frametitle{字元}
  \begin{exampleblock}{字元}
    \begin{itemize}[<+->]
    \item 字元:一個位元組長,是介於 -128 到 127 的\alert{整數}
      \begin{itemize}
      \item 保留字為 \lstinline{char}{}
      \end{itemize}
    \item ASCII:一個編碼表,將字元的編碼對應到一個符號
    \item 表示法
      \begin{itemize}
      \item 單引號夾著一個符號:\lstinline{'0'}{}
      \item 或者是字元的 ASCII 編碼 (8 進位):\lstinline{'\60'}{} \onslide<+->{$\Rightarrow$ \lstinline{'0'}{}}
      \end{itemize}
    \end{itemize}
  \end{exampleblock}
  \begin{columns}[T]
    \begin{column}[T]{5.5cm}
    \begin{alertblock}{ASCII 特性}<+->
      \begin{itemize}
      \item 某些字元是連續編碼
        \begin{itemize}[<+->]
        \item \lstinline{'0'}{} 到 \lstinline{'9'}{}
        \item \lstinline{'A'}{} 到 \lstinline{'Z'}{}
        \item \lstinline{'a'}{} 到 \lstinline{'z'}{}
        \end{itemize}
      \end{itemize}
    \end{alertblock}
    \end{column}
    \begin{column}[T]{5.5cm}
    \begin{block}{常見的 ASCII 編碼}<+->
    \begin{multicols}{2}
      \begin{itemize}[<+->]
      \item \lstinline{'\0'}{}:0
      \item \lstinline{'\n'}{}:32
      \item \lstinline{'0'}{}:48
      \item \lstinline{'A'}{}:65
      \item \lstinline{'a'}{}:97
      \end{itemize}
    \end{multicols}
    \end{block}
    \end{column}
  \end{columns}
\end{frame}

\begin{frame}[fragile]
  \frametitle{判斷英數字}
  \begin{exampleblock}{方法}<+->
  利用字元連續性,例如判斷數字:
  \onslide<+->
    \begin{lstlisting}
  bool isMyDigit(char ch) {
    return ('0' <= ch) && (ch <= '9');
    // 檢查是否在此區間
  }
    \end{lstlisting}
  \end{exampleblock}
  \begin{block}{內建函數}<+->
  在 \lstinline{<cctype>}{} 中,判斷是否為
  \begin{multicols}{2}
    \begin{itemize}[<+->]
    \item \lstinline{isalnum(char)}{}:英數字
    \item \lstinline{isalpha(char)}{}:英文字母
    \item \lstinline{isdigit(char)}{}:數字
    \item \lstinline{islower(char)}{}:小寫字母
    \item \lstinline{isupper(char)}{}:大寫字母
    \item etc.
    \end{itemize}
  \end{multicols}
  \end{block}
\end{frame}

\begin{frame}[fragile]
  \frametitle{字元轉數字}
  \begin{itemize}[<+->]
  \item 利用\alert{連續編碼}的特性,加上 \lstinline{char}{} 是\alert{存整數},可以將字元轉數字
  \item 常用到爛掉的技巧
  \end{itemize}
  \begin{exampleblock}{範例:字元轉數字}<+->
    \begin{lstlisting}
  int charToDigit(char ch) {
    return ch - '0'; // 直接減 '0',不需特別記 48
  }
    \end{lstlisting}
  \end{exampleblock}
  \begin{alertblock}{註}<+->
    \begin{itemize}
    \item 同樣的道理可以套用在大小寫字母中。
    \item<+-> 要是今天要轉換的\alert{字串}是 \lstinline{"10"}{} 呢?這方法還會湊效嗎?
    \end{itemize}
  \end{alertblock}
\end{frame}

\begin{frame}[fragile]
  \frametitle{字元的輸入}
  \begin{block}{輸入字元}<+->
    \begin{lstlisting}
  char a, b;
  cin >> a >> b;
  cout << "X" << a << "D" << b << "D" << endl;
    \end{lstlisting}
  \end{block}
  \begin{exampleblock}{觀察}<+->
  輸入「1 2」會發生什麼事?\onslide<+->{ \alert{Ans: b 會讀到空白}}
  \end{exampleblock}
  \begin{alertblock}{注意}<+->
  輸入字元時,會讀入\alert{空白}和\alert{換行}。
  \end{alertblock}
\end{frame}

\subsection{字串}

\begin{frame}[fragile]
  \frametitle{C 與 C++ 字串}
  \begin{columns}[T]
    \begin{column}[T]{5.5cm}
    \begin{block}{C++ 字串}<+->
      \begin{itemize}[<+->]
      \item 在 \lstinline{<string>}{} 中。
      \item 是一個類別。
      \end{itemize}
    \end{block}
    \end{column}
    \begin{column}[T]{5.5cm}
    \begin{exampleblock}{C 字串}<+->
      \begin{itemize}
      \item 本質上是字元陣列
      \item<+-> 宣告時就是字元陣列
      \end{itemize}
    \end{exampleblock}
    \end{column}
  \end{columns}
  \begin{alertblock}{字串宣告}<+->
    \begin{multicols}{2}
    \begin{itemize}[<+->]
      \item \lstinline{string str;}{}
      \item \lstinline{char str[110];}{}
    \end{itemize}
    \end{multicols}
  \end{alertblock}
  \begin{exampleblock}{常數}<+->
  字串是由兩個雙引號夾著一堆字元:\lstinline{"XD"}{}
  \end{exampleblock}
\end{frame}

\begin{frame}[fragile]
  \frametitle{C 字串儲存狀況}
  \begin{block}{C 字串}<+->
    \begin{lstlisting}
  char str[10] = "bird";
    \end{lstlisting}
  \end{block}
  \begin{exampleblock}{實際儲存狀況}<+->
  \begin{table}[h]
    \begin{tabular}{|c|c|c|c|c|c|c|c|c|c|}
    \hline
    \lstinline{'b'}{} & \lstinline{'i'}{} & \lstinline{'r'}{} & \lstinline{'d'}{} &
    \lstinline{'\0'}{} & 未知 & 未知 & 未知 & 未知 & 未知\\
    \hline
    \end{tabular}
  \end{table}
  \end{exampleblock}
  \begin{alertblock}{注意}<+->
  \lstinline{'\0'}{} 代表\alert{字串的結尾},佔一個字元,因此在使用 C 字串時要多注意空間上的問題。
  \end{alertblock}
\end{frame}

\begin{frame}[fragile]
  \frametitle{字串操作}
  \begin{block}{字串操作}<+->
    \begin{itemize}[<+->]
    \item 取長度
    \item 比較字串
    \item 複製字串
    \item 串接字串
    \item 其他操作
    \end{itemize}
  \end{block}
\end{frame}

\begin{frame}[fragile]
  \frametitle{字串操作:取長度}
  \begin{columns}[T]
    \begin{column}[T]{5.5cm}
    \begin{block}{C++ 字串}<+->
      \begin{itemize}
      \item \lstinline{str.size()}{}
      \end{itemize}
    \end{block}
    \end{column}
    \begin{column}[T]{5.5cm}
    \begin{block}{C 字串}<+->
      \begin{itemize}[<+->]
      \item 在 \lstinline{<cstring>}{} 中
      \item \lstinline{strlen(const char *)}{}
      \end{itemize}
    \end{block}
    \end{column}
  \end{columns}
  \begin{exampleblock}{C++ 字串用法}<+->
    \begin{lstlisting}
  for (int i = 0; i < str.size(); i++) {}
    \end{lstlisting}
  \end{exampleblock}
  \begin{exampleblock}{C 字串用法}<+->
    \begin{lstlisting}
  for (int i = 0; i < strlen(str); i++) {}
    \end{lstlisting}
    \onslide<+->{\alert{盡量少這樣用!}}
  \end{exampleblock}
\end{frame}

\begin{frame}[fragile]
  \frametitle{原理}
  \begin{exampleblock}{原理}<+->
    \begin{itemize}[<+->]
    \item \lstinline{strlen()}{} 是\alert{函數}
      \begin{itemize}
      \item \lstinline{strlen()}{} 計算長度的方法,就是從傳入指標開始數,數到 \lstinline{'\0'}{} 字元為止!
      \item 在沒優化下,每次 \lstinline{for}{} 迴圈判斷時都會執行 \lstinline{strlen()}{} 一次!
      \item 在字串長度不變下,這個方法會讓程式變慢
      \end{itemize}
    \item \lstinline{str.size()}{} 是\alert{取值}
      \begin{itemize}
      \item 字串本身就會把長度算好存起來,因此沒這問題。
      \end{itemize}
    \end{itemize}
  \end{exampleblock}
  \begin{alertblock}{改良}<+->
  宣告變數儲存長度。
  \onslide<+->
  \begin{lstlisting}
  int len = strlen(str);
  for (int i = 0; i < len; i++) {}
  \end{lstlisting}
  \end{alertblock}
\end{frame}

\begin{frame}[fragile]
  \frametitle{字串操作:比較字串}
  \begin{columns}[T]
    \begin{column}[T]{5.5cm}
    \begin{block}{C++ 字串}<+->
      \begin{itemize}
      \item \lstinline{strA == strB}{}
      \end{itemize}
    \end{block}
    \end{column}
    \begin{column}[T]{5.5cm}
    \begin{block}{C 字串}<+->
      \begin{itemize}[<+->]
      \item 在 \lstinline{<cstring>}{} 中
      \item \lstinline{strcmp(strA, strB)}{}
      \end{itemize}
    \end{block}
    \end{column}
  \end{columns}
  \begin{alertblock}{註}<+->
    \begin{lstlisting}
  strcmp(const char *, const char *);
    \end{lstlisting}
  \end{alertblock}
  \begin{exampleblock}{原理}<+->
  比較 \lstinline{strA}{} 和 \lstinline{strB}{} 每個字元,直到某一邊先\alert{碰到} \lstinline{'\0'}{} 為止。
  \end{exampleblock}
\end{frame}

\begin{frame}[fragile]
  \frametitle{差異}
  \begin{columns}[T]
    \begin{column}[T]{5.5cm}
    \begin{block}{C++ 字串}
      \begin{itemize}
      \item \lstinline{strA == strB}{}
      \end{itemize}
    \end{block}
    \end{column}
    \begin{column}[T]{5.5cm}
    \begin{block}{C 字串}
      \begin{itemize}
      \item 在 \lstinline{<cstring>}{} 中
      \item \lstinline{strcmp(strA, strB)}{}
      \end{itemize}
    \end{block}
    \end{column}
  \end{columns}
  \onslide<+->
  \begin{exampleblock}{回傳值}<+->
    \begin{itemize}[<+->]
    \item C++ 字串如果\alert{相等}回傳 \lstinline{true}{},否則回傳 \lstinline{false}{}。
    \item \lstinline{strcmp()}{} 函數原理類似\alert{兩字串相減}
      \begin{enumerate}
      \item \alert{相同回傳 0}
      \item \lstinline{strA}{} 的\alert{字典順序}比 \lstinline{strB}{} 大,回傳正值 (通常是 1)
      \item \lstinline{strA}{} 的\alert{字典順序}比 \lstinline{strB}{} 小,回傳負值 (通常是 -1)
      \end{enumerate}
    \end{itemize}
  \end{exampleblock}
\end{frame}

\begin{frame}[fragile]
  \frametitle{字串操作:複製字串}
  \begin{columns}[T]
    \begin{column}[T]{5.5cm}
    \begin{block}{C++ 字串}<+->
      \begin{itemize}
      \item \lstinline{dest = src}{}
      \end{itemize}
    \end{block}
    \end{column}
    \begin{column}[T]{5.5cm}
    \begin{block}{C 字串}<+->
      \begin{itemize}[<+->]
      \item 在 \lstinline{<cstring>}{} 中
      \item \lstinline{strcpy(dest, src)}{}
      \end{itemize}
    \end{block}
    \end{column}
  \end{columns}
  \begin{alertblock}{註}<+->
    \begin{lstlisting}
  strcpy(char *dest, const char *src);
  strncpy(char *dest, const char *src, size_t n);
    \end{lstlisting}
  \end{alertblock}
  \begin{exampleblock}{原理}<+->
    \begin{itemize}
    \item 將每個 \lstinline{src}{} 的字元一一複製到 \lstinline{dest}{} 中,直到\alert{碰到} \lstinline{'\0'}{} 為止。
    \end{itemize}
  \end{exampleblock}
\end{frame}

\begin{frame}[fragile]
  \frametitle{字串操作:複製字串}
  \begin{alertblock}{註}<+->
    \begin{lstlisting}
  strcpy(char *dest, const char *src);
  strncpy(char *dest, const char *src, size_t n);
    \end{lstlisting}
  \end{alertblock}
  \begin{exampleblock}{注意}<+->
    \begin{itemize}[<+->]
    \item 因為複製字串時,會一直複製\alert{直到}碰到 \lstinline{'\0'}{} 為止。
    \item 當 \lstinline{dest}{} 長度太短,\lstinline{strcpy()}{} 就會寫出 \lstinline{dest}{},造成 Runtime Error。
    \item 避免這種情況可以使用 \lstinline{strncpy()}{}
      \begin{itemize}
      \item 第三個參數代表複製的字元數。
      \item \alert{注意!}使用 \lstinline{strncpy()}{} 在 \lstinline{dest}{} 後面不會補 \lstinline{'\0'}{}, 因此要記得補。
      \end{itemize}
    \end{itemize}
  \end{exampleblock}
\end{frame}

\begin{frame}[fragile]
  \frametitle{字串操作:串接字串}
  \onslide<+->
  \begin{figure}[h]
    \caption{4 隻 pusheen}
    \centering\includegraphics[width=0.26\textwidth]{img/box.png}
  \end{figure}
  \onslide<+->
  \begin{figure}[h]
    \caption{strcat(pusheen)}
    \centering\includegraphics[width=0.6\textwidth]{img/box_long.jpg}
  \end{figure}
\end{frame}

\begin{frame}[fragile]
  \frametitle{字串操作:串接字串}
  \begin{columns}[T]
    \begin{column}[T]{5.5cm}
    \begin{block}{C++ 字串}<+->
      \begin{itemize}
      \item \lstinline{dest += src}{}
      \end{itemize}
    \end{block}
    \end{column}
    \begin{column}[T]{5.5cm}
    \begin{block}{C 字串}<+->
      \begin{itemize}[<+->]
      \item 在 \lstinline{<cstring>}{} 中
      \item \lstinline{strcat(dest, src)}{}
      \end{itemize}
    \end{block}
    \end{column}
  \end{columns}
  \begin{alertblock}{註}<+->
    \begin{lstlisting}
  strcat(char *dest, const char *src);
  strncat(char *dest, const char *src, size_t n);
    \end{lstlisting}
  \end{alertblock}
  \begin{exampleblock}{原理}<+->
    \begin{itemize}
    \item 從 \lstinline{dest}{} \alert{第一個} \lstinline{'\0'}{} 開始串接,直到碰到 \lstinline{src}{} 的 \lstinline{'\0'}{} 為止,與 \lstinline{strcpy}{} 有同樣的問題。
    \end{itemize}
  \end{exampleblock}
\end{frame}

\begin{frame}[fragile]
  \frametitle{其他字串操作}
  \begin{block}{其他字串操作}<+->
    \begin{itemize}[<+->]
    \item \lstinline{strstr()}{}
    \item \lstinline{strchr()}{}
    \item \lstinline{strrchr()}{}
    \item \lstinline{strtok()}{}
    \end{itemize}
  \end{block}
  \begin{exampleblock}{說明}<+->
    \begin{itemize}
    \item 概念較複雜,或者用自己的方法處理就好。
    \item<+-> 讀者有興趣可以自己查資料。
    \end{itemize}
  \end{exampleblock}
\end{frame}

\begin{frame}[fragile]
  \frametitle{字串小結}
  \begin{table}[h]
    \caption{常見字串操作}
    \begin{tabular}{c|c|c}
            & C++ 字串                   & C 字串\onslide<+->\\
    \hline\hline
    取長度   & \lstinline{str.size()}{}  & \lstinline{strlen(str)}{}\onslide<+->\\
    \hline
    比較字串 & \lstinline{strA == strB}{} & \lstinline{strcmp(strA, strB)}{}\onslide<+->\\
    \hline
    複製字串 & \lstinline{dest = src}{}   & \lstinline{strcpy(dest, src)}{}\onslide<+->\\
    \hline
    串接字串 & \lstinline{dest += src}{}  & \lstinline{strcat(dest, src)}{}
    \end{tabular}
  \end{table}
  \begin{exampleblock}{差異}<+->
    \begin{itemize}[<+->]
    \item C++ 字串是\alert{類別},C 字串是\alert{字元陣列}
    \item C++ 字串操作多是\alert{運算子}或\alert{成員函數},C 字串是\alert{函數}
    \item C 字串處理速度較 C++ 來得\alert{快},但要注意會不會超出\alert{陣列範圍}以及補 \lstinline{'\0'}{}
    \end{itemize}
  \end{exampleblock}
\end{frame}

\subsection{字串轉換}

\begin{frame}[fragile]
  \frametitle{C++ 字串轉 C 字串}
  \begin{block}{成員函數}<+->
    \begin{itemize}
    \item 在 \lstinline{<string>}{} 中
    \item<+-> \lstinline{str.c_str()}{}
    \end{itemize}
  \end{block}
  \begin{exampleblock}{範例}<+->
    \begin{lstlisting}
  string str = "bird";
  cout << str.c_str() << endl;
    \end{lstlisting}
  \end{exampleblock}
  \begin{alertblock}{用途}<+->
    \begin{itemize}
    \item C 語言的函數大多只能用 C 字串來做,因此我們可以藉由這個函數把 \lstinline{string}{} 轉成 C 字串。
    \end{itemize}
  \end{alertblock}
\end{frame}

\begin{frame}[fragile]
  \frametitle{C 字串轉 C++ 字串}
  \begin{exampleblock}{範例}<+->
    \begin{lstlisting}
  char strA[110] = "bird";
  string strB = strA; // 直接丟進去
  cout << strB << endl;
  cout << strB.size() << endl;
    \end{lstlisting}
  \end{exampleblock}
\end{frame}

\begin{frame}[fragile]
  \frametitle{字串練習}
  \begin{block}{練習一}<+->
    \begin{lstlisting}
  char str1[110] = "abcdefghijklmnopqrstuvwxyz";
    \end{lstlisting}
  長度為何?\onslide<+-> 用 \lstinline{sizeof}{} 和 \lstinline{strlen}{} 有何不同呢?\onslide<+-> 做完上面的問題後,試著思考下面的語句又和上面的問題有什麼差異呢?
    \begin{itemize}[<+->]
    \item \lstinline{strlen("abcdefghijklmnopqrstuvwxyz");}{}
    \item \lstinline{sizeof("abcdefghijklmnopqrstuvwxyz");}{}
    \end{itemize}
  \end{block}
  \begin{exampleblock}{提示}<+->
  「記憶體」和「字串」之間的差別與關係。
  \end{exampleblock}
\end{frame}

\begin{frame}[fragile]
  \frametitle{字串練習}
  \begin{block}{練習二}<+->
    \begin{lstlisting}
  char str2[110] = "bird";  
  char str3[4];
    \end{lstlisting}
    \begin{itemize}[<+->]
    \item 我們可以用 \lstinline{strcpy(str3, str2);}{} 嘛?
    \item 如果不行,若我們使用 \lstinline{strncpy(str3, str2, 4);}{} 來避免超出記憶體呢?試著 \lstinline{cout}{} 出來觀察?
    \end{itemize}
  \end{block}
  \begin{exampleblock}{提示}<+->
  \lstinline{strcpy}{} 的用法。
  \end{exampleblock}
\end{frame}

\begin{frame}[fragile]
  \frametitle{字串練習}
  \begin{block}{練習三}
    \begin{itemize}[<+->]
    \item
      \begin{lstlisting}
  char str4[110] = "cat\0bird";
      \end{lstlisting}
      請問這個字串的長度為何?\onslide<+-> 若是 \lstinline{strlen(str4 + 4)}{} 的回傳值為何?\onslide<+-> \lstinline{strlen(str4 + 8)}{} 呢?
    \item 今天我們使用 \lstinline{strcpy(str4, "dog")}{},會得到什麼結果?\onslide<+-> 此時印出 \lstinline{str4}{} 和 \lstinline{str4 + 4}{} 會有什麼反應呢?
    \item 如果我們對原先的 \lstinline{str4}{} 執行 \lstinline{strcat(str4, "dog")}{},有得到你預期的結果嘛?
    \item 要是 \lstinline{strcat(str4 + 4, "dog")}{} 呢? 
    \end{itemize}
  \end{block}
  \begin{exampleblock}{提示}<+->
  \lstinline{'\0'}{} 的用途和指標的用法。
  \end{exampleblock}
\end{frame}

\begin{frame}[fragile]
  \frametitle{字串練習}
  \begin{block}{練習四}<+->
    \begin{lstlisting}
  char str7[110] = "cat\0bird";
  char str8[110];
    \end{lstlisting}
  試比較下列兩者的不同
    \begin{itemize}
    \item \lstinline{strcpy(str8, str7);}{}
    \item \lstinline{memcpy(str8, str7, sizeof(str7));}{}
    \end{itemize}
  \end{block}
  \begin{exampleblock}{提示}<+->
  比較「記憶體」和「字串」的差異。
  \end{exampleblock}
\end{frame}

\section{格式字串}
\begin{frame}
  \frametitle{大綱}
  \begin{multicols}{2}
    \tableofcontents[currentsection]
  \end{multicols}
\end{frame}

\subsection{輸入輸出}

\subsection{scanf 和 printf}

\section{檔案輸入}

\subsection{標準輸入}

\section{羅馬數字}
\begin{frame}
  \frametitle{大綱}
  \begin{multicols}{2}
    \tableofcontents[currentsection]
  \end{multicols}
\end{frame}

\begin{frame}[fragile]
  \frametitle{練習}
  \begin{exampleblock}{ZeroJudge a013: 羅馬數字}<+->
  給你兩個羅馬數字,做相減後輸出羅馬數字。
  \end{exampleblock}
  \begin{exampleblock}{ZeroJudge d251: 94北縣賽-3-羅馬數字 (Roman)}<+->
  給你羅馬數字表示的時間,輸出加上 7 小時 30 分時差後的羅馬數字時刻。
  \end{exampleblock}
\end{frame}

\begin{frame}[fragile]
  \frametitle{練習}
  \begin{exampleblock}{ZeroJudge d369: 1. 羅馬數字 / UVa 11616: Roman Numerals}<+->
  給你羅馬數字,轉換為阿拉伯數字;給你阿拉伯數字,轉換為羅馬數字。
  \end{exampleblock}
  \begin{exampleblock}{UVa 344 / Ruby 344: Roman Digititis}<+->
  給你 n,輸出 n 以內需要用到多少個羅馬數字符號?
  \end{exampleblock}
\end{frame}

\subsection{阿拉伯數字轉羅馬數字}

\subsection{羅馬數字轉阿拉伯數字}

\section{字串和數字轉換}
\begin{frame}
  \frametitle{大綱}
  \begin{multicols}{2}
    \tableofcontents[currentsection]
  \end{multicols}
\end{frame}

\subsection{字串轉數字}

\subsection{數字轉字串}

\clearpage
\end{CJK}
\end{document}
\documentclass[utf8]{beamer}

%% === CJK 套件 ===
\usepackage{CJKutf8,CJKnumb}                 % 中文套件
%% === AMS 標準套件 ===
\usepackage{amsmath,amsfonts,amssymb,amsthm} % 數學符號
\usepackage{ulem}
%% ===  ===
%\usepackage[chapter]{algorithm}              % 演算法套件
%\usepackage[noend]{algpseudocode}            % pseudocode 套件
\usepackage{listings}                        % 程式碼
%% === TikZ 套件 ===
\usepackage{tikz,tkz-graph,tkz-berge}        % 繪圖
\usepackage{multicol}
\usepackage{xkeyval,xargs}
\usepackage{xcolor}

\usetheme{Boadilla}
\usecolortheme{whale}

\setbeamertemplate{items}[circle]

%% === 設定 C++ 格式 ===
\lstset{%
  language=C++,                     % 設定語言
  %% === 空白, tab 相關 ===
  tabsize=2,                            % 設定 tab = 多少空白
  %showspaces=true,                      % 設定是否標示空白
  %showtabs=true,                        % 設定是否標示 tab
  %tab=\rightarrowfill,                  % 設定 tab 樣式
  %% === 行數相關 ===
  numbers=left,                         % 行數標示位置
  stepnumber=1,                         % 每隔幾行標示行數
  numberstyle=\tiny,
  %% === 顏色設定 ===
  basicstyle=\ttfamily,
  keywordstyle=\color{blue}\ttfamily,
  stringstyle=\color{red!50!brown}\ttfamily,
  commentstyle=\color{green!50!black}\ttfamily,
  identifierstyle=\color{black}\ttfamily,
  emphstyle=\color{purple}\ttfamily,
  extendedchars=false,
  texcl=true,
  moredelim=[l][\color{magenta}]{\#}
}

\begin{document}
\begin{CJK}{UTF8}{bkai}

\title{基礎程式設計技巧(一)\\程式與計算}
\author{許胖}
\institute[PCSH]{板燒高中}

\begin{frame}
  \titlepage
\end{frame}

\section{簡介}

\begin{frame}
  \frametitle{寫程式的差別}
  \begin{block}{寫出一個完整的程式 ...}<2->
    \begin{itemize}[<3->]
    \item 只要照著講義、照著書打一打,就可以動了。
    \end{itemize}
  \end{block}
  \begin{exampleblock}{寫「好」一個程式 ...}<4->
    \begin{enumerate}
    \item<5-> 要了解資料怎麼儲存在電腦中
    \item<6-> 程式怎麼開始執行,為什麼會執行
    \item<7-> 什麼時候會岀什麼狀況,然後判斷出來、修正 (也就是 debug)
    \item<8-> 用適當的工具解決問題
    \item<9-> ... 族繁不及\sout{被宰}備載
    \end{enumerate}
  \end{exampleblock}
  \begin{itemize}
  \item<10-> 以上就是快樂寒訓營的\alert{培訓目標}!
  \item<11-> 也就是要讓大家熟悉\alert{基本的 C++ 語法},以及學會基本的 \alert{coding 技巧}。
  \end{itemize}
\end{frame}

\begin{frame}
  \frametitle{關於許胖講義 ...}
  \begin{alertblock}{給參與「演算法競賽」的人 ...}<2->
    \begin{enumerate}
    \item<3-> 使用一個「有效」的方法解決問題
    \item<4-> 不僅如此,還要知道不同工具使用上的優缺點
    \item<5-> 手爆出很多 code,勇往直前
    \item<6-> 進到 TOI 二階,保送大學
    \end{enumerate}
  \end{alertblock}
  \begin{itemize}
    \item<7-> 寫程式不是只有演算法比賽,生命也不是只有一個出口
    \item<7-> 越往這個領域深入,就會看到更多無盡的事物
    \begin{itemize}
      \item<8-> 寫遊戲引擎
      \item<9-> 網頁設計
      \item<10-> 手機 App
      \item<11-> 韌體 coding
    \end{itemize}
  \end{itemize}
  \begin{exampleblock}{XD}<12->
  祝各位接下來一周快樂寫程式 XD!
  \end{exampleblock}
\end{frame}

\begin{frame}
  \frametitle{大綱}
  \begin{multicols}{2}
    \tableofcontents
  \end{multicols}
\end{frame}

\section{程式架構}
\begin{frame}
  \frametitle{大綱}
  \begin{multicols}{2}
    \tableofcontents[currentsection]
  \end{multicols}
\end{frame}

\subsection{基本程式架構}

\begin{frame}[fragile]
  \frametitle{基本程式架構}
  \begin{block}{C++ 基本架構}
    \pause
    \begin{lstlisting}
    #include <iostream>
    using namespace std;
    int main() {
    }
    \end{lstlisting}
  \end{block}
  \begin{exampleblock}{註}<3->
    \begin{itemize}
    \item<3-> 怎麼理解?
      \begin{itemize}[<4->]
      \item 不需要理解,我們先記起來。
      \end{itemize}
    \item<5-> 我們的程式都寫在\alert{大括號}中。
    \item<6-> 裡面每個符號都要一樣 (分號也是)。
    \end{itemize}
  \end{exampleblock}
\end{frame}

\subsection{輸出}

\begin{frame}[fragile]
  \frametitle{程式的輸出}
  \begin{block}{輸出}
    \begin{enumerate}
    \item<1-> 試著在剛剛的大括號中打上「\lstinline{cout << 1;}」。
    \item<3-> 如果只打上「\lstinline{cout << 1}」(去掉分號) 會發生什麼結果?
    \item<5-> 試試看「\lstinline{cout << 1 << 2;}」,和你所想的有何不同?
    \item<7-> 那麼「\lstinline{cout << 1 << " " << 2;}」呢?
    \end{enumerate}
  \end{block}
  \begin{exampleblock}{註}
    \begin{itemize}
    \item<2-> \lstinline{cout} 是「輸出」符號,你要輸出的東西用「\lstinline{<<}」串連。
    \item<4-> 「分號」對 C++ 而言代表「一個句子的結束」,因此當一行\alert{指令結束就要加分號}。
    \item<6-> \lstinline{<<} 可以串很多東西一起輸出。
    \item<7-> \lstinline{" "} 是雙引號中間夾著一個「空白」,\alert{要注意}!
    \end{itemize}
  \end{exampleblock}
\end{frame}

\begin{frame}[fragile]
  \frametitle{換行「符號」}
  \begin{block}{換行}
    \begin{enumerate}[<+->]
    \item 試試看「\lstinline{cout << 1 << 2 << endl;}」,和「\lstinline{cout << 1 << 2;}」有什麼不同呢?
    \item 如果看不出來,試試看「\lstinline{cout << 1 << endl << 2;}」。
    \end{enumerate}
  \end{block}
  \begin{exampleblock}{註}<+->
    \begin{itemize}
    \item 「\lstinline{endl}」代表\alert{換行}符號,輸出中很好用。
    \end{itemize}
  \end{exampleblock}
\end{frame}

\subsection{變數}

\begin{frame}[fragile]
  \frametitle{變數}
  \begin{block}{變數}
    \begin{itemize}[<+->]
    \item 和數學「變數」的概念不太一樣
    \item 程式的變數像是「\alert{容器}」,可以裝資料。
    \item C++ 裡,每個容器都要先講好\alert{用途},這個步驟叫做「\alert{宣告}」。
    \end{itemize}
  \end{block}
  \begin{alertblock}{宣告變數}<+->
  \begin{lstlisting}
  int x;
  \end{lstlisting}
  \end{alertblock}
  \begin{exampleblock}{註}<+->
    \begin{itemize}[<+->]
    \item 宣告就是幫變數取名字,此例將變數取名為「\lstinline{x}」。
    \item 「\lstinline{int}」代表的意義是「\alert{整數}」,規定變數 \lstinline{x} \alert{只能裝整數}。
    \end{itemize}
  \end{exampleblock}
\end{frame}

\begin{frame}[fragile]
  \frametitle{變數的功用}
  \begin{block}{把數字裝到變數}<+->
  \begin{lstlisting}
  #include <iostream>
  using namespace std;
  int main() {
    int x;             // 宣告變數 x
    x = 5;             // 把整數 5 裝進 x 裡面
    cout << x << endl; // 印出變數 x 存的值
  }
  \end{lstlisting}
  \end{block}
  \begin{exampleblock}{註}<+->
    \begin{itemize}
    \item 「\lstinline{int}」宣告變數可以裝整數之外,還有很多不同的種類,以後會慢慢介紹
    \item<+-> 「\lstinline{x = 5;}」這行\alert{不要}和數學中的「等於」搞混。
    \end{itemize}
  \end{exampleblock}
\end{frame}

\begin{frame}
  \frametitle{練習看看}
  \begin{block}{練習}
    若把上個投影片「\lstinline{x = 5;}」改成
    \begin{enumerate}[<+->]
    \item 「\lstinline{x = 5.0;}」會發生什麼事?
    \item 「\lstinline{x = 0.5;}」呢?
    \item 那改成「\lstinline{5 = x;}」呢?
    \end{enumerate}
  \end{block}
  \begin{exampleblock}{註}<+->
    \begin{itemize}
      \item 這些練習目的是要讓你了解\alert{問題出現時}的現象,了解出問題的原因才有辦法 debug
      \item<+-> 為什麼會出現這些現象我們繼續下去就知道了
    \end{itemize}
  \end{exampleblock}
\end{frame}

\subsection{輸入}

\begin{frame}[fragile]
  \frametitle{程式的輸入}
  \begin{block}{輸入}
    執行以下程式
    \begin{lstlisting}
    #include <iostream>
    using namespace std;
    int main() {
      int x;
      cin >> x;
      cout << x << endl;
    }
    \end{lstlisting}
    會發生什麼事呢?
  \end{block}
  \begin{block}{練習}<2->
  如果沒發生什麼事,試著輸入「1」再按 enter 鍵,會發生什麼事呢?
  \end{block}
\end{frame}

\begin{frame}
  \frametitle{程式的輸入}
  \begin{block}{練習}
    \begin{itemize}
    \item<1-> 如果在 \lstinline{cout} 那一行後面加上「\lstinline{system("PAUSE");}」,再輸入「1」然後按 enter 呢?
    \item<6-> 如果輸入「5.0」呢?
    \item<7-> 如果輸入「0.5」呢?
    \item<8-> 如果輸入「XD」呢?
    \end{itemize}
  \end{block}
  \begin{exampleblock}{輸入「符號」}<2->
    \begin{itemize}
    \item 「\lstinline{cin}」代表輸入符號,可以輸入後面變數的資料。
      \begin{itemize}
      \item<3-> 此例中,\lstinline{x} 是整數,因此可以\alert{輸入一個整數}。
      \item<4-> \lstinline{cin} 的 \alert{\lstinline{>>}} 不要和 \lstinline{cout} 的 \alert{\lstinline{<<}} 搞混。
      \end{itemize}
    \item<5-> \lstinline{system("PAUSE");} 代表「暫停」的意思。
    \end{itemize}
  \end{exampleblock}
\end{frame}

\section{算術運算子}
\begin{frame}
  \frametitle{大綱}
  \begin{multicols}{2}
    \tableofcontents[currentsection]
  \end{multicols}
\end{frame}

\subsection{運算性質}

\begin{frame}
  \frametitle{算術運算子}
  \begin{table}[h]
  \begin{tabular}{|c|c|c|c|}
  \hline
  算術運算子  & 意義 & 運算順序 & 結合性\\
  \hline
  +         & 加法 & 5       & 左$\rightarrow$右\\
  \hline
  -         & 減法 & 5       & 左$\rightarrow$右\\
  \hline
  *         & 乘法 & 4       & 左$\rightarrow$右\\
  \hline
  /         & 除法 & 4       & 左$\rightarrow$右\\
  \hline
  \%        & 取餘數 & 4       & 左$\rightarrow$右\\
  \hline
  \end{tabular}
  \caption{算術運算子}
  \end{table}
  \begin{exampleblock}{註}<2->
    \begin{itemize}
    \item<2-> 不管\alert{運算順序}和\alert{結合性},一般來說可以用五則運算來理解
    \item<3-> 只不過程式跟數學還是有差距 ...
      \begin{itemize}[<4->]
      \item 這個故事說來話長,我們先舉個簡單的例子吧!
      \end{itemize}
    \end{itemize}
  \end{exampleblock}
\end{frame}

\begin{frame}
  \frametitle{舉個例子}
  \begin{exampleblock}{$1+2+3=?$}
    \begin{itemize}
    \item<2-> 答案:6。
    \item<3-> 為什麼? (謎之音:「什麼為什麼?」)
    \end{itemize}
  \end{exampleblock}
  \begin{block}{定義}<4->
  \alert{二元運算}有一個\alert{運算子}和兩個\alert{運算元},例如:
    \begin{enumerate}
    \item<5-> $1+2$:「$+$」稱為「運算元」,「$1$」和「$2$」稱為運算子 (我們常稱為「被加數」和「加數」)。
    \item<6-> 我們可以知道「加減乘除餘」都是二元運算。
    \end{enumerate}
  \end{block}
  \begin{itemize}[<7->]
  \item Well, 我們回到原來的問題 ...
  \end{itemize}
\end{frame}

\subsection{結合性與運算順序}

\begin{frame}
  \frametitle{回到原來的問題 ...}
  \begin{exampleblock}{$1+2+3=?$}
    \begin{itemize}
    \item \alert{出現大麻煩啦!}
      \begin{itemize}[<2->]
        \item 根據剛剛說的,加法只有兩個運算元,那麼「$1+2+3$」該怎麼辦呢?
      \end{itemize}
    \item<3-> 解法:決定運算的先後順序。例如:
      \begin{enumerate}
        \item<4-> 先算 $1+2=3$,再算 $\alert{3}+3=6$
        \item<5-> 先算 $2+3=5$,再算 $1+\alert{5}=6$
      \end{enumerate}
    \item<6-> 謎之音:「那還不是一樣嘛?廢話= =」
    \end{itemize}
  \end{exampleblock}
  \begin{alertblock}{注意}
  決定先後順序對「電腦」而言\alert{意義重大}!
  \end{alertblock}
\end{frame}

\begin{frame}
  \frametitle{再舉個例子 ...}
  \begin{exampleblock}{$1-2-3=?$}
    \begin{itemize}
    \item<1-> 我們直觀上會先算 $1-2=-1$,再算 $\alert{-1}-3=-4$。
    \item<2-> 因此 C++ 在設計上也會把加減乘除餘的運算順序「設定」成\alert{從左到右算}。
    \end{itemize}
  \end{exampleblock}
  \pause\pause
  \begin{table}[h]
  \begin{tabular}{|c|c|c|c|}
  \hline
  算術運算子  & 意義 & 運算順序 & 結合性\\
  \hline
  +         & 加法 & 5       & 左$\rightarrow$右\\
  \hline
  -         & 減法 & 5       & 左$\rightarrow$右\\
  \hline
  *         & 乘法 & 4       & 左$\rightarrow$右\\
  \hline
  /         & 除法 & 4       & 左$\rightarrow$右\\
  \hline
  \%        & 取餘數 & 4       & 左$\rightarrow$右\\
  \hline
  \end{tabular}
  \caption{算術運算子}
  \end{table}
\end{frame}

\subsection{除零問題}
\subsection{取餘數技巧}

\section{比較和邏輯運算子}
\begin{frame}
  \frametitle{大綱}
  \begin{multicols}{2}
    \tableofcontents[currentsection]
  \end{multicols}
\end{frame}

\subsection{簡化規則}

\subsection{短路運算}

\section{位元運算子}
\begin{frame}
  \frametitle{大綱}
  \begin{multicols}{2}
    \tableofcontents[currentsection]
  \end{multicols}
\end{frame}

\subsection{int 和 long long 的儲存形式}

\subsection{取特定餘數}

\subsection{mask}

\subsection{set bit}

\subsection{消失的數}

\subsection{parity}

\section{指定運算子}

\clearpage
\end{CJK}
\end{document}
\documentclass[utf8]{beamer}

%% === CJK 套件 ===
\usepackage{CJKutf8,CJKnumb}                 % 中文套件
%% === AMS 標準套件 ===
\usepackage{amsmath,amsfonts,amssymb,amsthm} % 數學符號
\usepackage{ulem}
%% ===  ===
%\usepackage[chapter]{algorithm}              % 演算法套件
%\usepackage[noend]{algpseudocode}            % pseudocode 套件
\usepackage{listings}                        % 程式碼
%% === TikZ 套件 ===
\usepackage{tikz,tkz-graph,tkz-berge}        % 繪圖
\usepackage{multicol}
\usepackage{xkeyval,xargs}
\usepackage{xcolor}

\usetheme{Boadilla}
\usecolortheme{whale}

\setbeamertemplate{items}[circle]

%% === 設定 C++ 格式 ===
\lstset{%
  language=C++,                     % 設定語言
  %% === 空白, tab 相關 ===
  tabsize=2,                            % 設定 tab = 多少空白
  %showspaces=true,                      % 設定是否標示空白
  %showtabs=true,                        % 設定是否標示 tab
  %tab=\rightarrowfill,                  % 設定 tab 樣式
  %% === 行數相關 ===
  %numbers=left,                         % 行數標示位置
  %stepnumber=1,                         % 每隔幾行標示行數
  %numberstyle=\tiny,
  %% === 顏色設定 ===
  basicstyle=\ttfamily,
  keywordstyle=\color{blue}\ttfamily,
  stringstyle=\color{red!50!brown}\ttfamily,
  commentstyle=\color{green!50!black}\ttfamily,
  %identifierstyle=\color{black}\ttfamily,
  emphstyle=\color{purple}\ttfamily,
  extendedchars=false,
  texcl=true,
  moredelim=[l][\color{magenta}]{\#}
}

\begin{document}
\begin{CJK}{UTF8}{bkai}

\title{基礎程式設計技巧(一)\\程式與計算}
\author{許胖}
\institute[PCSH]{板燒高中}

\begin{frame}
  \titlepage
\end{frame}
\begin{frame}
  \frametitle{大綱}
  \begin{multicols}{2}
    \tableofcontents
  \end{multicols}
\end{frame}

\section{簡介}
\begin{frame}
  \frametitle{大綱}
  \begin{multicols}{2}
    \tableofcontents[currentsection]
  \end{multicols}
\end{frame}

\begin{frame}
  \frametitle{寫程式的差別}
  \begin{block}{寫出一個完整的程式 ...}<2->
    \begin{itemize}[<3->]
    \item 只要照著講義、照著書打一打,就可以動了。
    \end{itemize}
  \end{block}
  \begin{exampleblock}{寫「好」一個程式 ...}<4->
    \begin{enumerate}
    \item<5-> 要了解資料怎麼儲存在電腦中
    \item<6-> 程式怎麼開始執行,為什麼會執行
    \item<7-> 什麼時候會岀什麼狀況,然後判斷出來、修正 (也就是 debug)
    \item<8-> 用適當的工具解決問題
    \item<9-> ... 族繁不及\sout{被宰}備載
    \end{enumerate}
  \end{exampleblock}
  \begin{itemize}
  \item<10-> 以上就是快樂寒訓營的\alert{培訓目標}!
  \item<11-> 也就是要讓大家熟悉\alert{基本的 C++ 語法},以及學會基本的 \alert{coding 技巧}。
  \end{itemize}
\end{frame}

\begin{frame}
  \frametitle{關於許胖講義 ...}
  \begin{alertblock}{給參與「演算法競賽」的人 ...}<2->
    \begin{enumerate}
    \item<3-> 使用一個「有效」的方法解決問題
    \item<4-> 不僅如此,還要知道不同工具使用上的優缺點
    \item<5-> 手爆出很多 code,勇往直前
    \item<6-> 進到 TOI 二階,保送大學
    \end{enumerate}
  \end{alertblock}
  \begin{itemize}
    \item<7-> 寫程式不是只有演算法比賽,生命也不是只有一個出口
    \item<7-> 越往這個領域深入,就會看到更多無盡的事物
    \begin{itemize}
      \item<8-> 寫遊戲引擎
      \item<9-> 網頁設計
      \item<10-> 手機 App
      \item<11-> 韌體 coding
    \end{itemize}
  \end{itemize}
  \begin{exampleblock}{XD}<12->
  祝各位接下來一周快樂寫程式 XD!
  \end{exampleblock}
\end{frame}

\section{程式架構}
\begin{frame}
  \frametitle{大綱}
  \begin{multicols}{2}
    \tableofcontents[currentsection]
  \end{multicols}
\end{frame}

\subsection{基本程式架構}

\begin{frame}[fragile]
  \frametitle{基本程式架構}
  \begin{block}{C++ 基本架構}
    \pause
    \begin{lstlisting}
  #include <iostream>
  using namespace std;
  int main() {
  }
    \end{lstlisting}
  \end{block}
  \begin{exampleblock}{註}<3->
    \begin{itemize}
    \item 怎麼理解?
      \begin{itemize}[<4->]
      \item 不需要理解,我們先記起來。
      \end{itemize}
    \item<5-> 基本上程式的內容都寫在\alert{大括號}中。
    \item<6-> 裡面每個符號都要一樣 (分號也是)。
    \end{itemize}
  \end{exampleblock}
\end{frame}

\subsection{輸出}

\begin{frame}[fragile]
  \frametitle{程式的輸出}
  \begin{block}{輸出}
    \begin{enumerate}[<+->]
      \item 試著在剛剛的大括號中打上「\lstinline{cout << 1;}{}」,會發生什麼事?
      \item 還不清楚的話,可以在更下一行加上「\lstinline{system("PAUSE");}{}」,在觀察看看。
    \end{enumerate}
  \end{block}
  \begin{exampleblock}{註}<+->
    \begin{itemize}
    \item \lstinline{cout}{} 是「輸出」符號,你要輸出的東西用「\lstinline{<<}{}」串連。
    \item<+-> \lstinline{system("PAUSE");}{} 代表「暫停」的意思。
      \begin{itemize}[<+->]
      \item 因為沒加上這行,程式就會直接執行結束。
      \item 加上這行,程式會在這裡「等你」。
      \end{itemize}
    \end{itemize}
  \end{exampleblock}
\end{frame}

\begin{frame}[fragile]
  \frametitle{注意事項}
  \begin{block}{輸出}
    \begin{enumerate}
    \item<1-> 如果改成「\lstinline{cout << 1}{}」(去掉分號) 會發生什麼結果?
    \item<3-> 試試看「\lstinline{cout << 1 << 2;}{}」,和你所想的有何不同?
    \item<5-> 那麼「\lstinline{cout << 1 << " " << 2;}{}」呢?
    \end{enumerate}
  \end{block}
  \begin{exampleblock}{註}<2->
    \begin{itemize}
    \item<2-> 「分號」對 C++ 而言代表「\alert{一個句子的結束}」,因此當一行\alert{指令結束就要加分號}。
    \item<4-> \lstinline{<<}{} 可以串很多東西一起輸出。
    \item<5-> \lstinline{" "}{} 是雙引號中間夾著一個「空白」,\alert{要注意}!
    \end{itemize}
  \end{exampleblock}
\end{frame}

\begin{frame}[fragile]
  \frametitle{換行「符號」}
  \begin{block}{換行}
    \begin{enumerate}[<+->]
    \item 試試看「\lstinline{cout << 1 << 2 << endl;}{}」,和「\lstinline{cout << 1 << 2;}{}」有什麼不同呢?
    \item 如果看不出來,試試看「\lstinline{cout << 1 << endl << 2;}{}」。
    \end{enumerate}
  \end{block}
  \begin{exampleblock}{註}<+->
    \begin{itemize}
    \item 「\lstinline{endl}」代表\alert{換行}符號,輸出中很好用。
    \end{itemize}
  \end{exampleblock}
\end{frame}

\subsection{變數}

\begin{frame}[fragile]
  \frametitle{變數}
  \begin{block}{變數}
    \begin{itemize}[<+->]
    \item 和數學「變數」的概念不太一樣
    \item 程式的變數像是「\alert{容器}」,可以裝資料。
    \item C++ 裡,每個容器都要先講好\alert{用途},這個步驟叫做「\alert{宣告}」。
    \end{itemize}
  \end{block}
  \begin{alertblock}{宣告變數}<+->
  \begin{lstlisting}
  int x;
  \end{lstlisting}
  \end{alertblock}
  \begin{exampleblock}{註}<+->
    \begin{itemize}
    \item 宣告就是幫變數取名字,此例將變數取名為「\lstinline{x}{}」。
    \item<+-> 「\lstinline{int}{}」代表的意義是「\alert{整數}」,規定變數 \lstinline{x}{} \alert{只能裝整數}。
    \end{itemize}
  \end{exampleblock}
\end{frame}

\begin{frame}[fragile]
  \frametitle{變數的功用}
  \begin{block}{把數字裝到變數}<+->
  \begin{lstlisting}
  #include <iostream>
  using namespace std;
  int main() {
    int x;             // 宣告變數 x
    x = 5;             // 把整數 5 裝進 x 裡面
    cout << x << endl; // 印出變數 x 存的值
  }
  \end{lstlisting}
  \end{block}
  \begin{exampleblock}{註}<+->
    \begin{itemize}
    \item 「\lstinline{int}{}」宣告變數可以裝整數之外,還有很多不同的種類,以後會慢慢介紹。
    \item<+-> 「\lstinline{x = 5;}{}」這行\alert{不要}和數學中的「等於」搞混。
    \end{itemize}
  \end{exampleblock}
\end{frame}

\begin{frame}
  \frametitle{練習看看}
  \begin{block}{練習}
    若把上個投影片「\lstinline{x = 5;}{}」改成
    \begin{enumerate}[<+->]
    \item 「\lstinline{x = 5.0;}{}」會發生什麼事?
    \item 「\lstinline{x = 0.5;}{}」呢?
    \item 那改成「\lstinline{5 = x;}{}」呢?
    \end{enumerate}
  \end{block}
  \begin{exampleblock}{註}<+->
    \begin{itemize}
      \item 這些練習目的是要讓你了解\alert{問題出現時}的現象,了解出問題的原因才有辦法 debug
      \item<+-> 為什麼會出現這些現象我們繼續下去就知道了
    \end{itemize}
  \end{exampleblock}
\end{frame}

\begin{frame}[fragile]
  \frametitle{宣告多個變數}
  \begin{exampleblock}{宣告兩個整數}
    \begin{itemize}[<+->]
    \item 可以寫成這樣:
      \begin{lstlisting}
  int a;
  int b;
      \end{lstlisting}
    \item 更可以簡化成這樣:
      \begin{lstlisting}
  int a, b;
      \end{lstlisting}
    \end{itemize}
  \end{exampleblock}
  \begin{exampleblock}{宣告三個整數}<+->
    \begin{lstlisting}
  int a, b, c;
    \end{lstlisting}
  \end{exampleblock}
\end{frame}

\begin{frame}[fragile]
  \frametitle{如果容器不塞東西呢 ...}
  \begin{block}{變數不塞整數進去}<+->
    \begin{lstlisting}
  #include <iostream>
  using namespace std;
  int main() {
    int x;             // 宣告變數 x
    cout << x << endl; // 印出變數 x 存的值
  }
    \end{lstlisting}
  \end{block}
  \begin{exampleblock}{練習}<+->
    \begin{enumerate}
    \item 執行看看,發生什麼事?
    \item<+-> 再執行幾次,又會發生什麼事呢?
    \end{enumerate}
  \end{exampleblock}
\end{frame}

\begin{frame}[fragile]
  \frametitle{初始化}
  \begin{alertblock}{初始化}
    \begin{itemize}[<+->]
    \item C++ 中,所有變數都要自己去\alert{初始化}。
      \begin{itemize}
      \item 例如:\lstinline{x = 5;}{},把整數 5 丟給 \lstinline{x}{} 等等。
      \end{itemize}
    \item 沒有初始化過的變數,裡面裝的資料是\alert{不確定}的。
      \begin{itemize}[<+->]
      \item 或許你很幸運看到 \lstinline{x}{} 都是 0
      \item 但那只是\alert{恰巧}而已。
      \end{itemize}
    \end{itemize}
  \end{alertblock}
\end{frame}

\subsection{輸入}

\begin{frame}[fragile]
  \frametitle{程式的輸入}
  \begin{block}{輸入}
    執行以下程式
    \begin{lstlisting}
  #include <iostream>
  using namespace std;
  int main() {
    int x;
    cin >> x;
    cout << x << endl;
  }
    \end{lstlisting}
    會發生什麼事呢?
  \end{block}
  \begin{block}{練習}<2->
  如果沒發生什麼事,試著輸入「1」再按 enter 鍵,會發生什麼事呢?
  \end{block}
\end{frame}

\begin{frame}[fragile]
  \frametitle{程式的輸入}
  \begin{exampleblock}{輸入「符號」}<1->
    \begin{itemize}
    \item 「\lstinline{cin}{}」代表輸入符號,可以輸入後面變數的資料。
      \begin{itemize}
      \item<2-> 此例中,\lstinline{x}{} 是整數,因此可以\alert{輸入一個整數}。
      \item<3-> \lstinline{cin}{} 的 \alert{\lstinline{>>}{}} 不要和 \lstinline{cout}{} 的 \alert{\lstinline{<<}{}} 搞混。
      \end{itemize}
    \end{itemize}
  \end{exampleblock}
  \begin{block}{練習 (續)}<4->
    \begin{itemize}
    \item 如果輸入「5.0」再按 enter 鍵呢?
    \item<5-> 如果輸入「0.5」再按 enter 鍵呢?
    \item<6-> 如果輸入「XD」再按 enter 鍵呢?
    \end{itemize}
  \end{block}
  \begin{alertblock}{多變數輸入}<7->
    \begin{lstlisting}
  int x, y;
  cin >> x >> y;
    \end{lstlisting}
    \begin{itemize}[<8->]
    \item 不要在輸入中加入「\lstinline{endl}{}」。
    \end{itemize}
  \end{alertblock}
\end{frame}

\subsection{資料型態}

\begin{frame}[fragile]
  \frametitle{資料型態}
  \begin{block}{資料型態}
    \begin{itemize}
    \item<1-> 有裝整數的容器,那麼當然也可以宣告裝「小數點」的容器啦!
    \item<2-> 這些不同用途的容器我們稱為「\alert{資料型態}」。
    \end{itemize}
  \end{block}
  \pause\pause
  \begin{table}[h]
    \begin{tabular}{|c|c|c|}
    \hline
    關鍵字                   & 意義 & 備註\\
    \hline
    \lstinline{bool}{}      & 布林值 & 只有 \lstinline{true}{} 和 \lstinline{false}{}\\
    \hline
    \lstinline{int}{}       & 整數 &\\
    \hline
    \lstinline{long long}{} & 長整數 & 存比較大的整數,以後會介紹\\
    \hline
    \lstinline{double}{}    & \alert{浮點數} & 也就是小數點\\
    \hline
    \end{tabular}
    \caption{資料型態}
  \end{table}
  \begin{exampleblock}{註}<4->
  詳細內容之後再介紹,先來用看看這些東西。
  \end{exampleblock}
\end{frame}

\begin{frame}[fragile]
  \frametitle{布林值}
  \begin{exampleblock}{布林值}
    \begin{itemize}[<+->]
    \item 一種資料型態,只拿來裝兩種數值:「\lstinline{true}{}」和「\lstinline{false}{}」。
    \end{itemize}
  \end{exampleblock}
  \begin{alertblock}{宣告}<+->
    \begin{lstlisting}
  bool b;
    \end{lstlisting}
  \end{alertblock}
  \begin{alertblock}{注意}<+->
    \begin{itemize}
    \item 兩種不同的宣告不能用「逗號」隔開:
      \begin{lstlisting}
  int a, bool b;
      \end{lstlisting}
    \item<+-> 逗號有\alert{特殊意義},不要想成一般的「逗號」。
    \end{itemize}
  \end{alertblock}
\end{frame}

\begin{frame}[fragile]
  \frametitle{賦值}
  \begin{block}{定義}
    \begin{itemize}[<+->]
    \item 將一個「數值」裝進一個變數中,稱為\alert{賦值}。
    \item 例如,之前把整數 5 裝進整數變數 \lstinline{x}{} 中:
      \begin{lstlisting}
  int x;
  x = 5;
      \end{lstlisting}
    \end{itemize}
  \end{block}
  \begin{exampleblock}{賦值簡化}<+->
    \begin{itemize}
    \item 變數宣告和賦值可以寫在一起:
      \begin{lstlisting}
  int x = 5; // 宣告一個整數變數 x 並且把 5 裝進去
      \end{lstlisting}
    \end{itemize}
  \end{exampleblock}
\end{frame}

\begin{frame}[fragile]
  \frametitle{布林值}
  \begin{block}{練習}
    \begin{lstlisting}
 bool b;
 cout << b << endl;
    \end{lstlisting}
    對程式碼的 \lstinline{b}{} 做以下賦值,會發生什麼事?
    \begin{enumerate}
    \item<2-> \lstinline{b = true;}{}
    \item<3-> \lstinline{b = false;}{}
    \item<4-> \lstinline{b = 2;}{}
    \item<5-> \lstinline{b = 0;}{}
    \item<6-> \lstinline{b = -1;}{}
    \end{enumerate}
  \end{block}
\end{frame}

\begin{frame}[fragile]
  \frametitle{布林值的重要觀念}
  \begin{alertblock}{觀念}
    \begin{itemize}
    \item C++ 中,「\alert{非零整數}」會被當做「\lstinline{true}{}」,印出時也會印出一個非零整數 (\alert{通常是 1})。
    \item<2-> 「0」會被當做「\lstinline{false}{}」,印出時會印出「\alert{0}」。
    \end{itemize}
  \end{alertblock}
\end{frame}

\begin{frame}[fragile]
  \frametitle{浮點數}
  \begin{itemize}[<+->]
    \item 先跳過 \lstinline{long long}{},先知道 \lstinline{long long}{} 也是存整數就好。
    \item 謎之音:「那幹嘛現在說= =」
  \end{itemize}
  \begin{alertblock}{浮點數宣告}<+->
    \begin{lstlisting}
  double d;
    \end{lstlisting}
  \end{alertblock}
  \begin{exampleblock}{賦值}<+->
    \begin{itemize}
    \item 把 1.0 丟給 \lstinline{d}{} $\Rightarrow$ \lstinline{d = 1.0;}{}
    \item<+-> 把 0.5 丟給 \lstinline{d}{} $\Rightarrow$ \lstinline{d = 0.5;}{}
      \begin{itemize}[<+->]
      \item 0.5 也可寫為 \lstinline{d = .5;}{}
      \end{itemize}
    \item<+-> \lstinline{18.23e5}{} $\Rightarrow$ 代表 $18.23\times{10^5}$ (\alert{科學記號})
    \end{itemize}
  \end{exampleblock}
\end{frame}

\section{算術運算子}
\begin{frame}
  \frametitle{大綱}
  \begin{multicols}{2}
    \tableofcontents[currentsection]
  \end{multicols}
\end{frame}

\subsection{運算性質}

\begin{frame}[fragile]
  \frametitle{算術運算子}
  \begin{table}[h]
    \begin{tabular}{|c|c|c|c|}
    \hline
    算術運算子      & 意義 & 運算順序 & 結合性\\
    \hline
    \lstinline{+}{} & 加法 & 6       & 左$\rightarrow$右\\
    \hline
    \lstinline{-}{} & 減法 & 6       & 左$\rightarrow$右\\
    \hline
    \lstinline{*}{} & 乘法 & 5       & 左$\rightarrow$右\\
    \hline
    \lstinline{/}{} & 除法 & 5       & 左$\rightarrow$右\\
    \hline
    \lstinline{%}{} & 取餘數 & 5       & 左$\rightarrow$右\\
    \hline
    \end{tabular}
  \caption{算術運算子}
  \end{table}
  \begin{exampleblock}{註}<2->
    \begin{itemize}
    \item<2-> 不管\alert{運算順序}和\alert{結合性},一般來說可以用五則運算來理解
    \item<3-> 只不過程式跟數學還是有差距 ...
      \begin{itemize}[<4->]
      \item 這個故事說來話長,我們先舉個簡單的例子吧!
      \end{itemize}
    \end{itemize}
  \end{exampleblock}
\end{frame}

\begin{frame}
  \frametitle{舉個例子}
  \begin{exampleblock}{$1+2+3=?$}
    \begin{itemize}
    \item<2-> 答案:6。
    \item<3-> 為什麼? (謎之音:「什麼為什麼?」)
    \end{itemize}
  \end{exampleblock}
  \begin{block}{定義}<4->
  \alert{二元運算}有一個\alert{運算子}和兩個\alert{運算元},例如:
    \begin{enumerate}
    \item<5-> $1+2$:「$+$」稱為「運算子」,「$1$」和「$2$」稱為運算元 (我們常稱為「被加數」和「加數」)。
    \item<6-> 我們可以知道「加減乘除餘」都是二元運算。
    \end{enumerate}
  \end{block}
  \begin{itemize}[<7->]
  \item Well, 我們回到原來的問題 ...
  \end{itemize}
\end{frame}

\subsection{結合性與運算順序}

\begin{frame}[fragile]
  \frametitle{回到原來的問題 ...}
  \begin{exampleblock}{$1+2+3=?$}
    \begin{itemize}
    \item \alert{出現大麻煩啦!}
      \begin{itemize}[<2->]
        \item 根據剛剛說的,加法只有兩個運算元,那麼「$1+2+3$」該怎麼辦呢?
      \end{itemize}
    \item<3-> 解法:決定運算的\alert{方向}。例如:
      \begin{enumerate}
        \item<4-> 先算 $1+2=3$,再算 $\alert{3}+3=6$
        \item<5-> 先算 $2+3=5$,再算 $1+\alert{5}=6$
      \end{enumerate}
    \item<6-> 謎之音:「那還不是一樣嘛?廢話= =」
    \end{itemize}
  \end{exampleblock}
  \begin{alertblock}{注意}<7->
  決定運算方向對「電腦」而言\alert{意義重大}!
  \end{alertblock}
\end{frame}

\begin{frame}[fragile]
  \frametitle{再舉個例子 ...}
  \begin{exampleblock}{$1-2-3=?$}
    \begin{itemize}
    \item<2-> 我們直觀上會先算 $1-2=-1$,再算 $\alert{-1}-3=-4$。
    \item<3-> 因此 C++ 在設計上也會把加減乘除餘的\alert{結合性}「設定」成\alert{從左到右算}。
    \end{itemize}
  \end{exampleblock}
  \pause \pause \pause
  \begin{table}[h]
    \begin{tabular}{|c|c|c|c|}
    \hline
    算術運算子      & 意義 & 運算順序 & \alert{結合性}\\
    \hline
    \lstinline{+}{} & 加法 & 6       & \alert{左$\rightarrow$右}\\
    \hline
    \lstinline{-}{} & 減法 & 6       & \alert{左$\rightarrow$右}\\
    \hline
    \lstinline{*}{} & 乘法 & 5       & \alert{左$\rightarrow$右}\\
    \hline
    \lstinline{/}{} & 除法 & 5       & \alert{左$\rightarrow$右}\\
    \hline
    \lstinline{%}{} & 取餘數 & 5       & \alert{左$\rightarrow$右}\\
    \hline
    \end{tabular}
    \caption{算術運算子}
  \end{table}
\end{frame}

\begin{frame}[fragile]
  \frametitle{萬一是四則運算呢?}
  \begin{exampleblock}{$1+2*3-4=?$}
    \begin{itemize}
    \item<2-> 我們的運算規則:「\alert{先乘除餘,後加減}」。
    \item<3-> 因此 C++ 發展出一套規則:\alert{運算順序}
      \begin{itemize}
      \item<4-> 運算順序小的優先運算
      \item<5-> 若運算順序相同,則依照運算方向做計算。
      \end{itemize}
    \end{itemize}
  \end{exampleblock}
  \pause \pause \pause \pause \pause
  \begin{table}[h]
    \begin{tabular}{|c|c|c|c|}
    \hline
    算術運算子      & 意義 & \alert{運算順序} & 結合性\\
    \hline
    \lstinline{+}{} & 加法 & \alert{6}       & 左$\rightarrow$右\\
    \hline
    \lstinline{-}{} & 減法 & \alert{6}       & 左$\rightarrow$右\\
    \hline
    \lstinline{*}{} & 乘法 & \alert{5}       & 左$\rightarrow$右\\
    \hline
    \lstinline{/}{} & 除法 & \alert{5}       & 左$\rightarrow$右\\
    \hline
    \lstinline{%}{} & 取餘數 & \alert{5}     & 左$\rightarrow$右\\
    \hline
    \end{tabular}
    \caption{算術運算子}
  \end{table}
\end{frame}

\begin{frame}
  \frametitle{回到原來例子}
  \begin{exampleblock}{$1+2*3-4=?$}
    \begin{align*}
    \onslide<1->{  & 1+\alert{2*3}-4 &\text{我們可以看到 }*\text{ 的運算順序最高}\\}
    \onslide<2->{= & \alert{1+6}-4   &\text{加法和減法運算順序相同,依照結合性從左到右算}\\}
    \onslide<3->{= & \alert{7-4}     &\text{依照結合性從左到右算}\\
                 = & 3}
    \end{align*}
  \end{exampleblock}
  \begin{alertblock}{觀念}<4->
    \begin{itemize}
    \item C++ 的四則運算用\alert{優先順序}和\alert{結合性}來處理。
    \item<5-> 這件事情非常重要,稍後就會知道為什麼。
    \end{itemize}
  \end{alertblock}
\end{frame}

\subsection{整數除法與除零問題}

\begin{frame}[fragile]
  \frametitle{整數除法}
  \begin{block}{整數除法}
    \begin{enumerate}[<+->]
    \item \lstinline{cout << 8 / 5 << endl;}{} 的結果?\onslide<+->{\alert{Ans: 1}}
    \item \lstinline{cout << 8.0 / 5.0 << endl;}{} 的結果?\onslide<+->{\alert{Ans: 1.6}}
    \end{enumerate}
  \end{block}
  \begin{exampleblock}{註}<+->
    \begin{itemize}
    \item 在 \lstinline{8 / 5}{} 中,8 和 5 被視為 \lstinline{int}{},因此 C++ 會做「\alert{整數除法}」。
    \item<+-> 而在 \lstinline{8.0 / 5.0}{} 中,8.0 和 5.0 被視為浮點數 \lstinline{double}{},因此會做「\alert{浮點數除法}」。
    \end{itemize}
  \end{exampleblock}
  \begin{itemize}[<+->]
  \item 除法還有另外一個問題點 ...
  \end{itemize}
\end{frame}

\begin{frame}[fragile]
  \frametitle{除以零}
  \begin{block}{試試看}
    我們知道數學上是不能除以零的,那程式呢?
    \begin{enumerate}
    \item \lstinline{cout << 1 / 0 << endl;}{}
    \item<3-> \lstinline{cout << 0 / 0 << endl;}{}
    \item<4-> \lstinline{cout << 1.0 / 0.0 << endl;}{}
    \item<5-> \lstinline{cout << 0.0 / 0.0 << endl;}{}
    \end{enumerate}
  \end{block}
  \begin{exampleblock}{註}<2->
  如果無法編譯成功,那麼就宣告一個變數,把分母裝進去再試試看。
  \end{exampleblock}
  \begin{alertblock}{注意}<6->
  通常編譯可以過,但是在執行時會出些狀況,各位知道出了哪些狀況就好,不用了解太詳細。
  \end{alertblock}
\end{frame}

\subsection{應用:取餘數}

\begin{frame}[fragile]
  \frametitle{取餘數}
  \begin{block}{觀察現象}
    \begin{enumerate}[<+->]
    \item \lstinline{cout << 5 % 3 << endl;}{} 會輸出什麼?\onslide<+->{\alert{Ans:2}}
    \item \lstinline{cout << (-5) % 3 << endl;}{} 呢?\onslide<+->{\alert{Ans:-2}}
    \end{enumerate}
  \end{block}
  \begin{exampleblock}{註}<+->
    \begin{itemize}
    \item 事情不該是這樣發展的啊!!!
    \item<+-> 謎之音:「應該結果是要 1 才對。」
      \begin{itemize}[<+->]
      \item C++ 一個\alert{奇怪的特性} ...
      \end{itemize}
    \end{itemize}
  \end{exampleblock}
\end{frame}

\begin{frame}[fragile]
  \frametitle{解決辦法?}
  \begin{block}{問題}<+->
  要怎麼做出取餘數的效果呢?
  \end{block}
  \begin{enumerate}[<+->]
  \item 假設 n 要 mod m ...
  \item 首先,我們取 \lstinline{n % m}{}
    \begin{itemize}[<+->]
    \item 如果 $n\geq{0}$\onslide<+->{,會得到介於 $0$ 到 $m-1$ 的數字}
    \item 如果 $n<0$\onslide<+->{,會得到介於 $-(m-1)$ 到 $0$ 的數字}
    \end{itemize}
  \item 接著加上 m
    \begin{itemize}[<+->]
    \item 如果 $n\geq{0}$\onslide<+->{,會得到介於 $m$ 到 $2m-1$ 的數字}
    \item 如果 $n<0$\onslide<+->{,會得到介於 $-(m-1)+m=1$ 到 $m$ 的數字}
    \item 全都修成正值了!\onslide<+->{\alert{但還差最後一步 ...}}
    \end{itemize}
  \item 最後,再 mod m 一次,把所有數字修正回 $0$ 到 $m-1$ 之間。
    \begin{itemize}[<+->]
    \item 大功告成啦~ \lstinline{(n % m + m) % m}{}
    \end{itemize}
  \end{enumerate}
\end{frame}

\begin{frame}
  \frametitle{練習題}
  \begin{exampleblock}{\href{http://unfortunate-dog.github.io/articles/100/p10071/}{UVa 10071: Back to High School Physics}}
  \label{uva:10071}
  這題只要能夠讀懂題意都不難寫。如果不知道怎樣讀取多筆測資請先參考迴圈部分 (EOF 版)。
  \end{exampleblock}
\end{frame}

\section{比較和邏輯運算子}
\begin{frame}
  \frametitle{大綱}
  \begin{multicols}{2}
    \tableofcontents[currentsection]
  \end{multicols}
\end{frame}

\subsection{簡化規則}

\begin{frame}[fragile]
  \frametitle{比較運算子}
  \begin{table}[h]
    \begin{tabular}{|c|c|c|c|}
    \hline
    比較運算子         & 意義   & 運算順序 & 結合性\\
    \hline
    \lstinline{==}{} & 等於   & 9       & 左$\rightarrow$右\\
    \hline
    \lstinline{!=}{} & 不等於 & 9       & 左$\rightarrow$右\\
    \hline
    \lstinline{>}{}  & 大於   & 8       & 左$\rightarrow$右\\
    \hline
    \lstinline{<}{}  & 小於   & 8       & 左$\rightarrow$右\\
    \hline
    \lstinline{>=}{} & 不小於 & 8       & 左$\rightarrow$右\\
    \hline
    \lstinline{<=}{} & 不大於 & 8       & 左$\rightarrow$右\\
    \hline
    \end{tabular}
    \caption{比較運算子}
  \end{table}
  \begin{alertblock}{注意}<2->
    \begin{itemize}
    \item C++ 的等於寫作「\lstinline{==}{}」,不要和賦值的「\lstinline{=}{}」搞混。
    \end{itemize}
  \end{alertblock}
\end{frame}

\begin{frame}[fragile]
  \frametitle{回傳值}
  \begin{block}{例子}<+->
    \begin{enumerate}
    \item \lstinline{cout << (3 < 5) << endl;}{},會發生什麼事?
    \end{enumerate}
  \end{block}
  \begin{exampleblock}{註}<+->
    \begin{itemize}
    \item 比較運算子也是\alert{二元運算},他會比較兩邊數字大小:
      \begin{itemize}[<+->]
      \item 如果正確,則為 \lstinline{true}{}
      \item 否則就是 \lstinline{false}{}
      \end{itemize}
    \item<+-> 這種概念我們稱為「\alert{回傳值}」
      \begin{itemize}[<+->]
      \item 比較運算子的回傳值是布林值 \lstinline{bool}{}
      \item \alert{\lstinline{3 < 5}{}} $\Rightarrow$ \lstinline{true}{}
      \item 因為我們要輸出 \lstinline{true}{},根據 C++ 的規則,我們知道 \lstinline{true}{} 代表\alert{非零},因此會印出一個非零的數字 (通常是 1)
      \end{itemize}
    \end{itemize}
  \end{exampleblock}
\end{frame}

\begin{frame}[fragile]
  \frametitle{運算簡化}
  \begin{block}{例子}<+->
  判斷整數 \lstinline{n % m}{} 是否「不是 0」。
  \end{block}
  \begin{exampleblock}{判斷整除}<+->
    \lstinline{n % m != 0}{}
    \begin{itemize}[<+->]
    \item 如果 \lstinline{n % m}{} 的回傳值 $\neq{0}$ $\Rightarrow$ \lstinline{true}{}
    \item 如果是 0,則為 \lstinline{false}{}
    \end{itemize}
  \end{exampleblock}
  \begin{alertblock}{簡化寫法}<+->
    \lstinline{n % m}{}
    \begin{itemize}[<+->]
    \item 如果 \lstinline{n % m}{} 的回傳值 $\neq{0}$ ,可以被當做「\lstinline{true}{}」
    \item 如果是 0,那麼就可以當做「\lstinline{false}{}」
    \end{itemize}
  \end{alertblock}
\end{frame}

\begin{frame}
  \frametitle{遙遠的記憶}
  \begin{alertblock}{布林值的重要觀念}<+->
    \begin{itemize}
    \item C++ 中,「\alert{非零整數}」會被當做「\lstinline{true}{}」,印出時也會印出一個非零整數 (\alert{通常是 1})。
    \item 「0」會被當做「\lstinline{false}{}」,印出時會印出「\alert{0}」。
    \end{itemize}
  \end{alertblock}
  \begin{exampleblock}{註}<+->
    \begin{itemize}
    \item 簡化的寫法大多時候可以取代原來一般寫法。
    \item<+-> 通常比較運算子要和 \lstinline{if}{}、\lstinline{else}{} 配合。
    \end{itemize}
  \end{exampleblock}
\end{frame}

\begin{frame}[fragile]
  \frametitle{邏輯運算子}
  \begin{table}[h]
    \begin{tabular}{|c|c|c|c|}
    \hline
    邏輯運算子         & 意義 & 運算順序 & 結合性\\
    \hline
    \lstinline{&&}{} & 且   & 13      & 左$\rightarrow$右\\
    \hline
    \lstinline{||}{} & 或   & 14      & 左$\rightarrow$右\\
    \hline
    \lstinline{!}{}  & 非   & 3       & \alert{右$\rightarrow$左}\\
    \hline
    \end{tabular}
    \caption{邏輯運算子}
  \end{table}
  \begin{exampleblock}{作用}<2->
    \begin{itemize}
    \item 一般來說是連接比較運算子
    \item<3-> 例如:\lstinline{1 < x && x < 5}{}
    \end{itemize}
  \end{exampleblock}
\end{frame}

\begin{frame}[fragile]
  \frametitle{舉個例子}
  \begin{block}{例子}<+->
  判斷 x 是否介於 a 和 b 之間能不能寫成 \lstinline{a <= x <= b;}{} 呢?\onslide<+->{\alert{Ans:不行}。}
  \end{block}
  \begin{exampleblock}{用回傳值的觀點}<+->
    \begin{itemize}
    \item 我們知道 \lstinline{<=}{} 運算子在列出很多個時,會\alert{由左到右算}
    \item<+-> \alert{\lstinline{a <= x}{}} 先算出 \lstinline{true}{} 或者是 \lstinline{false}{}
    \item<+-> 如果是 \lstinline{true}{},假設 \lstinline{a=-4}{}、\lstinline{b=-1}{}、\lstinline{x=-2}{}
      \begin{itemize}[<+->]
      \item \lstinline{a <= x <= b}{} 先算 \lstinline{a <= x}{} 得到 \lstinline{true}{}
      \item \lstinline{true <= b}{},因為 \lstinline{true}{} 通常是 1,但此時我們假設 \lstinline{b=-1}{},整句就會回傳 \lstinline{false}{}
      \item 但事實上 \lstinline{x}{} 是在 \lstinline{a}{} 和 \lstinline{b}{} 裡面。
      \end{itemize}
    \item<+-> \lstinline{a <= x}{} 是 \lstinline{false}{} 也會有同樣的問題。
    \end{itemize}
  \end{exampleblock}
\end{frame}

\begin{frame}[fragile]
  \frametitle{練習題 (1)}
  \begin{exampleblock}{\href{http://unfortunate-dog.github.io/articles/100/p10055/}{UVa 10055: Hashmat the brave warrior}}<+->
  \label{uva:10055}
  取絕對值有兩種做法,一種是用 \lstinline{if}{} 判斷;另一種是呼叫函數 \lstinline{abs()}{} 就好了。\lstinline{abs()}{} 函數被定義在 \lstinline{<cstdlib>}{} 中,雖然沒有 \lstinline{include}{} 在 Visual C++ 依然能編譯過,但是上傳時因為編譯器的原因會導致\alert{編譯錯誤} (Compilation Error, CE)。
  \end{exampleblock}
  \begin{alertblock}{注意}
  另外要注意這一題的整數型態需用 \lstinline{long long}{},用 \lstinline{int}{} 會造成溢位現象,這個原因會在後面說明。
  \end{alertblock}
\end{frame}

\begin{frame}
  \frametitle{練習題 (2)}
  \begin{exampleblock}{\href{http://unfortunate-dog.github.io/articles/111/p11172/}{UVa 11172: Relational Operators}}<+->
  \label{uva:11172}
  能夠理解題意就不難解決此道問題。
  \end{exampleblock}
  \begin{exampleblock}{\href{http://unfortunate-dog.github.io/articles/119/p11942/}{UVa 11942: Lumberjack Sequencing}}<+->
  \label{uva:11942}
  依序給你一些木頭的長度,問你這些木頭是不是由長到短,或是由短到長排列。
  \end{exampleblock}
\end{frame}

\subsection{短路運算}

\section{位元運算子}
\begin{frame}
  \frametitle{大綱}
  \begin{multicols}{2}
    \tableofcontents[currentsection]
  \end{multicols}
\end{frame}

\subsection{int 和 long long 的儲存形式}

\begin{frame}[fragile]
  \frametitle{基本觀念}
  \begin{exampleblock}{觀念}
    \begin{itemize}[<+->]
    \item \alert{位元} (bit, b):計算機儲存資料的基本單位,只儲存 \alert{0 和 1}
    \item \alert{位元組} (byte, B):因為位元很多,所以我們把 8 個位元「打包起來」,變成一個位元組
      \begin{table}[h]
        \begin{tabular}{|c|}
        \hline
        01001010\\
        \hline
        \end{tabular}
        \caption{位元組}
      \end{table}
    \item 常見應用
      \begin{itemize}[<+->]
      \item KB、MB、GB、TB、PB:資料大小
      \item Kbps、Mbps、Gbps:資料傳輸速度
      \end{itemize}
    \end{itemize}
  \end{exampleblock}
\end{frame}

\begin{frame}[fragile]
  \frametitle{int 表示法}
  \begin{block}{int ...}
    \begin{itemize}
    \item 有至少 2 個位元組
    \item<2-> 謎之音:「蝦米?」\onslide<3->{\alert{不是 4 個位元組嘛!!!}}
    \item<4-> 事實上當初定義時,\lstinline{int}{} 只有「至少」2 位元組。
    \item<5-> 現今大多是 4 位元組。
    \end{itemize}
  \end{block}
  \pause \pause \pause \pause \pause
  \begin{table}[h]
    \begin{tabular}{|c|c|}
      \hline
      型態                     & 長度\\
      \hline
      \lstinline{bool}{}      & 1 位元組\\
      \hline
      \lstinline{int}{}       & 2 或 \alert{4} 位元組\\
      \hline
      \lstinline{long long}{} & 4 或 \alert{8} 位元組\\
      \hline
      \lstinline{double}{}    & 8 位元組\\
      \hline
    \end{tabular}
    \caption{位元組長度}
  \end{table}
\end{frame}

\begin{frame}[fragile]
  \frametitle{int 表示法}
  \begin{alertblock}{int 表示法}
    \begin{itemize}[<+->]
    \item 一般來說,\lstinline{int}{} 由 4 個位元組組成
      \begin{table}[h]
      \begin{tabular}{|c|c|c|c|}
      \hline
      10100010 & 00110011 & 00100111 & 10101101\\
      \hline
      \end{tabular}
      \end{table}
    \item 可以視為一個長度是 32 的二進位數字,我們將位數依照高低編號
      \begin{table}[h]
      \begin{tabular}{|c|c|c|c|}
      \hline
      $x_{31}x_{30}\cdots{x_{24}}$ & $x_{23}x_{22}\cdots{x_{16}}$ & $x_{15}x_{14}\cdots{x_{8}}$ & $x_{7}x_{6}\cdots{x_{0}}$\\
      \hline
      \end{tabular}
      \end{table}
    \item $x_{31}$ 可表示正負號
      \begin{itemize}[<+->]
      \item 0 代表 \lstinline{int}{} 是正數
      \item 1 代表 \lstinline{int}{} 是負數
      \end{itemize}
    \end{itemize}
  \end{alertblock}
  \begin{exampleblock}{註}<+->
  \lstinline{int}{} 的儲存方式很特別,要多花一些力氣說明。
  \end{exampleblock}
\end{frame}

\begin{frame}[fragile]
  \frametitle{int 存正數的情況}
  \begin{alertblock}{規則}<+->
  依照一般的二進位方式儲存。
  \end{alertblock}
  \begin{exampleblock}{例如}<+->
    \begin{itemize}
    \item \lstinline{int x = 1;}{}
      \begin{table}[h]
      \begin{tabular}{|c|c|c|c|}
        \hline
        \alert{0}0000000 & 00000000 & 00000000 & 0000000\alert{1}\\
        \hline
      \end{tabular}
      \end{table}
    \item<+-> \lstinline{int x = 255;}{}
      \begin{table}[h]
      \begin{tabular}{|c|c|c|c|}
        \hline
        \alert{0}0000000 & 00000000 & 00000000 & \alert{11111111}\\
        \hline
      \end{tabular}
      \end{table}
    \end{itemize}
  \end{exampleblock}
\end{frame}

\begin{frame}[fragile]
  \frametitle{int 存負數的情況}
  \begin{exampleblock}{舉例}
    \begin{itemize}[<+->]
    \item \lstinline{int x = -1;}{}
      \begin{table}[h]
        \begin{tabular}{|c|c|c|c|}
        \hline
        \alert{1}1111111 & 11111111 & 11111111 & 11111111\\
        \hline
        \end{tabular}
      \end{table}
    \item 謎之音:「根本黑魔法!」
    \end{itemize}
  \end{exampleblock}
  \begin{block}{想法}<+->
    \begin{itemize}
    \item 我們知道 $(-1)+1=0$,那麼拿這種表示法加加看
      \begin{table}[h]
        \begin{tabular}{|c|r|c|c|c|}
        \hline
            &          11111111 & 11111111 & 11111111 & 11111111\\
        \hline
        $+$ &          00000000 & 00000000 & 00000000 & 00000001\\
        \hline
        \hline
            & \alert{1}00000000 & 00000000 & 00000000 & 00000000\\
        \hline
        \end{tabular}
      \end{table}
    \item<+-> 紅色的 \alert{1} 因為超過 32 位元,因此被捨棄,稱為\alert{溢位}
    \end{itemize}
  \end{block}
\end{frame}

\begin{frame}[fragile]
  \frametitle{int 負數規則}
  \begin{exampleblock}{練習}
    \begin{itemize}[<+->]
    \item \lstinline{int x = -2;}{}
    \onslide<+->{
      \begin{table}[h]
        \begin{tabular}{|c|c|c|c|}
        \hline
        11111111 & 11111111 & 11111111 & 11111110\\
        \hline
        \end{tabular}
      \end{table}
    }
    \item \lstinline{int x = -256;}{}
    \onslide<+->{
      \begin{table}[h]
        \begin{tabular}{|c|c|c|c|c|}
        \hline
        11111111 & 11111111 & 11111111 & 00000000\\
        \hline
        \end{tabular}
      \end{table}
    }
    \end{itemize}
  \end{exampleblock}
  \begin{alertblock}{重點}<+->
    \begin{itemize}
    \item 這種表示法稱為\alert{二補數 (2's complement)}
    \item<+-> 要想像負數 $-x$ 的表示法,訣竅是 $(-x)+x$ 會因為溢位等於 0
    \item<+-> 記得 \lstinline{0}{} 是\alert{全 0},\lstinline{-1}{} 是\alert{全 1}
    \end{itemize}
  \end{alertblock}
\end{frame}

\begin{frame}[fragile]
  \frametitle{另外一個訣竅}
  \begin{alertblock}{問題}<1->
  給你一個正數 \lstinline{x}{},問如何不用負號的情況下求出 \lstinline{-x}{} 呢?
  \end{alertblock}
  \begin{block}{一元運算子}<2->
    \alert{一元運算子}就是只有一個運算元的運算子。
    \pause \pause
    \begin{table}[h]
      \begin{tabular}{|c|c|c|c|}
      \hline
      運算子           & 意義 & 運算順序 & 結合性\\
      \hline
      \lstinline{+}{} & 正號 & 3      & \alert{右$\rightarrow$左}\\
      \hline
      \lstinline{-}{} & 負號 & 3      & \alert{右$\rightarrow$左}\\
      \hline
      \end{tabular}
    \end{table}
  \end{block}
  \begin{exampleblock}{舉例}<4->
  \lstinline{--3}{} 會先算右邊的 \lstinline{-3}{},接著 \lstinline{-3}{} 再和左邊的負號運算子「運算」,回傳結果 \lstinline{3}{}。
  \end{exampleblock}
\end{frame}

\subsection{取特定餘數}

\subsection{mask}

\subsection{set bit}

\subsection{消失的數}

\subsection{parity}

\section{指定運算子}

\subsection{未定義行為}

\section{其他運算子}

\begin{frame}[fragile]
  \frametitle{其他運算子}
  \begin{table}[h]
    \begin{tabular}{|c|c|c|c|}
    \hline
    運算子                & 意義       & 運算順序 & 結合性\\
    \hline
    \lstinline{sizeof}{} & 求記憶體大小 & 3      & \alert{右$\rightarrow$左}\\
    \hline
    \lstinline{(type)}{} & 強制轉型    & 3      & \alert{右$\rightarrow$左}\\
    \hline
    \lstinline{,}{}      & 逗號       & 18      & 左$\rightarrow$右\\
    \hline
    \end{tabular}
  \end{table}
  \begin{alertblock}{觀念}<2->
    \begin{itemize}
    \item 萬物對計算機而言皆是「\alert{運算}」。
    \item<3-> 既然是運算,就有「結合性」和「運算順序」。
    \end{itemize}
  \end{alertblock}
\end{frame}

\begin{frame}
  \
\end{frame}

\section{結論}

\begin{frame}[fragile]
  \frametitle{小結}
  \begin{alertblock}{重點整理}
    \begin{enumerate}[<+->]
    \item 句子結尾是分號「\lstinline{;}{}」。
    \item 初始化的重要性。
    \item C++ 依照運算順序和結合性做運算。
    \item 除以零會遇到的現象。
    \item 「零」代表 \lstinline{false}{},「非零」代表 \lstinline{true}{}。
    \item 邏輯運算子是短路運算。
    \item \lstinline{int}{} 和 \lstinline{long long}{} 如何儲存,以及位元運算技巧。
    \item 注意未定義行為。
    \end{enumerate}
  \end{alertblock}
\end{frame}

\clearpage
\end{CJK}
\end{document}
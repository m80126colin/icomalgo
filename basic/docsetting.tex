% ======================================================================
%
%  替換中文
%
% ======================================================================
%% === 常用的指令,替換成中文 ===
\renewcommand{\figurename}{圖}
\renewcommand{\tablename}{表}
\renewcommand{\contentsname}{目~錄~}
\renewcommand{\listfigurename}{插~圖~目~錄}
\renewcommand{\listtablename}{表~格~目~錄}
\renewcommand{\appendixname}{附~錄}
%\renewcommand{\refname}{參~考~資~料}    % article
%\renewcommand{\bibname}{參~考~文~獻}     % book
\renewcommand{\indexname}{索~引}
\renewcommand{\today}{\number\year~年~\number\month~月~\number\day~日}
%\newcommand{\zhtoday}{\CJKdigits{\the\year}年\CJKnumber{\the\month}月\CJKnumber{\the\day}日}

%% === 
\floatname{algorithm}{演算法}

%% ===
\setcounter{secnumdepth}{3}                                 % 設定計數到 subsubsection
\renewcommand{\thepart}{第\CJKnumber{\arabic{part}}部分}
%\renewcommand{\thechapter}{\arabic{chapter}}
\renewcommand{\thesection}{\arabic{section}}                % 改 section 為 1, 2, 3 非 1.1, 1.2, 1.3
\renewcommand{\thesubsection}{\arabic{subsection}}          % subsection 也改一改
\renewcommand{\thesubsubsection}{\arabic{subsubsection}}    % subsubsection 也改一改

%% === 設定章節標題格式 (配合 titlesec) ===
% part
%\titleformat{\part}[display]
%  {\center\huge\bfseries}
%  {\thepart}
%  {1em}{\Huge}
% chapter
\titleformat{\chapter}[display]
  {\bf\Large}
  {第~\CJKnumber{\thechapter}~章}
  {1ex}{\Huge}[\vspace{2ex}]
% section
\titleformat{\section}
  {\Large\bfseries}
  {第\CJKnumber{\thesection}節}
  {1em}{}
% subsection
\titleformat{\subsection}
  {\large\bfseries}
  {\CJKnumber{\thesubsection}、}
  {0.5em}{}
% subsubsection
\titleformat{\subsubsection}
  {\bfseries}
  {(\CJKnumber{\thesubsubsection})}
  {0.5em}{}
% \titlespacing{\subsubsection}{0pt}{5pt}{-10pt}

%% === 設定目錄標題 (配合 titletoc) ===
\titlecontents{part}[0em]
{\center\Large}{}
{}{}
%
\titlecontents{chapter}[0em]
{}{\large\bf{第\CJKnumber{\thecontentslabel}章~~}}
{}{~~\titlerule*{.}\bf\contentspage}
%
\titlecontents{section}[2em]
{}{第\CJKnumber{\thecontentslabel}節\quad}
{}{~~\titlerule*{.} \contentspage}
%
\titlecontents{subsection}[4em]
{}{\CJKnumber{\thecontentslabel}、}
{}{~~\titlerule*{.} \contentspage}

%% === 目錄深度 ===
\setcounter{tocdepth}{2}

%% === 定義 ===
\newtheorem{myrule}{\begin{CJK}{UTF8}{bkai}原理\end{CJK}}[section]
\newtheorem{mythm}{\begin{CJK}{UTF8}{bkai}定理\end{CJK}}[section]
\newtheorem{mydef}{\begin{CJK}{UTF8}{bkai}定義\end{CJK}}[section]
\newtheorem*{mydef*}{\begin{CJK}{UTF8}{bkai}定義\end{CJK}}
\newtheorem{mypropo}{\begin{CJK}{UTF8}{bkai}性質\end{CJK}}[section]
\numberwithin{equation}{section}
\renewenvironment{proof}{\begin{CJK}{UTF8}{bkai}\textbf{證明}\end{CJK}}{\qed}
\newenvironment{mysol}{\begin{CJK}{UTF8}{bkai}\textbf{解答}\end{CJK}}{\qed}

\newcommand*\bigo{\ensuremath{\mathcal{O}}}

\renewcommand{\codename}{程式碼}
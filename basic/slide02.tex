\documentclass[utf8]{beamer}

%% === CJK 套件 ===
\usepackage{CJKutf8,CJKnumb}                 % 中文套件
%% === AMS 標準套件 ===
\usepackage{amsmath,amsfonts,amssymb,amsthm} % 數學符號
\usepackage{ulem}
%% ===  ===
%\usepackage[chapter]{algorithm}              % 演算法套件
%\usepackage[noend]{algpseudocode}            % pseudocode 套件
\usepackage{listings}                        % 程式碼
%% === TikZ 套件 ===
\usepackage{tikz,tkz-graph,tkz-berge}        % 繪圖
\usepackage{multicol}
\usepackage{xkeyval,xargs}
\usepackage{xcolor}

\usetheme{Boadilla}
\usecolortheme{whale}

\setbeamertemplate{items}[circle]

%% === 設定 C++ 格式 ===
\lstset{%
  language=C++,                     % 設定語言
  %% === 空白, tab 相關 ===
  tabsize=2,                            % 設定 tab = 多少空白
  %showspaces=true,                      % 設定是否標示空白
  %showtabs=true,                        % 設定是否標示 tab
  %tab=\rightarrowfill,                  % 設定 tab 樣式
  %% === 行數相關 ===
  %numbers=left,                         % 行數標示位置
  %stepnumber=1,                         % 每隔幾行標示行數
  %numberstyle=\tiny,
  %% === 顏色設定 ===
  basicstyle=\ttfamily,
  keywordstyle=\color{blue}\ttfamily,
  stringstyle=\color{red!50!brown}\ttfamily,
  commentstyle=\color{green!50!black}\ttfamily,
  %identifierstyle=\color{black}\ttfamily,
  emphstyle=\color{purple}\ttfamily,
  extendedchars=false,
  texcl=true,
  moredelim=[l][\color{magenta}]{\#}
}

\begin{document}
\begin{CJK}{UTF8}{bkai}

\title{基礎程式設計技巧(二)\\程式與結構}
\author{許胖}
\institute[PCSH]{板燒高中}

\section{程式控制}
\begin{frame}
  \frametitle{大綱}
  \begin{multicols}{2}
    \tableofcontents[currentsection]
  \end{multicols}
\end{frame}

\subsection{選擇結構}

\begin{frame}
  \frametitle{選擇結構}
\end{frame}

\subsection{迴圈結構}
\subsection{陣列}

\begin{frame}
  \frametitle{練習題}
  \begin{exampleblock}{\href{http://unfortunate-dog.github.io/articles/100/p10035/}{UVa 10035 - Primary Arithmetic}}<+->
  \label{uva:10035}
  不難。
  \end{exampleblock}
  \begin{exampleblock}{\href{http://unfortunate-dog.github.io/articles/100/p10038/}{UVa 10038 - Jolly Jumpers}}<+->
  \label{uva:10038}
  讀懂題意就不難。
  \end{exampleblock}
  \begin{exampleblock}{\href{http://unfortunate-dog.github.io/articles/1/p109/}{UVa 109 - SCUD Busters}}<+->
  \label{uva:591}
  這一題敘述和公式稍嫌複雜,但是用我們之前所學的工具依然可以輕鬆解決。
  \end{exampleblock}
\end{frame}

\section{函數}
\subsection{傳值呼叫}
\subsection{傳址呼叫}
\subsection{傳參考呼叫}
\subsection{函數多載}

\section{程式技巧}
\subsection{函式化}
\subsection{define與inline}

\section{C++物件導向}
\subsection{物件與類別}
\subsection{建構子與解構子}
\subsection{運算子多載}

\section{暴力搜尋法}

\clearpage
\end{CJK}
\end{document}
\documentclass[utf8]{beamer}

%% === CJK 套件 ===
\usepackage{CJKutf8,CJKnumb}                 % 中文套件
%% === AMS 標準套件 ===
\usepackage{amsmath,amsfonts,amssymb,amsthm} % 數學符號
\usepackage{ulem}
%% ===  ===
%\usepackage[chapter]{algorithm}              % 演算法套件
%\usepackage[noend]{algpseudocode}            % pseudocode 套件
\usepackage{listings}                        % 程式碼
%% === TikZ 套件 ===
\usepackage{tikz,tkz-graph,tkz-berge}        % 繪圖
\usepackage{multicol}
\usepackage{xkeyval,xargs}
\usepackage{xcolor}

\usetheme{Boadilla}
\usecolortheme{whale}

\setbeamertemplate{items}[circle]

%% === 設定 C++ 格式 ===
\lstset{%
  language=C++,                     % 設定語言
  %% === 空白, tab 相關 ===
  tabsize=2,                            % 設定 tab = 多少空白
  %showspaces=true,                      % 設定是否標示空白
  %showtabs=true,                        % 設定是否標示 tab
  %tab=\rightarrowfill,                  % 設定 tab 樣式
  %% === 行數相關 ===
  %numbers=left,                         % 行數標示位置
  %stepnumber=1,                         % 每隔幾行標示行數
  %numberstyle=\tiny,
  %% === 顏色設定 ===
  basicstyle=\ttfamily,
  keywordstyle=\color{blue}\ttfamily,
  stringstyle=\color{red!50!brown}\ttfamily,
  commentstyle=\color{green!50!black}\ttfamily,
  %identifierstyle=\color{black}\ttfamily,
  emphstyle=\color{purple}\ttfamily,
  extendedchars=false,
  texcl=true,
  moredelim=[l][\color{magenta}]{\#}
}

\begin{document}
\begin{CJK}{UTF8}{bkai}

\title{基礎程式設計技巧(四)\\程式與實作}
\author{許胖}
\institute[PCSH]{板燒高中}

\begin{frame}
  \titlepage
\end{frame}
\begin{frame}
  \frametitle{大綱}
  \begin{multicols}{2}
    \tableofcontents
  \end{multicols}
\end{frame}

\section{簡介}
\begin{frame}
  \frametitle{大綱}
  \begin{multicols}{2}
    \tableofcontents[currentsection]
  \end{multicols}
\end{frame}

\section{程式語言}
\begin{frame}
  \frametitle{大綱}
  \begin{multicols}{2}
    \tableofcontents[currentsection]
  \end{multicols}
\end{frame}

\subsection{機器語言與組合語言}

\subsection{高階語言的種類}

\subsection{程式編譯}


\section{模擬問題}
\begin{frame}
  \frametitle{大綱}
  \begin{multicols}{2}
    \tableofcontents[currentsection]
  \end{multicols}
\end{frame}

\section{數學問題}
\begin{frame}
  \frametitle{大綱}
  \begin{multicols}{2}
    \tableofcontents[currentsection]
  \end{multicols}
\end{frame}

\subsection{數學函數}

\subsection{質數問題}

\subsection{質因數分解}

\clearpage
\end{CJK}
\end{document}
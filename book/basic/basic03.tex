\ifx \allfiles \undefined
\documentclass[12pt,a4paper,oneside]{article}

% ======================================================================
%
%  引入套件
%    * CJK 套件 - CJK, CJKnumb
%    * AMS 標準套件 - amsmath, amsfonts, amssymb, amsthm
%    * 章節內容
%      - titletoc: 目錄修改套件
%      - titlesev: 美化章節標題
%      - imakeidx: 索引製作
%    * 演算法相關
%      - algorithm: 演算法環境套件
%      - algpseudicode: pseudocode 套件
%      - listings: 程式碼套件
%    * TikZ 套件
%      - tikz: 本體
%      - tkz-graph: 圖論套件
%      - tkz-berge: 另一個圖論套件
%
% ======================================================================
%% === 基本設定 ===
\usepackage{CJKutf8,CJKnumb}
\usepackage{amsmath,amsfonts,amssymb,amsthm}
\usepackage{enumitem}                           % 修改 enumerate, item
\usepackage{bbding}
\usepackage{titletoc,titlesec,imakeidx}
\usepackage{imakeidx}
\usepackage[unicode]{hyperref,xcolor}
\hypersetup{
    colorlinks,
    linkcolor={blue!100!black},
    citecolor={blue!75!black},
    urlcolor={blue!50!black}
}
\usepackage{import}
%% === 演算法相關 ===
\usepackage[chapter]{algorithm}
\usepackage[noend]{algpseudocode}
\usepackage{listings}
%% === TikZ ===
\usepackage{tikz,tkz-graph,tkz-berge}
\usepackage{ifthen}
%% ===  ===
\usepackage{xkeyval,xargs}
\usepackage{ulem}

%% === itemize,enumerate 設定 ===
%  使用 enumitem 套件
\setlist[itemize]{itemsep=0pt,parsep=0pt}
\setlist[enumerate]{itemsep=0pt,parsep=0pt}

%% === 設定 C++ 格式 ===
%\lstset{
  %language=[11]C++,                     % 設定語言
  %% === 空白, tab 相關 ===
  %tabsize=2,                            % 設定 tab = 多少空白
  %showspaces=true,                      % 設定是否標示空白
  %showtabs=true,                        % 設定是否標示 tab
  %tab=\rightarrowfill,                  % 設定 tab 樣式
  %% === 行數相關 ===
  %numbers=left,                         % 行數標示位置
  %stepnumber=3,                         % 每隔幾行標示行數
  %numberstyle=\tiny,
  %% === 顏色設定 ===
  %basicstyle=\ttfamily,
  %keywordstyle=\color{blue}\ttfamily,
  %stringstyle=\color{red}\ttfamily,
  %commentstyle=\color{green}\ttfamily,
  %morecomment=[l][\color{magenta}]{\#},
  %morekeywords={}
%}

% ======================================================================
%
%  設定頁面格式
%
% ======================================================================
%% === 設定頁面格式 ===
%\hoffset         = 10pt                      % 水平位移,預設為 0pt
\voffset         = -15pt                     % 垂直位移,預設為 0pt
\oddsidemargin   = 0pt                       % 預設為 31pt
%\topmargin       = 20pt                      % 預設為 20pt
%\headheight      = 12pt                      % header 的高度,預設為 12pt
%\headsep         = 25pt                      % header 和 body 的距離,預設為 25pt
\textheight      = 620pt                     % body 內文部分的高度,預設為 592pt
\textwidth       = 450pt                     % body 內文部分的寬度,預設為 390pt
%\marginparsep    = 10pt                      % margin note 和 body 的距離,預設為 10pt
%\marginparwidth  = 35pt                      % margin note 的寬度,預設為 35pt
%\footskip        = 30pt                      % footer 高度 + footer 和 body 的距離,預設為 30pt

% ======================================================================
%
%  其他
%
% ======================================================================
\linespread{1.24}
\makeindex[name=noun]        % 索引生成

\begin{document}
\begin{CJK}{UTF8}{bkai}

\subimport{./config/}{docsetting.tex}
\setcounter{section}{2}

\fi

\section{字串處理}

\subsection{字元與字串}

\subsubsection{字元與 \texttt{ASCII}}

\paragraph{}\texttt{C++} 中的\index{字元}{\textbf{{\color{red}字元} (Character)}} \lstinline!char! 其實是儲存一個 0 到 255 的整數,在電腦中有一個\textbf{符號表},每個符號都有他各自的編號。輸出字元時,計算機就會自動將 \lstinline!char! 裡面的整數去查符號表,印出對應符號,這個表格我們稱為 \index{ASCII}{\color{red}\textbf{ASCII 碼}}。雖然 \lstinline!char! 印出來是符號,但實際上儲存的是整數。
\paragraph{}ASCII 碼網路上都能查到,在這邊只提到幾個重要的觀念就好,各位讀者可以邊看網路上查到的 ASCII 碼邊看接下來的內容。
\paragraph{}字元表示方法是用單引號包上一個符號,如:\lstinline!'0'!。每個符號都有一個編號,例如 \lstinline!'0'! 的編號是 48,其他常用的字元編號如表 \ref{string:mani:table:char:ascii}。

\begin{table}[h!]
  \centering
  \begin{tabular}{|c||c|c|c|c|c|}
  \hline
  \textbf{字元} & \lstinline!'0'! & \lstinline!'A'! & \lstinline!'a'! & \lstinline!' '! & \lstinline!'\n'!\\
  \hline
  \textbf{編號} & 48 & 65 & 97 & 32 & 10\\
  \hline
  \textbf{備註} & & & & 空白 & 換行字元\\
  \hline
  \end{tabular}
  \caption{常用字元編號}
  \label{string:mani:table:char:ascii}
\end{table}

\paragraph{特殊字元}有些字元因為沒辦法直接用單引號包住符號的方式表示,就會用「倒斜線+字元」來代表,表 \ref{string:mani:table:char:special} 是一些常見的特殊字元。

\begin{table}[h!]
  \centering
  \begin{tabular}{|c|c|c|}
  \hline
  \textbf{字元} & \textbf{意義} & \textbf{備註}\\
  \hline\hline
  \lstinline!'\n'! & 換行字元 &\\
  \hline
  \lstinline!'\t'! & Tab &\\
  \hline
  \lstinline!'\r'! & 迴車鍵 & Windows 系統中以 \lstinline!\r\n! 代表換行\\
  \hline
  \lstinline!'\''! & 單引號 &\\
  \hline
  \lstinline!'\"'! & 雙引號 &\\
  \hline
  \lstinline!'\0'! & 空字元 & 用來代表字串的結束\\
  \hline
  \lstinline!'\\'! & 倒斜線 & 倒斜線被用做跳脫字元,因此要用兩個倒斜線表示\\
  \hline
  \end{tabular}
  \caption{常用特殊字元}
  \label{string:mani:table:char:special}
\end{table}

\paragraph{常用技巧:字元判斷}在 ASCII 碼中,有三個區塊是連續的:
\begin{itemize}
\item 數字區塊:\lstinline!'0'! 到 \lstinline!'9'!
\item 大寫字母區塊:\lstinline!'A'! 到 \lstinline!'Z'!
\item 小寫字母區塊:\lstinline!'a'! 到 \lstinline!'z'!
\end{itemize}

\paragraph{}因為 \lstinline!char! 實際上是存整數,所以可以用大於小於來判斷,程式碼 \ref{string:mani:code:is:digit} 可以判斷一個字元是否是數字字元。
\begin{code}[h!]
  \centering
  \begin{tabular}{c}
  \begin{lstlisting}
bool myIsDigit(char ch) {
  return '0' <= ch && ch <= '9';
}
  \end{lstlisting}
  \end{tabular}
  \caption{判斷數字字元}
  \label{string:mani:code:is:digit}
\end{code}

\paragraph{}在 \index{標頭檔!cctype}{\texttt{<cctype>}} 中有一些可以判斷字元類型的函式,常用的如表 \ref{string:mani:table:cctype},留意這些函式傳進去的參數型態就直接是 \lstinline!int!。

\begin{table}[h!]
  \centering
  \begin{tabular}{|c|c|c|}
  \hline
  \textbf{函式} & \textbf{範圍} & \textbf{意義}\\
  \hline\hline
  \lstinline!int isalnum(int)! & A 到 Z、a 到 z、0 到 9 & 判斷是否為英文字母或數字字元\\
  \hline
  \lstinline!int isalpha(int)! & A 到 Z、a 到 z & 判斷是否為英文字母\\
  \hline
  \lstinline!int isdigit(int)! & 0 到 9 & 判斷是否為數字字元\\
  \hline
  \lstinline!int islower(int)! & a 到 z & 判斷是否為小寫字母\\
  \hline
  \lstinline!int isupper(int)! & A 到 Z & 判斷是否為大寫字母\\
  \hline
  \end{tabular}
  \caption{\lstinline!<cctype>! 常用函式}
  \label{string:mani:table:cctype}
\end{table}

\paragraph{}不過臨時記不得這些函式,筆者建議自己寫一個,不難寫。

\paragraph{常用技巧:計算數字}用同樣的概念,可以把一個數字字元「轉換」成 \lstinline!int! 的 0。字元 \lstinline!'0'! 的編號為 48,因此我們將 \lstinline!'0' - 48! 就可以取得實際的 \lstinline!int! 值,但這個方法稍嫌笨重,因為我們必須記得 \lstinline!'0'! 的編號,在 \texttt{C++} 中有提供字元相減的方法,於是我們可以寫為程式碼 \ref{string:mani:code:char:to:number},將一個數字字元減去 \lstinline!'0'!。

\begin{code}[h!]
  \centering
  \begin{tabular}{c}
  \begin{lstlisting}
int charToNumber(char ch) {
  return ch - '0';
}
  \end{lstlisting}
  \end{tabular}
  \caption{數字字元轉換為 \lstinline!int!}
  \label{string:mani:code:char:to:number}
\end{code}

\paragraph{}同樣的方法也可以用在具有連續區間的大寫字母、小寫字母。

\subsubsection{\texttt{C++} 字串與 \texttt{C} 字串}

\paragraph{}\texttt{C++} 的字串為 \lstinline!string! 物件,需要引入標頭檔 \index{標頭檔!string}{\lstinline!<string>!}。這裡要講 \texttt{C} 字串,\texttt{C} 字串事實上就是{\color{red}\textbf{字元陣列}},宣告如程式碼 \ref{string:mani:code:c:string:declare}。

\begin{code}[h!]
  \centering
  \begin{tabular}{c}
  \begin{lstlisting}
char str[110];
  \end{lstlisting}
  \end{tabular}
  \caption{\texttt{C} 字串宣告}
  \label{string:mani:code:c:string:declare}
\end{code}

\paragraph{}使用 \texttt{C} 語言的字串時,就有很多東西要注意:第一、字串本身是用 \lstinline!'\0'! 來當結尾,{\color{red}\textbf{注意}}!不是阿拉伯數字的 \lstinline!'0'! 而是 ASCII 碼為 0 的 \lstinline!'\0'! (表 \ref{string:mani:table:char:special}),代表「字串的結尾」。

\begin{code}[h!]
  \centering
  \begin{tabular}{c}
  \begin{lstlisting}
char str[10] = "bird";
  \end{lstlisting}
  \end{tabular}
  \caption{\texttt{C} 字串}
  \label{string:mani:code:c:string:assign}
\end{code}

\paragraph{}例如程式碼 \ref{string:mani:code:c:string:assign} 的字串為 \lstinline!"bird"!,事實上儲存情形如圖 \ref{string:mani:fig:char:array}。

\begin{figure}[h!]
  \centering
  \begin{tikzpicture}
  \def\sz{1.2cm};
  \tikzset{array block/.style={
    draw,
    rectangle,
    minimum height=\sz,
    minimum width=\sz
  }};
  \def\arr{b,i,r,d,0,?,?,?,?,?};
  \foreach \item [count=\i] in \arr {
    \ifthenelse{\i=5}{
      \node[array block] at (\i*\sz,0) {\lstinline!\\0!};
    }{
      \node[array block] at (\i*\sz,0) {\texttt{\item}};
    };
    \pgfmathparse{int(\i-1)};
    \edef\res{\pgfmathresult};
    \node at (\i*\sz,.75*\sz) {\texttt{[\res]}};
  }
  \end{tikzpicture}
  \caption{字串就是字元陣列}
  \label{string:mani:fig:char:array}
\end{figure}

\paragraph{}因為 \lstinline!'\0'! 被當作「字串的結尾」 (很重要所以還要再說一次),所有 \texttt{C} 字串函數的操作都會跟這個有關。

\subsubsection{\texttt{C} 字串函數}

\paragraph{}\texttt{C} 字串函數都在 \index{標頭檔!cstring}{\lstinline!<cstring>!} 底下,以下解說幾個常用的 \texttt{C} 字串函數,如程式碼 \ref{string:mani:code:string:function}。

\begin{code}[h!]
  \centering
  \begin{tabular}{c}
  \begin{lstlisting}
size_t strlen(const char *str);
int    strcmp(const char *str1, const char *str2);
char*  strcpy(char* dest, const char* src);
char* strncpy(char* dest, const char* src, size_t num);
char*  strcat(char* dest, const char* src);
char* strncat(char* dest, const char* src, size_t num);
  \end{lstlisting}
  \end{tabular}
  \caption{常用的字串函數}
  \label{string:mani:code:string:function}
\end{code}

\paragraph{\lstinline!strlen! 函數}\lstinline!strlen! 函數可以知道一個 \texttt{C} 字串的長度,參數就是傳一個字串指標,這個函數就是從一開始的位址往後掃,直到碰到 \lstinline!'\0'! 為止,並回傳長度值。

\begin{code}[h!]
  \centering
  \begin{tabular}{c}
  \begin{lstlisting}
strlen("bird");
  \end{lstlisting}
  \end{tabular}
  \caption{\lstinline!strlen! 範例}
  \label{string:mani:code:strlen:example}
\end{code}

\paragraph{}程式碼 \ref{string:mani:code:strlen:example} 回傳的長度值為 4。當然我們也可以對字串變數做 \lstinline!strlen!,程式碼 \ref{string:mani:code:strlen:example:variable} 中 \lstinline!num! 字串的長度為 \lstinline!10!。

\begin{code}[h!]
  \centering
  \begin{tabular}{c}
  \begin{lstlisting}
char num[110] = "0123456789";
strlen(num);
  \end{lstlisting}
  \end{tabular}
  \caption{\lstinline!strlen! 範例}
  \label{string:mani:code:strlen:example:variable}
\end{code}

\paragraph{}此外,\lstinline!strlen! 是{\color{blue}\textbf{函式}},它和 \texttt{C++} 字串的 \lstinline!str.size()! 不同的地方在於:每次呼叫 \lstinline!strlen! 就會重新計算一次字串長度,像程式碼 \ref{string:mani:code:strlen:for:naive} 的寫法非常浪費時間,每次迴圈都會重新呼叫一次 \lstinline!strlen!。

\begin{code}[h!]
  \centering
  \begin{subcode}{.58\textwidth}
    \centering
    \begin{tabular}{c}
    \begin{lstlisting}
for (i = 0; i < strlen(str); i++)
{
  // do something
}
    \end{lstlisting}
    \end{tabular}
    \caption{直觀寫法}
    \label{string:mani:code:strlen:for:naive}
  \end{subcode}
  ~
  \begin{subcode}{.38\textwidth}
    \centering
    \begin{tabular}{c}
    \begin{lstlisting}
int len = strlen(str);
for (i = 0; i < len; i++)
{
  // do something
}
    \end{lstlisting}
    \end{tabular}
    \caption{較好的寫法}
    \label{string:mani:code:strlen:for:better}
  \end{subcode}
  \caption{注意 \lstinline!strlen! 的用法}
  \label{string:mani:code:strlen:for}
\end{code}

\paragraph{}但很多時候字串的長度沒有改變,因此會浪費很多時間,比較好的做法是用一個變數儲存長度,再下去執行迴圈,如程式碼 \ref{string:mani:code:strlen:for:better}。
\paragraph{}補充,雖然現代的編譯器大多會去辨別這一情況加以優化,但盡量不要依賴編譯器,最好從一開始養成好習慣,才不會被坑。

\paragraph{\lstinline!strcmp! 函數}\lstinline!strcmp! 函數用來比較兩個 \texttt{C} 字串,參數為兩個字串指標。它會從兩個字串一開始的字元逐個比較,比較到\textbf{其中一個}為 \lstinline!'\0'! 為止,\lstinline!strcmp! 的功能和 \texttt{C++} 字串之間用 \lstinline!==!、\lstinline!>!、\lstinline!<! 來比較的功能類似。回傳值分成三類:

\begin{itemize}
\item 等於 \lstinline!0!:代表兩個字串相等。
\item 大於 \lstinline!0!:代表 \lstinline!str1! 的字典順序大於 \lstinline!str2!,通常是回傳 \lstinline!1!。
\item 小於 \lstinline!0!:代表 \lstinline!str1! 的字典順序小於 \lstinline!str2!,通常是回傳 \lstinline!-1!。
\end{itemize}

\paragraph{\lstinline!strcpy! 函數}\lstinline!strcpy! 函數就是把 \lstinline!src! 字串複製到 \lstinline!dest! 字串,複製的原理和上述函式類似:就是從第一個字元開始複製,直到遇到 \lstinline!'\0'! 為止。接著對 \lstinline!dest! 最尾端補 \lstinline!'\0'!。\lstinline!strcpy! 相當於在 \texttt{C++} 中 \lstinline!dest = src!。
\paragraph{}使用 \lstinline!strcpy! 時需要特別{\color{blue}\textbf{注意}},\lstinline!strcpy! 函數{\color{red}\textbf{不會檢查}} \lstinline!dest! 的長度,換句話說,如果 \lstinline!src! 比 \lstinline!dest! 長,\lstinline!strcpy! 會持續複製直到複製完為止,這對程式而言是非常危險的,因為寫出字串的範圍很有可能會寫到很重要的記憶體,然後{\color{red}\textbf{電腦就炸了}}。
\paragraph{}其他較為安全的替代方案,就是使用 \lstinline!strncpy! 函數,\lstinline!strncpy! 的第三個參數代表\textbf{最多}複製幾個字元,不過要注意 \lstinline!strncpy! 只負責複製字元,複製完後{\color{blue}\textbf{不會補 \lstinline!'\0'!}}。

\paragraph{\lstinline!strcat! 函數}\lstinline!strcat! 函數主要是把 \lstinline!src! 字串接到 \lstinline!dest! 字串後面,它是從 \lstinline!dest! 字串的\textbf{第一個} \lstinline!'\0'! 開始串接,因此也和 \lstinline!strcpy! 一樣可能會造成寫出記憶體的問題,對應的函數是 \lstinline!strncat!。\lstinline!strcat! 函數相當於 \texttt{C++} 字串中,直接做 \lstinline!dest += src!。

\paragraph{}其它有關字串的函數還有:\lstinline!strstr!、\lstinline!strchr!、\lstinline!strrchr!、\lstinline!strtok!、\lstinline!strspn! 等,這些函數在此不贅述。

\subsubsection{字串轉換}

\paragraph{}實作上,\texttt{C++} 字串操作上來得 \texttt{C} 字串\textbf{容易},但因為 \texttt{C++} 字串是物件,因此操作 \texttt{C} 字串會比 \texttt{C++} 字串來得快。我們可以選擇混合使用這兩種字串,在不難處理又快速時我們選擇 \texttt{C} 字串;而在 \texttt{C} 字串比較難解決的事情時,我們可以利用方便的 \texttt{C++} 字串來處理。

\paragraph{\texttt{C++} 字串轉 \texttt{C} 字串}如果要將一個 \texttt{C++} 字串轉成 \texttt{C} 字串,我們利用 \index{標頭檔!string}{\lstinline!<string>!} 中的一個成員函式 \lstinline!c_str()!。

\begin{code}[h!]
  \centering
  \begin{tabular}{c}
  \begin{lstlisting}
string str = "bird";
cout << str.c_str() << endl;
  \end{lstlisting}
  \end{tabular}
  \caption{\texttt{C++} 字串轉 \texttt{C} 字串}
  \label{string:mani:code:string:cpp:to:c}
\end{code}

\paragraph{}要注意的是,\lstinline!c_str()! 產生一個\textbf{唯讀}的 \texttt{C} 字串。
\paragraph{\texttt{C} 字串轉 \texttt{C++} 字串}正確將 \texttt{C} 字串轉成 \texttt{C++} 字串的方法則是將 \texttt{C} 字串丟進 \texttt{C++} 字串中,如程式碼 \ref{string:mani:code:string:c:to:cpp}。

\begin{code}[h!]
  \centering
  \begin{tabular}{c}
  \begin{lstlisting}
char str1[110] = "bird";
string str2 = str1;
cout << str2.size() << endl;
cout << str2 << endl;
  \end{lstlisting}
  \end{tabular}
  \caption{\texttt{C} 字串轉 \texttt{C++} 字串}
  \label{string:mani:code:string:c:to:cpp}
\end{code}

\subsubsection{字串練習}

\paragraph{}可能有些人看完上面的敘述後,可能還不是很理解這些函數的用途。要理解 \texttt{C} 字串的用法不難,但如果要完全掌握住這些函數,還需要配合指標的觀念,以下問題,筆者就不寫上解答,留給讀者思考、討論。

\begin{enumerate}
\item 假設現在有一個字串
\begin{code}[h!]
  \centering
  \begin{tabular}{c}
  \begin{lstlisting}
char str1[110] = "abcdefghijklmnopqrstuvwxyz";
  \end{lstlisting}
  \end{tabular}
\end{code}
  \begin{enumerate}
  \item 它的長度為何?
  \item 用 \lstinline!sizeof! 和 \lstinline!strlen! 有何不同呢?
  \item 請問下面的語句和上面又有什麼差異呢?
  \begin{code}[h!]
    \centering
    \begin{tabular}{c}
    \begin{lstlisting}
strlen("abcdefghijklmnopqrstuvwxyz");
sizeof("abcdefghijklmnopqrstuvwxyz");
    \end{lstlisting}
    \end{tabular}
  \end{code}
  \end{enumerate}
\item 有一字串
\begin{code}[h!]
  \centering
  \begin{tabular}{c}
  \begin{lstlisting}
char str2[110] = "bird";
char str3[4];
  \end{lstlisting}
  \end{tabular}
\end{code}
  \begin{enumerate}
  \item 我們可以用 \lstinline!strcpy(str3, str2)! 嘛?
  \item 如果不行,我們使用 \lstinline!strncpy(str3, str2, 4);! 來避免超出記憶體呢?試著印出來觀察看看。
  \end{enumerate}
\item 有一字串
\begin{code}[h!]
  \centering
  \begin{tabular}{c}
  \begin{lstlisting}
char str4[110] = "cat\0bird";
  \end{lstlisting}
  \end{tabular}
\end{code}
  \begin{enumerate}
  \item 這個字串的長度為何?
  \item \lstinline!strlen(str4 + 4)! 的回傳值為何?\lstinline!strlen(str4 + 8)! 呢?
  \item 若我們使用 \lstinline!strcpy(str4, "dog")!,會得到什麼結果?此時執行下面語句會有什麼反應呢?
  \begin{code}[h!]
    \centering
    \begin{tabular}{c}
    \begin{lstlisting}
cout << str4 << " " << str4 + 4 << endl;
    \end{lstlisting}
    \end{tabular}
  \end{code}
  \item 若我們對原先的 \lstinline!str4! 執行 \lstinline!strcat(str4, "dog")!,有得到你預期的結果嘛?
  \item 要是 \lstinline!strcat(str4 + 4, "dog")! 呢?
  \end{enumerate}
\item 假設有一 \texttt{C++} 字串和一 \texttt{C} 字串
\begin{code}[h!]
  \centering
  \begin{tabular}{c}
  \begin{lstlisting}
string str5 = "bird";
char str6[110];
  \end{lstlisting}
  \end{tabular}
\end{code}
\\要怎樣將 \lstinline!str5! 複製給 \lstinline!str6! 呢?
\item 現有兩字串
\begin{code}[h!]
  \centering
  \begin{tabular}{c}
  \begin{lstlisting}
char str7[110] = "cat\0bird";
char str8[110];
  \end{lstlisting}
  \end{tabular}
\end{code}
\\試比較 \lstinline!strcpy(str8, str7);! 和 \lstinline!memcpy(str8, str7, sizeof(str7));! 的不同。
\end{enumerate}

\subsection{輸入與輸出}

\subsubsection{\lstinline!scanf! 和 \lstinline!printf!}

\paragraph{}\texttt{C++} 中常使用的輸入輸出是 \lstinline!cin! 和 \lstinline!cout!,有時候 \lstinline!cin! 和 \lstinline!cout! 的速度並不能滿足我們的需求,這時候就需要使用 \texttt{C} 語言本身的輸入輸出。
\paragraph{}\texttt{C} 語言的輸入輸出在 \index{標頭檔!cstdio}{\lstinline!<cstdio>!} 內,輸入的函數為 \lstinline!scanf!,而輸出的函數為 \lstinline!printf!。
\paragraph{}{\color{red}\textbf{注意}}!\texttt{C} 語言輸入輸出是\textbf{函式},這兩個函式的用法和原本 \texttt{C++} 的相比較為繁瑣,但也有比較方便的地方。

\begin{code}[h!]
  \centering
  \begin{tabular}{c}
  \begin{lstlisting}
int  scanf(const char *format, ...);
int printf(const char *format, ...);
  \end{lstlisting}
  \end{tabular}
  \caption{\lstinline!scanf! 和 \lstinline!printf!}
  \label{string:mani:code:scanf:and:printf}
\end{code}

\paragraph{}\lstinline!scanf! 和 \lstinline!printf! 的參數中,後面 \lstinline!...! 稱為\textbf{不定參數},代表參數的數量是不固定的,決定參數的數量是靠前面的 \lstinline!format! 字串決定,這個字串我們稱為\index{格式字串}{\color{red}\textbf{格式字串}}。
\paragraph{}格式字串可以像我們平常輸出字串一樣作法,如程式碼 \ref{string:mani:code:printf:string}。

\begin{code}[h!]
  \centering
  \begin{tabular}{c}
  \begin{lstlisting}
  printf("Hello world!\n");
  \end{lstlisting}
  \end{tabular}
  \caption{輸出字串}
  \label{string:mani:code:printf:string}
\end{code}

\paragraph{}記得 \lstinline!endl! 屬於 \texttt{C++} 當中的換行,在此須使用換行字元 \lstinline!'\n'! 來換行。輸出整數變數,在格式字串中我們用 \lstinline!"%d"! 來代表,每個 \lstinline!"%d"! 都代表著一個整數,如程式碼 \ref{string:mani:code:print:int}。

\begin{code}[h!]
  \centering
  \begin{subcode}{.33\textwidth}
    \centering
    \begin{tabular}{c}
    \begin{lstlisting}
printf("%d\n", 3);
    \end{lstlisting}
    \end{tabular}
    \caption{印出常數}
    \label{string:mani:code:print:int:const}
  \end{subcode}
  ~
  \begin{subcode}{.33\textwidth}
    \centering
    \begin{tabular}{c}
    \begin{lstlisting}
int x = 3;
printf("%d\n", x);
    \end{lstlisting}
    \end{tabular}
    \caption{印出 \lstinline!int! 變數}
    \label{string:mani:code:print:int:variable}
  \end{subcode}
  \caption{印出整數}
  \label{string:mani:code:print:int}
\end{code}

\paragraph{}程式碼 \ref{string:mani:code:print:int:const} 直接印出數字,也可以寫為 \lstinline!printf("3\n");!,程式碼 \ref{string:mani:code:print:int:variable} 會印出變數 \lstinline!x! 的值,印出的位置在 \lstinline!"%d"! 處。
\paragraph{}輸入的話,因為我們是呼叫函數,若要改到變數值就需要使用傳址呼叫,因此輸入一個整數寫為程式碼 \ref{string:mani:code:scan:int}。

\begin{code}[h!]
  \centering
  \begin{tabular}{c}
  \begin{lstlisting}
int x;
scanf("%d", &x);
  \end{lstlisting}
  \end{tabular}
  \caption{輸入一個整數}
  \label{string:mani:code:scan:int}
\end{code}

\paragraph{}同樣地,\lstinline!unsigned int!、\lstinline!long long!、\lstinline!unsigned long long!、\lstinline!float!、\lstinline!double! 都有對應的的格式,如表 \ref{string:mani:table:printf:scanf:format}。

\begin{table}[h!]
  \centering
  \begin{tabular}{|c|c|c|}
  \hline
  \textbf{型態} & \textbf{對應格式} & \textbf{備註}\\
  \hline\hline
  \lstinline!unsigned int! & \lstinline!"%u"! &\\
  \hline
  \lstinline!long long! & \lstinline!"%lld"! & windows 環境下可能會用 \lstinline!"%I64d"!\\
  \hline
  \lstinline!unsigned long long! & \lstinline!"%llu"! & windows 環境下可能會用 \lstinline!"%I64u"!\\
  \hline
  \lstinline!float! & \lstinline!"%f"! &\\
  \hline
  \lstinline!double! & \lstinline!"%lf"! & \lstinline!printf! 時須用 \lstinline!"%f"!\\
  \hline
  \lstinline!char! & \lstinline!"%c"! &\\
  \hline
  \lstinline!char[]! & \lstinline!"%s"! & \texttt{C++} 的字串 \lstinline!string! 沒辦法直接使用 \lstinline!scanf!、\lstinline!printf!\\
  \hline
  \end{tabular}
  \caption{\lstinline!scanf! 和 \lstinline!printf! 格式表}
  \label{string:mani:table:printf:scanf:format}
\end{table}

\paragraph{}因為 \lstinline!%! 在 \lstinline!scanf! 和 \lstinline!printf! 當中作為跳脫字元,因此印出 \lstinline!%! 要使用 \lstinline!"%%"!。
\paragraph{}和 \lstinline!cin! 類似,\lstinline!scanf! 中也不會用空白和換行,如程式碼 \ref{string:mani:code:scanf:example}。

\begin{code}[h!]
  \centering
  \begin{tabular}{c}
  \begin{lstlisting}
scanf("%d%d%d", &a, &b, &c);
  \end{lstlisting}
  \end{tabular}
  \caption{\lstinline!scanf! 範例}
  \label{string:mani:code:scanf:example}
\end{code}

\paragraph{字元讀取問題}讀取字元時,\lstinline!cin! 和 \lstinline!scanf! 的行為會不一致。

\begin{code}[h!]
  \centering
  \begin{tabular}{c}
  \begin{lstlisting}
char a, b;
cin >> a >> b;
scanf("%c%c", &a, &b);
printf("\'%c\' \'%c\'\n", a, b);
  \end{lstlisting}
  \end{tabular}
  \caption{\lstinline!cin! 和 \lstinline!scanf! 行為不一致}
  \label{string:mani:code:cin:scanf:behavior}
\end{code}

\paragraph{}程式碼 \ref{string:mani:code:cin:scanf:behavior} 第 2 行和第 3 行中可以嘗試輸入「\lstinline!1 2!」,可以發現 \lstinline!cin! 會跳過空白,可是 \lstinline!scanf! 並不會。
\paragraph{\lstinline!scanf_s! 函數}\texttt{VC++} 在讀取字串時很多時候會被擋下,因為 \lstinline!scanf! 讀取字串不會檢查長度,會有安全問題。\lstinline!scanf_s! 函式在讀取字串時要多傳一個參數,作為最多讀取的長度,程式碼 \ref{string:mani:code:scanf:s:example} 為 \lstinline!scanf_s! 的一個範例。

\begin{code}[h!]
  \centering
  \begin{tabular}{c}
  \begin{lstlisting}
char str[5];
scanf_s("%s", str, 4);
  \end{lstlisting}
  \end{tabular}
  \caption{\lstinline!scanf_s! 的範例}
  \label{string:mani:code:scanf:s:example}
\end{code}

\paragraph{}程式碼 \ref{string:mani:code:scanf:s:example} 中,因為 \lstinline!str! 字串最後要有 \lstinline!'\0'! 字元,因此最多只能讀 4 個字元。
\paragraph{}到目前為止,可看出 \lstinline!scanf! 和 \lstinline!printf! 在用法上比 \lstinline!cin! 和 \lstinline!cout! 繁瑣,除了在效能上的優勢外,還有什麼其他優點呢?

\subsubsection{格式字串}

\paragraph{進制輸出}在 \texttt{C++} 中整數提供八進位、十進位、十六進位的輸出方法,分別是使用 \lstinline!std::oct!、\lstinline!std::dec!、\lstinline!std::hex! 這三種,如程式碼 \ref{string:mani:code:notation:cout}。

\begin{code}[h!]
  \centering
  \begin{subcode}{.43\textwidth}
    \centering
    \begin{tabular}{c}
    \begin{lstlisting}
int a = 11;
cout << oct << a << endl;
cout << dec << a << endl;
cout << hex << a << endl;
    \end{lstlisting}
    \end{tabular}
    \caption{\lstinline!cout! 版本}
    \label{string:mani:code:notation:cout}
  \end{subcode}
  ~
  \begin{subcode}{.43\textwidth}
    \centering
    \begin{tabular}{c}
    \begin{lstlisting}
int a = 11;
printf("%o\n", a); // 13
printf("%d\n", a); // 11
printf("%x\n", a); // b
    \end{lstlisting}
    \end{tabular}
    \caption{\lstinline!printf! 版本}
    \label{string:mani:code:notation:printf}
  \end{subcode}
  \caption{輸出進制比較}
  \label{string:mani:code:notation}
\end{code}

\paragraph{}程式碼 \ref{string:mani:code:notation:printf} 是對應的版本,分別使用 \lstinline!"%o"!、\lstinline!"%d"!、\lstinline!"%x"!。如要將十六進位印為大寫,\lstinline!cout! 須加上 \lstinline!std::uppercase!,\lstinline!printf! 則使用 \lstinline!"%X"!。
\paragraph{\texttt{iomanip} 格式}在 \texttt{C++} 中,標頭檔 \index{標頭檔!iomanip}{\lstinline!<iomanip>!},專門做輸入輸出的處理。表 \ref{string:mani:table:iomanip} 列出 \lstinline!<iomanip>!
中常用的\index{串流操縱符}{\textbf{串流操縱符 (Stream manipulator)}},也就是用來串在 \lstinline!cin!、\lstinline!cout! 的東西。

\begin{table}[h!]
  \centering
  \begin{tabular}{|c|c|}
  \hline
  \textbf{操縱符} & \textbf{作用}\\
  \hline\hline
  \lstinline!std::setprecision! & 設定精準度\\
  \hline
  \lstinline!std::setw! & 設定輸出寬度\\
  \hline
  \lstinline!std::setfill! & 設定填充字元\\
  \hline
  \end{tabular}
  \caption{\lstinline!<iomanip>! 常用操縱符}
  \label{string:mani:table:iomanip}
\end{table}

\lstinline!std::setprecision! 通常是用來限定輸出數字的精準度,例如程式碼 \ref{string:mani:code:set:precision},\lstinline!setprecision! 的參數為 \lstinline!5!,因此會保留 5 位有效位數。

\begin{code}[h!]
  \centering
  \begin{tabular}{c}
  \begin{lstlisting}
double f = 3.14159;
cout << setprecision(5) << f << endl; // 3.1415
cout << fixed << setprecision(5) << f << endl; // 3.14159
  \end{lstlisting}
  \end{tabular}
  \caption{設定有效位數}
  \label{string:mani:code:set:precision}
\end{code}

\paragraph{}當 \lstinline!setprecision! 加上 \lstinline!std::fixed! 的話,會變作印出小數點後 n 位,也就是\textbf{四捨五入},如程式碼 \ref{string:mani:code:set:precision} 第 3 行。
\paragraph{}\lstinline!std::setw! 代表輸出的寬度,例如程式碼 \ref{string:mani:code:setw:example},輸出的結果為 \lstinline!"   16"!。

\begin{code}[h!]
  \centering
  \begin{tabular}{c}
  \begin{lstlisting}
int a = 16;
cout << setw(5) << a << endl;
  \end{lstlisting}
  \end{tabular}
  \caption{設定輸出寬度}
  \label{string:mani:code:setw:example}
\end{code}

\paragraph{}\lstinline!std::setfill! 可以設定空白處的填充字元,傳入的參數是一個字元,通常和 \lstinline!std::setw! 混用,如程式碼 \ref{string:mani:code:setfill:example}。

\begin{code}[h!]
  \centering
  \begin{tabular}{c}
  \begin{lstlisting}
cout << setw(5) << setfill('x') << 16 << endl;
  \end{lstlisting}
  \end{tabular}
  \caption{設定填充字元}
  \label{string:mani:code:setfill:example}
\end{code}

\paragraph{}可以試試看以下程式碼,在此不贅述。
\begin{itemize}
\item \lstinline!cout << setw(3) << 55688 << endl;!
\item \lstinline!cout << setfill('x') << setw(5) << left << 16 << endl;!
\item \lstinline!cout << setfill('x') << setw(5) << right << 16 << endl;!
\item \lstinline!cout << setfill('x') << setw(5) << internal << -16 << endl;!
\end{itemize}

\paragraph{\lstinline!printf! 格式}\lstinline!printf! 本身就有內建和 \index{標頭檔!iomanip}{\lstinline!<iomanip>!} 相似的功能,比如我們可以設定輸出寬度、印出小數點後 n 位、填充前導零等,如程式碼 \ref{string:mani:code:printf:format}。

\begin{code}
  \centering
  \begin{tabular}{c}
  \begin{lstlisting}
printf("%5d\n", 16); // 設定寬度
printf("%.3f", 3.14159); // 小數點後 3 位
printf("%05d\n", 16); // 前導 0,結果為 00016
  \end{lstlisting}
  \end{tabular}
  \caption{\lstinline!printf! 格式範例}
  \label{string:mani:code:printf:format}
\end{code}

\paragraph{\lstinline!scanf! 格式}學習 \lstinline!scanf! 最有價值的是他可以對輸入的格式做設定,比如要讀入一個時間的格式「\lstinline!hh:mm!」,\lstinline!cin! 一般需要多使用一個變數來讀入 \lstinline!':'!,如程式碼 \ref{string:mani:code:time:format:cin}。

\begin{code}[h!]
  \centering
  \begin{subcode}{.43\textwidth}
    \centering
    \begin{tabular}{c}
    \begin{lstlisting}
int hh, mm;
char ch;
cin >> hh >> ch >> mm;
    \end{lstlisting}
    \end{tabular}
    \caption{用 \lstinline!cin! 輸入}
    \label{string:mani:code:time:format:cin}
  \end{subcode}
  ~
  \begin{subcode}{.33\textwidth}
    \centering
    \begin{tabular}{c}
    \begin{lstlisting}
int hh, mm;
scanf("%d:%d", &hh, &mm);
    \end{lstlisting}
    \end{tabular}
    \caption{用 \lstinline!scanf! 輸入}
    \label{string:mani:code:time:format:scanf}
  \end{subcode}
  \caption{比較 \lstinline!cin! 與 \lstinline!scanf! 差異}
  \label{string:mani:code:time:format}
\end{code}

\paragraph{}\lstinline!scanf! 可以直接設定格式,省去多使用一個變數的麻煩,如程式碼 \ref{string:mani:code:time:format:scanf}。

\paragraph{回傳值}\lstinline!scanf! 也可以使用在 0 尾版、n 行版等迴圈中,不難,在此集中講解 EOF 版。
\paragraph{}當 \lstinline!scanf! 讀到檔尾時,會回傳一個常數 \lstinline!EOF! (數值通常是 -1),因此 EOF 版就會寫為程式碼 \ref{string:mani:code:scanf:eof}。

\begin{code}[h!]
  \centering
  \begin{tabular}{c}
  \begin{lstlisting}
while (scanf("%d", &n) != EOF) {
  // do something ...
}
  \end{lstlisting}
  \end{tabular}
  \caption{EOF 版}
  \label{string:mani:code:scanf:eof}
\end{code}

\subsubsection{標準 I/O}

\subsubsection{檔案 I/O}
\paragraph{\lstinline!fscanf! 和 \lstinline!fprintf!}

\subsubsection{字串 I/O}

\paragraph{\lstinline!sscanf! 和 \lstinline!sprintf!}

\paragraph{\lstinline!stringstream!}\texttt{C++} 中也有提供字串 I/O,稱為 \lstinline!stringstream! 類別,在 \index{標頭檔!sstream}{\lstinline!<sstream>!} 標頭檔裡面,用法與 \lstinline!cin!、\lstinline!cout! 差不多,如程式碼 \ref{string:mani:code:string:stream:usage}。

\begin{code}[h!]
  \centering
  \begin{tabular}{c}
  \begin{lstlisting}
stringstream ss;
string str;
ss << "Hello world!"; // 將東西塞進 stringstream
ss >> str; // 丟到字串
cout << str << endl; // Hello world!
  \end{lstlisting}
  \end{tabular}
  \caption{\lstinline!stringstream! 基本用法}
  \label{string:mani:code:string:stream:usage}
\end{code}

\paragraph{}程式碼 \ref{string:mani:code:string:stream:usage} 會印出 \lstinline!"Hello world!"! 字串,再來看看 \lstinline!stringstream! 怎麼解決 \lstinline!sscanf! 遇到迴圈無法解決的事,如程式碼 \ref{string:mani:code:string:stream:loop}。

\begin{code}[h!]
  \centering
  \begin{tabular}{c}
  \begin{lstlisting}
int a;
stringstream ss;
string str = "1 2 3 4 5";
ss << str;
for (int i = 0; i < 5; i++) {
  ss >> a;
  cout << a << endl;
}
  \end{lstlisting}
  \end{tabular}
  \caption{可以用迴圈來讀取}
  \label{string:mani:code:string:stream:loop}
\end{code}

\begin{code}[h!]
  \centering
  \begin{tabular}{c}
  \begin{lstlisting}
while (ss >> a) {
  // do something
}
  \end{lstlisting}
  \end{tabular}
  \caption{\lstinline!stringstream! 碰到 EOF}
  \label{string:mani:code:string:stream:eof}
\end{code}

\subsection{字串技巧}

\subsubsection{善用 index}

\subsubsection{回文}

\subsubsection{二維問題}

\subsubsection{子字串}

\subsubsection{其他}


\subsection{字串應用}
\subsubsection{羅馬數字}

\paragraph{}對大多數的人而言,這並不是一個陌生的主題。羅馬數字用一些特別的字母來當做某個數字,如表 \ref{string:mani:table:roman:number}。

\begin{table}[h!]
  \centering
  \begin{tabular}{|c||c|c|c|c|c|c|c|}
  \hline
  \textbf{數字} & 1 & 5 & 10 & 50 & 100 & 500 & 1000\\
  \hline
  \textbf{符號} & I & V & X & L & C & D & M\\
  \hline
  \end{tabular}
  \caption{羅馬數字}
  \label{string:mani:table:roman:number}
\end{table}

\paragraph{}羅馬數字系統遵守兩個原則:

\begin{itemize}
\item \textbf{加法原則}:透過累加符號遞增數字。例如:數字 1 寫為「I」、數字 2 寫為「II」、3 表示為「III」,以此類推。
\item \textbf{減法原則}:為簡化書寫,4 不寫作「IIII」,而是用「$5-1$」寫為「IV」。
\end{itemize}

\paragraph{}總結來說,兩種規則的區分如下:如果較小的數寫在較大的數的右邊,則為加法;反之則為減法。例如:11 為 $10+1$,表示為 XI;9 被當作 $10-1$,表示為 IX。
\paragraph{}除此之外,還有兩個小細節:

\begin{itemize}
\item \textbf{統一書寫規則}:羅馬人規定個位數由個位數決定、十位數由十位數決定,以此類推。例如,99 雖然可以視為 $100-1$,但羅馬數字會統一看做 $90+9$,也就是 XC (90) 和 IX (9),寫為 XCIX。
\item 符號不超過\textbf{三個}:4 會寫為 IV 而非 IIII,9 被寫為 IX 而非 XIIII。因此,不管怎麼湊,羅馬數字會在 3999 以內 (在此不討論更大數的表示法)。
\end{itemize}

\paragraph{阿拉伯數字轉羅馬數字}羅馬數字之中,個位數的符號表示個位數、十位數用來表示十位數,這之間不會混用,因此我們用「{\color{red}\textbf{位數}}」來觀察較合適。在此我們可以用表 \ref{string:mani:table:roman:number:regular} 去觀察出個位、十位、百位的規律:

\begin{table}[h!]
  \centering
  \begin{tabular}{|c||c|c|c|c|c|c|c|c|c|}
  \hline
  \textbf{數字} & 1 & 2 & 3 & 4 & 5 & 6 & 7 & 8 & 9\\
  \hline
  \textbf{符號} & I & II & III & IV & V & VI & VII & VIII & IX\\
  \hline
  \hline
  \textbf{數字} & 10 & 20 & 30 & 40 & 50 & 60 & 70 & 80 & 90\\
  \hline
  \textbf{符號} & X & XX & XXX & XL & L & LX & LXX & LXXX & XC\\
  \hline
  \hline
  \textbf{數字} & 100 & 200 & 300 & 400 & 500 & 600 & 700 & 800 & 900\\
  \hline
  \textbf{符號} & C & CC & CCC & CD & D & DC & DCC & DCCC & CM\\
  \hline
  \end{tabular}
  \caption{羅馬數字找規律}
  \label{string:mani:table:roman:number:regular}
\end{table}

\paragraph{}可以得到以下結果:
\begin{itemize}
\item 個位、十位、百位之間符號不同,但「{\color{blue}\textbf{格式}}」相同,也就是只差在 IXC 用 VLD、XCM 來代換。既然格式相同,我們就可以試圖用迴圈簡化其結果。
\item 若只觀察個位數,可發現 1、2、3 的符號恰好是重複符號。除此之外,1、2、3 和 6、7、8 之間只差一個「5」的符號。這代表我們可以用 \lstinline!if! 把「5」的符號判斷掉,扣掉後用 1、2、3 的方式處理。
\item 4 和 9 如果要取巧可能會比較困難,平常練習可以思考如何修改,若思考時間不夠建議直接用特殊判斷處理掉。
\end{itemize}

\paragraph{}根據上面的討論,我們利用迴圈判斷的同時,我們需要利用陣列來儲存我們想要的東西,程式碼 \ref{string:mani:code:number:to:roman:array} 是一個很簡便的實現方法。

\begin{code}[h!]
  \centering
  \begin{tabular}{c}
  \begin{lstlisting}
#define L 3
string Nine[L] = {"CM", "XC", "IX"};
string Four[L] = {"CD", "XL", "IV"};
string Five[L] = {"D", "L", "V"};
string One[L]  = {"C", "X", "I"};
int    Mod[L]  = {100, 10, 1};
int    Mod5[L] = {500, 50, 5};
  \end{lstlisting}
  \end{tabular}
  \caption{儲存符號、餘數}
  \label{string:mani:code:number:to:roman:array}
\end{code}

\paragraph{}程式碼 \ref{string:mani:code:number:to:roman:array} 簡單紀錄每一次判斷的餘數 (\lstinline!Mod! 和 \lstinline!Mod5! 陣列),並且會對應到 \lstinline!Five!、\lstinline!One! 這兩個陣列做處理。此外,\lstinline!Nine!、\lstinline!Four! 陣列是針對 4 和 9 直接例外處理。

\begin{code}[h!]
  \centering
  \begin{tabular}{c}
  \begin{lstlisting}
string roman = "";
for (;n >= 1000 && n; n -= 1000)
  roman += "M"; // 千位數例外判斷
for (int i = 0; i < L; i++) { // 百位、十位、個位數
  if (n / Mod[i] == 9) // 特殊判斷 9
    roman += Nine[i], n -= 9 * Mod[i];
  if (n / Mod[i] == 4) // 特殊判斷 4
    roman += Four[i], n -= 4 * Mod[i];
  while (n >= Mod5[i]) // 判斷 5 以上
    roman += Five[i], n -= Mod5[i];
  while (n >= Mod[i])  // 剩下的部分
    roman += One[i],  n -= Mod[i];
}
  \end{lstlisting}
  \end{tabular}
  \caption{阿拉伯數字轉羅馬數字}
  \label{string:mani:code:number:to:roman}
\end{code}

\paragraph{}程式碼 \ref{string:mani:code:number:to:roman} 是對應的處理方法,可以看到第 1 行針對千位數做判斷,在迴圈中第 5 行、第 7 行也是根據先前的討論,優先處理 4 和 9 的情形 (可以想一下這兩個判斷為什麼不能反過來)。

\paragraph{羅馬數字轉阿拉伯數字}至於如何把羅馬數字換回阿拉伯數字?由剛剛的表格知道,除了 4 和 9 的羅馬數字是兩個字元 (由於\textbf{減法規則}),其他都可以拆成一個字元來看待,因此我們可以藉由{\color{red}\textbf{記錄目前}}掃到的羅馬數字,來判斷下個羅馬數字該使用加法規則還是減法規則。

\subsubsection{字串和數字轉換}

\paragraph{字串轉數字}一開始有提到,把一個數字字元轉換成數字的方法,若現在有一個數字字串,例如 \lstinline!"123"!,要怎麼做呢?

\paragraph{}程式碼 \ref{string:mani:code:char:to:number} 的概念可以繼續延伸,我們可以對每個字元減掉 \lstinline!'0'! 之後再乘上對應的值,最後加總就會轉換成對應數字。

\begin{align*}
123&=1\times{{10}^2}+2\times{{10}^1}+3\times{{10}^0}\\
&=(\text{\lstinline!'1'!}-\text{\lstinline!'0'!})\times{{10}^2}+(\text{\lstinline!'2'!}-\text{\lstinline!'0'!})\times{{10}^1}+(\text{\lstinline!'3'!}-\text{\lstinline!'0'!})\times{{10}^0}
\end{align*}

\paragraph{}實作上我們會從高位的字元往後做,以節省計算 ${10}^n$ 的時間。

\begin{code}[h!]
  \centering
  \begin{tabular}{c}
  \begin{lstlisting}
int MyAtoI(string str)
{
  int res = 0, i;
  for (i = 0; i < str.size(); i++)
  {
    res *= 10;
    res += str[i] - '0';
  }
  return res;
}
  \end{lstlisting}
  \end{tabular}
  \caption{從高位數開始做}
  \label{string:mani:code:my:atoi}
\end{code}

\paragraph{}如程式碼 \ref{string:mani:code:my:atoi},我們把剛剛的算式做轉換,可以得到如下算式:

\begin{align*}
123={({(0\times{10}+1)}\times{10}+2)}\times{10}+3
\end{align*}

\paragraph{}除此之外,也有些人會從個位數開始做,只是這樣就要額外變數來紀錄 $10$ 的次方。

\paragraph{數字轉字串}

\subsubsection{進位變換}

\subsubsection{習題}

\ifx \allfiles \undefined

\printindex

\clearpage
\end{CJK}
\end{document}

\fi
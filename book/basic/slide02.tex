\documentclass[utf8]{beamer}

%% === CJK 套件 ===
\usepackage{CJKutf8,CJKnumb}                 % 中文套件
%% === AMS 標準套件 ===
\usepackage{amsmath,amsfonts,amssymb,amsthm} % 數學符號
\usepackage{ulem}
%% ===  ===
%\usepackage[chapter]{algorithm}              % 演算法套件
%\usepackage[noend]{algpseudocode}            % pseudocode 套件
\usepackage{listings}                        % 程式碼
%% === TikZ 套件 ===
\usepackage{tikz,tkz-graph,tkz-berge}        % 繪圖
\usepackage{multicol}
\usepackage{xkeyval,xargs}
\usepackage{xcolor}

\usetheme{Boadilla}
\usecolortheme{whale}

\setbeamertemplate{items}[circle]

%% === 設定 C++ 格式 ===
\lstset{%
  language=C++,                     % 設定語言
  %% === 空白, tab 相關 ===
  tabsize=2,                            % 設定 tab = 多少空白
  %showspaces=true,                      % 設定是否標示空白
  %showtabs=true,                        % 設定是否標示 tab
  %tab=\rightarrowfill,                  % 設定 tab 樣式
  %% === 行數相關 ===
  %numbers=left,                         % 行數標示位置
  %stepnumber=1,                         % 每隔幾行標示行數
  %numberstyle=\tiny,
  %% === 顏色設定 ===
  basicstyle=\ttfamily,
  keywordstyle=\color{blue}\ttfamily,
  stringstyle=\color{red!50!brown}\ttfamily,
  commentstyle=\color{green!50!black}\ttfamily,
  %identifierstyle=\color{black}\ttfamily,
  emphstyle=\color{purple}\ttfamily,
  extendedchars=false,
  texcl=true,
  moredelim=[l][\color{magenta}]{\#}
}

\begin{document}
\begin{CJK}{UTF8}{bkai}

\title{基礎程式設計技巧(二)\\程式與結構}
\author{許胖}
\institute[PCSH]{板燒高中}

\begin{frame}
  \titlepage
\end{frame}
\begin{frame}
  \frametitle{大綱}
  \begin{multicols}{2}
    \tableofcontents
  \end{multicols}
\end{frame}

\section{簡介}
\begin{frame}
  \frametitle{大綱}
  \begin{multicols}{2}
    \tableofcontents[currentsection]
  \end{multicols}
\end{frame}

\section{溢位現象}
\begin{frame}
  \frametitle{大綱}
  \begin{multicols}{2}
    \tableofcontents[currentsection]
  \end{multicols}
\end{frame}

\begin{frame}[fragile]
  \frametitle{溢位}
  \begin{block}{問題}
  執行這段程式碼之後,會發生什麼現象?
  \pause
  \begin{lstlisting}
  int x = 2147483647;
  cout << x + 1 << endl;
  \end{lstlisting}
  \end{block}
  \begin{exampleblock}{答案}<3->
    \begin{itemize}
    \item \texttt{-2147483648}
    \item<4-> 為什麼?\onslide<5->{\alert{Ans:溢位現象}}
    \end{itemize}
  \end{exampleblock}
\end{frame}

\begin{frame}
  \frametitle{溢位}
  \begin{block}{說明}<1->
    \onslide<2->{
    \begin{table}[h]
      \begin{tabular}{c|c|c|c|c|l}
          & 01111111         & 11111111 & 11111111 & 11111111 & \onslide<3->{\alert{2147483647}}\\
      $+$ & 00000000         & 00000000 & 00000000 & 00000001 & \onslide<3->{\alert{1}}\\
      \hline
      \hline
          & \alert{1}0000000 & 00000000 & 00000000 & 00000000 & \onslide<3->{\alert{-2147483648}}\\
      \end{tabular}
    \end{table}
    }
  \end{block}
  \begin{alertblock}{觀念}<4->
    \begin{itemize}
    \item 只要存資料的記憶體是有限的,那麼
      \begin{itemize}[<5->]
      \item 儲存的資料就有\alert{範圍}上的限制
      \item<6-> 不然,就是\alert{精確度}的限制
      \end{itemize}
    \end{itemize}
  \end{alertblock}
\end{frame}

\begin{frame}[fragile]
  \frametitle{資料型態}
  \begin{table}[h]
    \begin{tabular}{|c|c|c|c|}
    \hline
    資料型態                   & 位元組 & 下界                       & 上界\\
    \hline
    \lstinline{char}{}        & 1     & -128                      & 127\\
    \hline
    \lstinline{short}{}       & 2     & -32768                    & 32767\\
    \hline
    \lstinline{int}{}         & 4     & -2147483648               & 2147483647\\
    \hline
    \lstinline{long long}{}   & 8     & -9223372036854775808      & 9223372036854775807\\
    \hline
    \lstinline{float}{}       & 4     & $-3.40282\times{10^{38}}$ & $3.40282\times{10^{38}}$\\
    \hline
    \lstinline{double}{}      & 8     & $-1.79769\times{10^{308}}$     & $1.79769\times{10^{308}}$\\
    \hline
    \end{tabular}
    \caption{資料型態上下界}
  \end{table}
\end{frame}

\section{程式的執行}
\begin{frame}
  \frametitle{大綱}
  \begin{multicols}{2}
    \tableofcontents[currentsection]
  \end{multicols}
\end{frame}

\subsection{記憶體}

\begin{frame}
  \frametitle{重點回顧}
  \begin{exampleblock}{重點回顧}
    \begin{itemize}[<+->]
    \item \alert{位元} (bit, b):計算機儲存資料的基本單位,只儲存 \alert{0 和 1}
    \item \alert{位元組} (byte, B):因為位元很多,所以我們把 8 個位元「打包起來」,變成一個位元組
      \begin{table}[h]
        \begin{tabular}{|c|}
        \hline
        01001010\\
        \hline
        \end{tabular}
        \caption{位元組}
      \end{table}
    \item 常見應用
      \begin{itemize}
      \item KB、MB、GB、TB、PB:資料大小
      \item Kbps、Mbps、Gbps:資料傳輸速度
      \end{itemize}
    \end{itemize}
  \end{exampleblock}
\end{frame}

\begin{frame}
  \frametitle{記憶體分類}
  \begin{block}{根據用途來分類,可分為 ...}<+->
    \begin{enumerate}[<+->]
    \item 硬碟,速度最慢
      \begin{itemize}[<+->]
      \item 長時間保存資料
      \item 容量大
      \end{itemize}
    \item \alert{主記憶體},俗稱記憶體
      \begin{itemize}[<+->]
      \item \alert{程式執行的地方}
      \item 我們說的記憶體、RAM 多是指這裡
      \item 關機後資料就消失
      \end{itemize}
    \item 快取記憶體
    \item 暫存器
      \begin{itemize}[<+->]
      \item 這兩個與 CPU 有關,可以先不管
      \end{itemize}
    \end{enumerate}
  \end{block}
  \begin{exampleblock}{主軸}<+->
    \begin{itemize}
    \item 我們將集中記憶體和程式之間的關係來討論。
    \end{itemize}
  \end{exampleblock}
\end{frame}

\subsection{位址與指標}

\begin{frame}[fragile]
  \frametitle{位址}
  \begin{block}{記憶體 ...}<+->
    \begin{itemize}[<+->]
    \item 記憶體就像「土地」一樣,可以分配用途
      \begin{itemize}
      \item 例如 \lstinline{int}{}、\lstinline{double}{} 等分配方式
      \end{itemize}
    \item 既然要把土地分配,那麼每一塊地都要有個類似「\alert{地址}」(地籍) 的編號
    \item 每個\alert{位元組}都有一個地址,稱為「\alert{位址}」。
    \end{itemize}
  \end{block}
  \begin{alertblock}{觀念}<+->
    \begin{itemize}
    \item 位元組是計算機中,\alert{擁有位址}的最小單位。
    \end{itemize}
  \end{alertblock}
\end{frame}

\begin{frame}[fragile]
  \frametitle{觀察位址}
  \begin{block}{取址運算子}
  下列程式會印出變數 \lstinline{x}{} 的位址。
  \pause
  \begin{lstlisting}
  #include <iostream>
  using namespace std;
  int main() {
    int x;
    cout << &x << endl;
  }
  \end{lstlisting}
  \end{block}
  \begin{exampleblock}{註}<3->
    \begin{itemize}
    \item 印出來是一個 16 進位的數字,代表變數被分配到的「位址」。
    \item<4-> 分配到的位址會依據機器、作業系統、編譯器不同而不同。
    \item<5-> 總之就是\alert{不一定}啦!(╯°□°)╯︵ ╧╧
    \end{itemize}
  \end{exampleblock}
\end{frame}

\begin{frame}[fragile]
  \frametitle{連續宣告}
  \begin{block}{觀察}
  觀察一下多變數的位址。
  \pause
  \begin{lstlisting}
  int x, y, z;
  cout << &x << endl << &y << endl << &z << endl;
  \end{lstlisting}
  \end{block}
  \begin{exampleblock}{結果}<3->
    \begin{itemize}
    \item 以筆者的機器來說,會出現以下結果:
      \onslide<4->{
      \begin{center}
      \texttt{0040EC64}\onslide<5->{ \alert{$\leftarrow$ 變數 \texttt{x} 的位址}}\\
      \texttt{0040EC68}\onslide<5->{ \alert{$\leftarrow$ 變數 \texttt{y} 的位址}}\\
      \texttt{0040EC6C}\onslide<5->{ \alert{$\leftarrow$ 變數 \texttt{z} 的位址}}
      \end{center}
      }
    \item<6-> 以筆者的機器,\lstinline{int}{} 的大小是 \alert{4} 個位元組。
    \end{itemize}
  \end{exampleblock}
\end{frame}

\begin{frame}
  \frametitle{記憶體分配}
  \begin{block}{說明}<+->
    \begin{itemize}
    \item 觀察這幾個記憶體
      \onslide<+->{
      \begin{table}[h]
        \begin{tabular}{r|c|c|c|c|l}
        \cline{2-5}
        \texttt{0040EC64} & & & & & $\leftarrow$ 變數 \texttt{x} 存放位置\\
        \cline{2-5}
        \texttt{0040EC68} & & & & & $\leftarrow$ 變數 \texttt{y} 存放位置\\
        \cline{2-5}
        \texttt{0040EC6C} & & & & & $\leftarrow$ 變數 \texttt{z} 存放位置\\
        \cline{2-5}
        \end{tabular}
      \end{table}
      }
    \end{itemize}
  \end{block}
  \begin{exampleblock}{註}<+->
    \begin{itemize}
    \item \texttt{x} 佔 4 個位元組 (\texttt{0040EC64}、\texttt{0040EC65}、\texttt{0040EC66}、\texttt{0040EC67})
    \item<+-> \texttt{y} 佔 4 個位元組 (\texttt{0040EC68} 到 \texttt{0040EC6B})
    \item<+-> \texttt{z} 佔 4 個位元組 (\texttt{0040EC6C} 到 \texttt{0040EC6F})
    \end{itemize}
  \end{exampleblock}
\end{frame}

\begin{frame}[fragile]
  \frametitle{大印第安和小印第安}
  \begin{block}{問題}<+->
  假設現在 \lstinline{int x = 0x12345678;}{} 的位址是 \texttt{0040EC64},那麼這個 16 進位的數字「實際上」是怎樣儲存的呢?
  \end{block}
  \begin{exampleblock}{大印第安與小印第安}<+->
    \begin{itemize}
    \item 大印第安存法:大位數存在\alert{低}位址
    \onslide<+->{
      \begin{table}[h]
      \begin{tabular}{|c|c|c|c|}
      \hline
      0x12 & 0x34 & 0x56 & 0x78\\
      \hline
      \end{tabular}
      \end{table}
    }
    \item<+-> 小印第安存法:大位數存在\alert{高}位址
    \end{itemize}
  \end{exampleblock}
\end{frame}

\begin{frame}[fragile]
  \frametitle{指標}
  \begin{block}{概念}
    \begin{itemize}[<+->]
    \item 像是一根「\alert{手指}」,指著一塊記憶體。
    \item<+-> 指著的記憶體,可以指定用途。
    \end{itemize}
  \end{block}
  \begin{alertblock}{宣告}<+->
    \begin{lstlisting}
  int *x;
    \end{lstlisting}
  \end{alertblock}
  \begin{exampleblock}{註}<+->
    \begin{itemize}
    \item \lstinline{x}{} 是一個指標,會指向一塊記憶體。\onslide<+->{謎之音:「怎麼指?」}
      \begin{itemize}[<+->]
      \item 去\alert{儲存}某一塊記憶體的\alert{位址}。
      \end{itemize}
    \item<+-> 那塊記憶體的用途是 \lstinline{int}{}。
    \end{itemize}
  \end{exampleblock}
\end{frame}

\begin{frame}[fragile]
  \frametitle{指標實戰}
  \begin{block}{實戰}
    \begin{lstlisting}
  int x = 5;
  int *ptr = &x;
  cout << x << endl;
  cout << *ptr << endl;
    \end{lstlisting}
  \end{block}
\end{frame}

\section{程式控制}
\begin{frame}
  \frametitle{大綱}
  \begin{multicols}{2}
    \tableofcontents[currentsection]
  \end{multicols}
\end{frame}

\subsection{選擇結構}

\begin{frame}
  \frametitle{選擇結構}
\end{frame}

\subsection{迴圈結構}
\subsection{陣列}

\begin{frame}
  \frametitle{練習題}
  \begin{exampleblock}{\href{http://unfortunate-dog.github.io/articles/100/p10035/}{UVa 10035 - Primary Arithmetic}}<+->
  \label{uva:10035}
  不難。
  \end{exampleblock}
  \begin{exampleblock}{\href{http://unfortunate-dog.github.io/articles/100/p10038/}{UVa 10038 - Jolly Jumpers}}<+->
  \label{uva:10038}
  讀懂題意就不難。
  \end{exampleblock}
  \begin{exampleblock}{\href{http://unfortunate-dog.github.io/articles/1/p109/}{UVa 109 - SCUD Busters}}<+->
  \label{uva:591}
  這一題敘述和公式稍嫌複雜,但是用我們之前所學的工具依然可以輕鬆解決。
  \end{exampleblock}
\end{frame}

\section{函數}
\subsection{傳值呼叫}
\subsection{傳址呼叫}
\subsection{傳參考呼叫}
\subsection{函數多載}

\section{程式技巧}
\subsection{函式化}
\subsection{define與inline}

\section{C++物件導向}
\subsection{物件與類別}
\subsection{建構子與解構子}
\subsection{運算子多載}

\section{暴力搜尋法}

\clearpage
\end{CJK}
\end{document}
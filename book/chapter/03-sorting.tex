\ifx \allfiles \undefined
\documentclass[12pt,a4paper,oneside]{report}

%% === CJK 套件 ===
\usepackage{CJKutf8,CJKnumb}                 % 中文套件
%% === AMS 標準套件 ===
\usepackage{amsmath,amsfonts,amssymb,amsthm} % 數學符號
%% === ===
\usepackage{algorithm}
\usepackage{listings}                        % 程式碼
%% === TikZ 套件 ===
\usepackage{tikz,tkz-graph,tkz-berge}        % 繪圖
\usepackage{multicol}
%% == ==
\usepackage[unicode]{hyperref}
\usepackage{xcolor}
\hypersetup{
    colorlinks,
    linkcolor={blue!100!black},
    citecolor={blue!75!black},
    urlcolor={blue!50!black}
}
%% == 調整設定 ==
\usepackage{enumitem}                           % 修改 enumerate, item
\usepackage{bbding}
\usepackage{titletoc,titlesec,imakeidx}
%% == ==
\usepackage{newfloat}
\usepackage{caption,subcaption}
\usepackage{xkeyval,xargs}
\usepackage{ulem}
\usepackage{import}

%% === 設定 C++ 格式 ===
\lstset{%
  language=C++,             % 設定語言
  %% === 空白, tab 相關 ===
  tabsize=2,                % 設定 tab = 多少空白
  %showspaces=true,          % 設定是否標示空白
  %showtabs=true,            % 設定是否標示 tab
  %tab=\rightarrowfill,      % 設定 tab 樣式
  %% === 行數相關 ===
  numbers=left,             % 行數標示位置
  stepnumber=1,             % 每隔幾行標示行數
  numberstyle=\tiny,
  %breaklines=true,          % 設定斷行
  %% === 顏色設定 ===
  basicstyle=\ttfamily,
  keywordstyle=\color{blue}\ttfamily,
  stringstyle=\color{red!50!brown}\ttfamily,
  commentstyle=\color{green!50!black}\ttfamily,
  %identifierstyle=\color{black}\ttfamily,
  emphstyle=\color{purple}\ttfamily,
  extendedchars=false,
  texcl=true,
  moredelim=[l][\color{magenta}]{\#},
  captionpos=b,
  %% === 其他 ===
  %frame=single
}

% ===============================================
%
%  設定頁面格式
%
% ===============================================
%% === 設定頁面格式 ===
%\hoffset         = 10pt                      % 水平位移,預設為 0pt
\voffset         = -15pt                     % 垂直位移,預設為 0pt
\oddsidemargin   = 0pt                       % 預設為 31pt
%\topmargin       = 20pt                      % 預設為 20pt
%\headheight      = 12pt                      % header 的高度,預設為 12pt
%\headsep         = 25pt                      % header 和 body 的距離,預設為 25pt
\textheight      = 620pt                     % body 內文部分的高度,預設為 592pt
\textwidth       = 450pt                     % body 內文部分的寬度,預設為 390pt
%\marginparsep    = 10pt                      % margin note 和 body 的距離,預設為 10pt
%\marginparwidth  = 35pt                      % margin note 的寬度,預設為 35pt
%\footskip        = 30pt                      % footer 高度 + footer 和 body 的距離,預設為 30pt

%% ==  ==
\DeclareFloatingEnvironment[fileext=frm,placement={!ht},name=Frame]{code}
\captionsetup[code]{labelfont=bf}

\makeindex

\linespread{1.14}

\begin{document}
\begin{CJK}{UTF8}{bkai}

\subimport{../config/}{document-config.tex}
\setcounter{chapter}{2}

\fi

\chapter{排序 (Sorting)}

\paragraph{}這一章中我們會講到排序,在競賽中大多數的排序已被內建函式取代 (例如 C 語言的 \lstinline{qsort}{}、C++ 的 \lstinline{std::sort}{}),但是在多數筆試中,題目通常會要求\textbf{熟悉}每個排序的過程。因此本章介紹最經典的幾個排序法,務必在腦中多模擬演練。

\paragraph{}我們要探討最經典的\textbf{排序問題}:給你 n 個數字,要如何將他\textbf{從小到大}排序好呢?例如,給你 5 個數字 3、7、1、2、5,這個問題最後要得到的答案是 1、2、3、5、7,那麼如何將前面的數字排序好就是本章的重點所在。

\paragraph{}本章附教學影片,教學影片可以在稍微了解各個排序法之後再觀看,更能熟悉理解每個排序,以免看不懂。

\section{比較排序法 (Comparing Sorting Algorithms)}

\paragraph{}排序中,最基礎的即是比較排序法。比較排序法,意思就是裡面的元素兩兩比較,基本上來說,如果 n 個元素兩兩比較,總共要比較 $\displaystyle{{n\times{(n - 1)}}\over{2}}$ 次,時間複雜度會達到 $O(n^2)$,但事實上,有些比較是多餘的,例如:我們已知 $a<b$,$b<c$,這時我們再比較 a 和 c 就是多餘的 (很明顯,因為我們可以從上述兩個關係式找出 $a<c$),因此發展出特殊的排序法來減少無謂的比較,時間複雜度可達到 $O(n\lg{n})$。

\paragraph{}接下來介紹三個基礎的排序法:泡沫排序、選擇排序、插入排序。

\subsection{泡沫排序法 (Bubble Sort)}

\paragraph{}教學影片:\href{https://www.youtube.com/watch?v=lyZQPjUT5B4}{泡沫排序法}

\paragraph{}泡沫排序法正如其名,排序時最大的數字會「\textbf{浮出來}」,大略的過程如下:首先我們先讓最大的數字浮上來 (到最右邊),接著在讓第二大的數字浮上來,以此類推。那要如何讓他「浮上來」呢?答案是「交換」,別忘了我們是比較排序法,因此我們要將數字兩兩比較,如果發現左邊的數字比右邊來得大,那麼就把較大的數字交換到右邊去。

\paragraph{}詳細過程如圖 3.1 所示,第一次我們比較第 0 個和第 1 個數字 (第一行)-- 3 和 7,發現右邊的 7 比較大,所以不需要做交換的動作;接著比較第 1 個和第 2 個數字 (第二行)-- 7 和 1,發現左邊的 7 比右邊的 1 大,因此 7 和 1 交換;第 2 個和第 3 個數字 (第三行)-- 7 和 2,第 2 個數字是 7 是剛剛我們對 7 和 1 比較後交換而來,因此這次 7 大於 2,就將 7 交換到右側;第 3 個和第 4 個數字 (第四行)-- 7 和 5 交換;最後 (第五行),我們發現最右邊的 7 就是最大的數字,剩下前 4 個數字還未進行排序的動作。

\subsection{選擇排序法 (Selection Sort)}
\subsection{插入排序法 (Insertion Sort)}
\subsection{合併排序法 (Merge Sort)}
\subsection{快速排序法 (Quick Sort)}
\section{非比較排序法 (Non-comparing Sorting Algorithms)}
\subsection{計數排序法 (Counting Sort)}
\subsection{基數排序法 (Radix Sort)}
\subsection{桶排序法 (Bucket Sort)}
\subsection{穩定性 (Stability)}
\subsection{綜合比較}

\ifx \allfiles \undefined

\printindex[noun]
\clearpage

\end{CJK}
\end{document}

\fi
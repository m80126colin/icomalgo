\ifx \allfiles \undefined
\documentclass[12pt,a4paper,oneside]{report}

% ======================================================================
%
%  引入套件
%    * CJK 套件 - CJK, CJKnumb
%    * AMS 標準套件 - amsmath, amsfonts, amssymb, amsthm
%    * 章節內容
%      - titletoc: 目錄修改套件
%      - titlesev: 美化章節標題
%      - imakeidx: 索引製作
%    * 演算法相關
%      - algorithm: 演算法環境套件
%      - algpseudicode: pseudocode 套件
%      - listings: 程式碼套件
%    * TikZ 套件
%      - tikz: 本體
%      - tkz-graph: 圖論套件
%      - tkz-berge: 另一個圖論套件
%
% ======================================================================
%% === 基本設定 ===
\usepackage{CJKutf8,CJKnumb}
\usepackage{amsmath,amsfonts,amssymb,amsthm}
\usepackage{enumitem}                           % 修改 enumerate, item
\usepackage{bbding}
\usepackage{titletoc,titlesec,imakeidx}
\usepackage{imakeidx}
\usepackage[unicode]{hyperref,xcolor}
\hypersetup{
    colorlinks,
    linkcolor={blue!100!black},
    citecolor={blue!75!black},
    urlcolor={blue!50!black}
}
\usepackage{import}
%% === 演算法相關 ===
\usepackage[chapter]{algorithm}
\usepackage[noend]{algpseudocode}
\usepackage{listings}
%% === TikZ ===
\usepackage{tikz,tkz-graph,tkz-berge}
\usepackage{ifthen}
%% ===  ===
\usepackage{xkeyval,xargs}
\usepackage{ulem}

%% === itemize,enumerate 設定 ===
%  使用 enumitem 套件
\setlist[itemize]{itemsep=0pt,parsep=0pt}
\setlist[enumerate]{itemsep=0pt,parsep=0pt}

%% === 設定 C++ 格式 ===
%\lstset{
  %language=[11]C++,                     % 設定語言
  %% === 空白, tab 相關 ===
  %tabsize=2,                            % 設定 tab = 多少空白
  %showspaces=true,                      % 設定是否標示空白
  %showtabs=true,                        % 設定是否標示 tab
  %tab=\rightarrowfill,                  % 設定 tab 樣式
  %% === 行數相關 ===
  %numbers=left,                         % 行數標示位置
  %stepnumber=3,                         % 每隔幾行標示行數
  %numberstyle=\tiny,
  %% === 顏色設定 ===
  %basicstyle=\ttfamily,
  %keywordstyle=\color{blue}\ttfamily,
  %stringstyle=\color{red}\ttfamily,
  %commentstyle=\color{green}\ttfamily,
  %morecomment=[l][\color{magenta}]{\#},
  %morekeywords={}
%}

% ======================================================================
%
%  設定頁面格式
%
% ======================================================================
%% === 設定頁面格式 ===
%\hoffset         = 10pt                      % 水平位移,預設為 0pt
\voffset         = -15pt                     % 垂直位移,預設為 0pt
\oddsidemargin   = 0pt                       % 預設為 31pt
%\topmargin       = 20pt                      % 預設為 20pt
%\headheight      = 12pt                      % header 的高度,預設為 12pt
%\headsep         = 25pt                      % header 和 body 的距離,預設為 25pt
\textheight      = 620pt                     % body 內文部分的高度,預設為 592pt
\textwidth       = 450pt                     % body 內文部分的寬度,預設為 390pt
%\marginparsep    = 10pt                      % margin note 和 body 的距離,預設為 10pt
%\marginparwidth  = 35pt                      % margin note 的寬度,預設為 35pt
%\footskip        = 30pt                      % footer 高度 + footer 和 body 的距離,預設為 30pt

% ======================================================================
%
%  其他
%
% ======================================================================
\linespread{1.24}
\makeindex[name=noun]        % 索引生成

\begin{document}
\begin{CJK}{UTF8}{bkai}

\subimport{../config/}{document-config.tex}
\setcounter{chapter}{7}

\fi

\newcounter{ind}

\tikzset{plain/.style={
	draw=none,
	fill=none
}}
\tikzset{array block/.style={
	draw,
	rectangle,
	fill=black!10,
	minimum height=\sz,
	minimum width=\sz
}}
\tikzset{selected array block/.style={
	draw,
	rectangle,
	fill=black!30,
	minimum height=\sz,
	minimum width=\sz
}}

\chapter{樹形結構}\label{tree:chap}

\section{樹 (Tree)}

\subsection{緒論 (Introduction)}

\subsection{樹的表示法 (Representation of Tree)}

\subsection{樹的性質 (Properties of Tree)}

\subsection{樹的遍歷 (Traversal of Tree)}
\subsection{二元搜尋樹 (Binary Search Tree)}
\subsection{一般樹轉二元樹}

\section{基礎樹形結構}

\subsection{并查集 (Disjoint Sets, Union-find Sets)}

\paragraph{}現在,如果有很多集合,而且兩兩集合間沒有交集,那麼這些集合被稱為互斥集 (Disjoint Sets)。簡單的例子即 A={1,2,3} 和 B={4,5,6} 是互斥集。生活上也有許多例子都有互斥集的性質,例如:社團分組,每個組的組員都只會待在一個組別 (正常情況);把一百本書打包成三箱,每本書都是完整的;......等等。因此遇到類似問題時,我們會使用互斥集的概念來解決此類問題。

\paragraph{}集合通常作以下運算:交集、聯集、差集三種運算。

\paragraph{}交集運算 (Intersection) 指的就是多個集合中,取所有集合中共有的元素,形成一個新的集合。運算上,我們用AB表示取集合 A 和 B 的交集。例如兩個集合 A={1,2,3}、B={2,3,4} 的交集就是 {2,3};三集合 X={1,2}、Y={2,3}、Z={1,3},雖然這三個集合中,X 和 Y 的交集為 {2}、Y 和 Z 交集為 {3}、X 和 Z 交集為 {1},但沒有一個元素是這三個集合所共有的,因此 X、Y、Z 的交集是一個集合但是沒有元素,這種集合我們稱為空集合 (Empty Set),記為 ,直得一提的是,雖然三個的交集是空集合,但因為兩兩間不為空集合,因此 X、Y、Z 並非互斥集。圖 8.21 用文氏圖來表現交集運算。

\paragraph{}聯集運算 (Union) 在多個集合中,取出所有在任意集合中出現過的元素,形成一個新的集合。計算上用AB表示取集合 A 和 B 的聯集。例如兩個集合 A={1,2,3}、B={2,3,4} 的聯集為 {1,2,3,4} ,元素 1 曾在第一個集合出現過, 4 曾在第二個集合出現過,而 2 和 3 曾在兩個集合都出現過。三集合 X={1,2}、Y={2,3}、Z={1,3},其聯集為 {1,2,3} 。圖 8.22 表示聯集運算。

\paragraph{}差集運算 (Difference) 是在兩個集合中,取出一個集合擁有、但另一個集合沒有的元素,形成一個新的集合。運算上,我們以A-B表示取 A 和 B 的差集。例如,有兩個集合 A={1,2,3}、B={2,3,4},如果我們求 A - B 的話,我們會得到 {1},因為 1 出現在 A 但是沒有出現在 B, 2 和 3 不僅出現在 A 而且出現在 B;同樣地,如果我們求 B - A 的話會求得 {4},因此 A - B 和 B - A 是不同的,這代表差集運算是不可交換的!下圖表示差集運算。

\paragraph{}差集運算可以用交集和聯集來表示,現在一個差集A-B,如果用交集來表示,我們可以表示成A-AB;如果我們用聯集來表示,則為AB-B。

\paragraph{}回來看互斥集,如果我們對互斥集作運算,我們會發現有趣的事情:因為互斥集最大的特性是:任意兩個集合沒有共同交集,因此做交集運算必為空集合;如果我們做聯集運算,會發現新形成的集合恰好是兩個集合的所有元素直接合併;如果做差集運算,會發現兩個集合的差集即是其中一個集合,例如兩個集合 A={1,2} 和 B={3,4} 中,A - B 會得到 A,B - A 會得到 B,因為從先前我們得到的結論:差集運算可以用交集來表示,不難發現AB為空集合。

\paragraph{}綜合以上結論,互斥集作交集運算和差集運算沒有多大意義,因此我們會把注意力放在聯集運算上。另外,在這類型的題目中,往往會詢問任意兩個元素是否在同一集合中,因此我們建構一種資料結構叫做并查集 (Union-Find Set),其重點在於如何合併 (Union) 和查詢 (Find)。

\paragraph{}并查集主要應用在一類問題:有 n 個人要分組,但是我們不知道彼此確切的組別,只知道某兩個人是同一組、或是不同組之類的訊息,而現在我們要問三種問題:

第一、任意兩個人 a、b 是否同組;
第二、某一人 c 所在的組別有多少人;
第三、最多總共有多少組別。

\paragraph{}最直觀的想法,就是把每個人都實際分組,人和組別都以編號表示,如表 8.24,我們看到第 9 人被分在第 1 團,第 1、3、4、8 人被分在第 2 團,第 2、5 人被分在第 3 團,...... 以此類推。


\subsection{Disjoint-set Forest}
\subsection{堆積 (Heap)}
\subsubsection{二元堆積 (Binary Heap)}
\subsubsection{堆積排序法 (Heap Sort)}

\section{平衡二元樹 (Balancing Binary Tree)}

\paragraph{}\index[noun]{平衡樹}{\textbf{平衡樹} (Balacing Tree)} 是一種的樹,他所有子樹的高度相差\textbf{不會超過} 1,在此我們會特別討論\textbf{平衡二元樹},又稱為\textbf{\index[noun]{自平衡二元樹}{自平衡二元樹}} (Self-Balancing Binary Tree),也就是高度相差不超過 1 的二元樹,目的是為了避免極端測資。以二元搜尋樹作為例子,若我們插入的數字恰好是遞增或遞減排序時,因為二員搜尋樹插入的特性,會使得整棵樹退化成\textbf{鏈狀},此時查詢操作就會由原本的 $O(\lg{n})$ 退化為 $O(n)$,下圖是依序插入 1、2、3、4、5 的二元搜尋樹。

\paragraph{}常見的平衡樹有:\index[noun]{線段樹}{\textbf{線段樹}}、\index[noun]{AVL 樹}{AVL 樹}、\index[noun]{2-3-4 樹}{2-3-4 樹}、\index[noun]{B 樹}{B 樹}、\index[noun]{紅黑樹}{紅黑樹}、\index[noun]{伸展樹}{\textbf{伸展樹}}、\index[noun]{樹堆}{\textbf{樹堆}}、\index[noun]{節點大小平衡樹}{\textbf{節點大小平衡樹}}等等。其中以線段樹、伸展樹、樹堆、節點大小平衡樹等最為常用在競賽中,因為他們在編寫時\textbf{難度較低},又有不錯的\textbf{應用彈性},常常成為比賽時強大的利器。

\subsection{淺談區間查詢問題 (Range Query Problem)}

\paragraph{}\index[noun]{區間查詢問題}{\textbf{區間查詢問題} (Range Query Problem)} 是一類問題,通常有兩個特性:
\begin{itemize}
\item 查詢內容會在\textbf{特定的區間}內執行
\item 會有\textbf{多次查詢}
\end{itemize}
\paragraph{}基於以上兩點特性,解決這類型的演算法必須要能夠針對大量的查詢去做優化。而區間查找問題需要搜尋的資料形式有很多種,我們會針對以下兩種作探討:
\begin{itemize}
\item \index[noun]{區間求和查詢}{\textbf{區間求和查詢} (Range Sum Query)}:此類型問題會詢問一個區間內數值總和
\item \index[noun]{區間極值查詢}{\textbf{區間極值查詢} (Range Minimum Query)}:又稱為 \textbf{RMQ 問題},會詢問一段區間內的最大值或最小值,在本章第 \ref{tree:sec:rmq} 節會有更深入的探討
\end{itemize}

\subsubsection{靜態查詢}

\paragraph{}以一維的區間求和查詢為例:給你 $n$ 個整數 $a_1,a_2,\cdots{},a_n$,以及 $q$ 筆詢問,每次詢問都是求介於 $a_i$ 到 $a_j$ 所有數字的和。

\paragraph{}最直觀的做法就是每次直接對 $a_i$ 到 $a_j$ 求和,假設現在有 9 個數字 $(1,5,4,8,7,2,6,4,3)$,我們要查詢 $a_3$ 到 $a_7$ 的數字和,我們知道用直觀法將它求和可以得到 $4+8+7+2+6=27$,如圖 \ref{tree:fig:rsq:naive},總查詢時間複雜度為 $O(qn)$。

\begin{figure}[h]
\centering
\begin{tikzpicture}
% array element definition
\def\arr{1,5,4,8,7,2,6,4,3};
%\def\x{0}; % x pos of arr
%\def\y{0}; % y pos of arr
% size of each node
\def\sz{10mm}
% 繪製 array
%\node[plain] { A };
\foreach \item [count=\i] in \arr
{
	\ifthenelse{2<\i \AND \i<8}{"\node[selected array block] at (\i*\sz,0) {\item};"}{"\node[array block] at (\i*\sz,0) {\item};"}
	\node[plain] at (\i*\sz,0.8*\sz) {$a_\i$};
}
\end{tikzpicture}
\caption{區間求和查詢--直觀法}
\label{tree:fig:rsq:naive}
\end{figure}

\paragraph{}然而這個時間複雜度是\textbf{不夠理想}的,我們在第 \ref{dp:chap} 章動態規劃時有學過一個較快的作法:利用 DP 我們先求出前綴和 $b_k=\displaystyle\sum_{i=1}^{k}{a_i}$ 的和,當我們要求 $a_i$ 到 $a_j$ 時,就直接計算 $b_j-b_{i-1}=\displaystyle\sum_{k=i}^{j}{a_k}$,如此做法,我們需要 $O(n)$ 預處理的時間,每次查詢複雜度為 $O(1)$,總查詢的時間複雜度為 $O(q)$,如圖 \ref{tree:fig:rsq:dp}。

\begin{figure}[h]
\centering
\begin{tikzpicture}
% array element definition
\def\arr{1,5,4,8,7,2,6,4,3};
%\def\x{0}; % x pos of arr
%\def\y{0}; % y pos of arr
% size of each node
\def\sz{10mm}
% 畫出 array
%\node[plain] { A };
\foreach \item [count=\i] in \arr
{
	\ifthenelse{2<\i \AND \i<8}{"\node[selected array block] at (\i*\sz,0) {\item};"}{"\node[array block] at (\i*\sz,0) {\item};"}
	\node[plain] at (\i*\sz,0.8*\sz) { $a_\i$ };
}
\edef\myres{0}
\setcounter{ind}{0};
\foreach \item [count=\i] in \arr
{
	\pgfmathparse{int(\myres+\item)}
	\xdef\myres{\pgfmathresult}
	\ifthenelse{\i=2 \OR \i=7}{"\node[selected array block] at (\i*\sz,0-2*\sz) {\myres};"}{"\node[array block] at (\i*\sz,0-2*\sz) {\myres};"}
	\node[plain] at (\i*\sz,0.8*\sz-2*\sz) { $b_\i$ };
}
\end{tikzpicture}
\caption{區間求和查詢--動態規劃}
\label{tree:fig:rsq:dp}
\end{figure}

\subsubsection{動態查詢}

\paragraph{}上述的問題我們稱為「靜態」的區間查詢問題,若我們在詢問中多了一些操作,使得我們可以去隨時隨地\textbf{修改}某些數字的值,此時會發生什麼事呢?

\paragraph{}我們發現上面的直觀法時間複雜度依然不變,總查詢的複雜度依然是 $O(qn)$,但是動態規劃的精隨在於前綴和,若是以上述的例子而言,如果我們在詢問中修改了 $a_1$ 的值,那麼勢必會影響到 $b_1$ 到 $b_9$,這將使得 DP 方法的總時間會退化成 $O(qn)$。

\begin{figure}[h]
\centering
\begin{tikzpicture}
% array element definition
\def\arr{1,5,4,8,7,2,6,4,3};
%\def\x{0}; % x pos of arr
%\def\y{0}; % y pos of arr
% size of each node
\def\sz{10mm}
%\node[plain] { A };
\foreach \item [count=\i] in \arr
{
	\ifthenelse{\i=3}{"\node[selected array block] at (\i*\sz,0) {\item};"}{"\node[array block] at (\i*\sz,0) {\item};"}
	\node[plain] at (\i*\sz,0.8*\sz) {$a_\i$};
}
\edef\myres{0}
\foreach \item [count=\i] in \arr
{
	\pgfmathparse{int(\myres+\item)}
	\xdef\myres{\pgfmathresult}
	\ifthenelse{\i>2}{"\node[selected array block] at (\i*\sz,0-2*\sz) {\myres};"}{"\node[array block] at (\i*\sz,0-2*\sz) {\myres};"}
	\node[plain] at (\i*\sz,0.8*\sz-2*\sz) {$b_\i$};
}
\end{tikzpicture}
\caption{修改 $a_3$ 會影響到的範圍}
\label{tree:fig:rsq:dynamic}
\end{figure}

\paragraph{}我們將建表、修改、查詢拆開來看,我們可以知道在動態區間查詢問題中,這兩種解題策略的時間複雜度:

\begin{table}[h]
\centering
\begin{tabular}{c|ccc}
& 建表 & 單次修改 & 單次查詢\\
\hline
\hline
直觀法 & (不須建表) & $O(1)$ & $O(n)$\\
動態規劃 & $O(n)$ & $O(n)$ & O(1)
\end{tabular}
\caption{直觀法、動態規劃的時間複雜度}
\label{tree:tab:rsq:time}
\end{table}

\paragraph{}從表 \ref{tree:tab:rsq:time} 可以看出,儘管兩種演算法耗費時間的部分不同,但是在動態查詢問題中,兩者均要消耗 $O(qn)$ 的時間,於是我們想知道,能不能有一套新的策略,使得查詢的速度不比 DP 快,修改的速度也不比直觀法快,但同時整體的時間可以降到可接受的範圍呢?

\subsubsection{分治策略}

\paragraph{}其中一個思路就是沿用 DP 的想法,但是我們不需要一口氣建出一整條陣列的前綴和,我們採用常常用的「\index{分治法}{\textbf{分治}}」思想:我們把陣列拆成兩半,個別建出前綴和,這樣一來我們可以減少被修改值的數量,同時也可以兼顧快速查詢的優點,如圖 \ref{tree:fig:rsq:dnc}。

\begin{figure}[h]
\centering
\begin{tikzpicture}
% array element definition
\def\arr{1,5,4,8,7,2,6,4,3};
%\def\x{0}; % x pos of arr
%\def\y{0}; % y pos of arr
% size of each node
\def\sz{10mm}
%\node[plain] { A };
\foreach \item [count=\i] in \arr
{
	\ifthenelse{\i=3}{"\node[selected array block] at (\i*\sz,0) {\item};"}{"\node[array block] at (\i*\sz,0) {\item};"}
	\node[plain] at (\i*\sz,0.8*\sz) {$a_\i$};
}
\edef\myres{0}
\foreach \item [count=\i] in \arr
{
	\ifthenelse{\i<5}{%
		\pgfmathparse{int(\myres+\item)}
		\xdef\myres{\pgfmathresult}
		\ifthenelse{\i>2}{"\node[selected array block] at (\i*\sz-0.3*\sz,-2*\sz) {\myres};"}{"\node[array block] at (\i*\sz-0.3*\sz,-2*\sz) {\myres};"}
		\node[plain] at (\i*\sz-0.3*\sz,0.8*\sz-2*\sz) {$b_\i$};
	}{}
}
\edef\myres{0}
\foreach \item [count=\i] in \arr
{
	\ifthenelse{\i>4}{%
		\pgfmathparse{int(\myres+\item)}
		\xdef\myres{\pgfmathresult}
		\node[array block] at (\i*\sz+0.3*\sz,0-2*\sz) {\myres};
		\node[plain] at (\i*\sz+0.3*\sz,0.8*\sz-2*\sz) {$b_\i$};
	}{}
}
\end{tikzpicture}
\caption{二分後,修改 $a_3$ 會影響到的範圍}
\label{tree:fig:rsq:dnc}
\end{figure}

\paragraph{}但是這樣還不夠,儘管我們減少被修改的數量,但是在最糟的情形下,每次修改的時間複雜度依然是 $O(n)$ ($\displaystyle{n\over{2}}$,比剛才少了一半),如果我們繼續分割 \texttt{b} 陣列,直到它只剩下最多兩個數字的前綴和,我們發現每個部份的第一個元素 $b_i$ 都會恰好是 $a_i$ 的值,如圖 \ref{tree:fig:rsq:dnc3} 中 $b_1,b_3,b_5,b_7,b_9$。

\begin{figure}[h]
\centering
\begin{tikzpicture}
% array element definition
\def\arr{1,5,4,8,7,2,6,4,3};
%\def\x{0}; % x pos of arr
%\def\y{0}; % y pos of arr
% size of each node
\def\sz{10mm}
% a 陣列
\foreach \item [count=\i] in \arr
{
	\node[array block] at (\i*\sz,0) {\item};
	\node[plain] at (\i*\sz,0.8*\sz) {$a_\i$};
}
% b 陣列
\edef\myres{0}
\foreach \item [count=\i] in \arr
{
	\pgfmathparse{Mod(\i,2)==0?0:1}
	\xdef\modi{\pgfmathresult}
	\ifnum\modi>0
		\edef\myres{0}
	\fi
	\pgfmathparse{int(\myres+\item)}
	\xdef\myres{\pgfmathresult}
	
	\pgfmathparse{int((\i+1)/2)}
	\pgfmathparse{\i-0.9+0.3*\pgfmathresult}
	\ifnum\modi>0
		\node[selected array block] at (\pgfmathresult*\sz,-2*\sz) {\myres};
	\else
		\node[array block] at (\pgfmathresult*\sz,-2*\sz) {\myres};
	\fi
	\node[plain] at (\pgfmathresult*\sz,0.8*\sz-2*\sz) {$b_\i$};
}
\end{tikzpicture}
\caption{持續切割小問題,直到不能再切割,會發現有許多重複資料點}
\label{tree:fig:rsq:dnc3}
\end{figure}

\paragraph{}此時不難發現前綴和在過度分治下,會留下許多重複的資訊,但是這也是一個很好的轉折點,如果我們把這些重複的資訊拿掉,此時分治的方向就變成\textbf{陣列的長度},我們引入分治的精隨--\index{遞迴}{\textbf{遞迴}},每一次遞迴就是求兩數的和,而每一次遞迴後新陣列的長度是原陣列的\textbf{一半},如此反覆下去,會得到一種嶄新的樹形資料結構--\textbf{線段樹}。



\subsection{線段樹與樹狀數組}

\subsubsection{線段樹 (Range Query Tree)}

\paragraph{}這時我們從「分塊」循序搜尋區間,第一塊的區間是 a1 ~ a3,和為 10;第二塊的區間是 a4 ~ a6,和為 17;第三塊的區間是 a7 ~ a9,和為 13。

\paragraph{}我們發現:第一塊的區間有涵蓋到 a3 ~ a7,但是只有「部分」涵蓋到,因此我們就直接把對 a3 取值;接著第二塊的區間完全在 a3 ~ a7 內,這時我們就不必往下取 a4、a5、a6 的值,直接將原本已經求和的 (a4 ~ a6) 取出來;第三塊如同第一塊的方法,將 a7 取值。最後將取到的值加起來 4 + 17 + 6 = 27。

\paragraph{}這類分塊的表格我們稱為「塊狀表」,它類似直觀法和動態規劃之間的折衷方案:我們一方面保留暴搜的優勢,也就是每次擁有快速改值的特性 (在暴搜中是 O(1) 改值);另一方面,我們也希望擁有動態規劃中,快速求和的性質 (在 DP 中是 O(1) 求和)。於是折衷就形成了「塊狀表」的構造。

\paragraph{}塊狀表的時間複雜度為 O(n) 預處理,每次查詢的時間複雜度為 O(sqrt(n)),總查詢時間複雜度為 O(q * sqrt(n)),較直觀法和動態規劃相比都還要平均。

\subsubsection{樹狀數組 (Binary Indexed Tree, BIT)}

\paragraph{}樹狀數組,又稱二元索引樹 (Binary Indexed Tree)、Fenwick 樹 (Fenwick Tree),可以看作是變形的線段樹

\subsubsection{zkw 線段樹}


\subsection{伸展樹 (Splay Tree)}

\subsection{節點大小平衡樹 (Size Balanced Tree)}

\subsection{樹堆 (Treap, Tree + Heap)}

\subsection{高維樹}

\section{LCA 問題 (Least Common Ancestor)}
\subsection{樸素算法}
\subsection{線上 LCA 算法}
\subsection{離線 LCA 算法--Tarjan 演算法}

\section{RMQ 問題 (Range Maximum/Minimum Query)}\label{tree:sec:rmq}

\subsection{樸素與遞推算法}
\subsection{線段樹與樹狀數組}
\subsection{ST 算法 (Sparse Table)}
\subsection{±1 RMQ}
\subsection{LCA 轉化為 RMQ}
\subsection{一般性 LCA 及 RMQ 問題}
\subsection{笛卡兒樹 (Cartesian Tree)}

\ifx \allfiles \undefined
\printindex[noun]

\clearpage
\end{CJK}
\end{document}
\fi

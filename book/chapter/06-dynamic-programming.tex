\ifx \allfiles \undefined
\documentclass[12pt,a4paper,oneside]{report}

% ======================================================================
%
%  引入套件
%    * CJK 套件 - CJK, CJKnumb
%    * AMS 標準套件 - amsmath, amsfonts, amssymb, amsthm
%    * 章節內容
%      - titletoc: 目錄修改套件
%      - titlesev: 美化章節標題
%      - imakeidx: 索引製作
%    * 演算法相關
%      - algorithm: 演算法環境套件
%      - algpseudicode: pseudocode 套件
%      - listings: 程式碼套件
%    * TikZ 套件
%      - tikz: 本體
%      - tkz-graph: 圖論套件
%      - tkz-berge: 另一個圖論套件
%
% ======================================================================
%% === 基本設定 ===
\usepackage{CJKutf8,CJKnumb}
\usepackage{amsmath,amsfonts,amssymb,amsthm}
\usepackage{enumitem}                           % 修改 enumerate, item
\usepackage{bbding}
\usepackage{titletoc,titlesec,imakeidx}
\usepackage{imakeidx}
\usepackage[unicode]{hyperref,xcolor}
\hypersetup{
    colorlinks,
    linkcolor={blue!100!black},
    citecolor={blue!75!black},
    urlcolor={blue!50!black}
}
\usepackage{import}
%% === 演算法相關 ===
\usepackage[chapter]{algorithm}
\usepackage[noend]{algpseudocode}
\usepackage{listings}
%% === TikZ ===
\usepackage{tikz,tkz-graph,tkz-berge}
\usepackage{ifthen}
%% ===  ===
\usepackage{xkeyval,xargs}
\usepackage{ulem}

%% === itemize,enumerate 設定 ===
%  使用 enumitem 套件
\setlist[itemize]{itemsep=0pt,parsep=0pt}
\setlist[enumerate]{itemsep=0pt,parsep=0pt}

%% === 設定 C++ 格式 ===
%\lstset{
  %language=[11]C++,                     % 設定語言
  %% === 空白, tab 相關 ===
  %tabsize=2,                            % 設定 tab = 多少空白
  %showspaces=true,                      % 設定是否標示空白
  %showtabs=true,                        % 設定是否標示 tab
  %tab=\rightarrowfill,                  % 設定 tab 樣式
  %% === 行數相關 ===
  %numbers=left,                         % 行數標示位置
  %stepnumber=3,                         % 每隔幾行標示行數
  %numberstyle=\tiny,
  %% === 顏色設定 ===
  %basicstyle=\ttfamily,
  %keywordstyle=\color{blue}\ttfamily,
  %stringstyle=\color{red}\ttfamily,
  %commentstyle=\color{green}\ttfamily,
  %morecomment=[l][\color{magenta}]{\#},
  %morekeywords={}
%}

% ======================================================================
%
%  設定頁面格式
%
% ======================================================================
%% === 設定頁面格式 ===
%\hoffset         = 10pt                      % 水平位移,預設為 0pt
\voffset         = -15pt                     % 垂直位移,預設為 0pt
\oddsidemargin   = 0pt                       % 預設為 31pt
%\topmargin       = 20pt                      % 預設為 20pt
%\headheight      = 12pt                      % header 的高度,預設為 12pt
%\headsep         = 25pt                      % header 和 body 的距離,預設為 25pt
\textheight      = 620pt                     % body 內文部分的高度,預設為 592pt
\textwidth       = 450pt                     % body 內文部分的寬度,預設為 390pt
%\marginparsep    = 10pt                      % margin note 和 body 的距離,預設為 10pt
%\marginparwidth  = 35pt                      % margin note 的寬度,預設為 35pt
%\footskip        = 30pt                      % footer 高度 + footer 和 body 的距離,預設為 30pt

% ======================================================================
%
%  其他
%
% ======================================================================
\linespread{1.24}
\makeindex[name=noun]        % 索引生成

\begin{document}
\begin{CJK}{UTF8}{bkai}

\subimport{../config/}{document-config.tex}
\setcounter{chapter}{5}

\fi

\chapter{動態規劃 (Dynamic Programming, DP)}

\section{概論 (Introduction)}

\paragraph{}動態規劃 (常稱 DP) 是一種解題常用的手段,它的想法與遞迴與分治法相近,將大問題分割成小問題來解決,但是在切割小問題的過程中,常常會遭遇到重複切割小問題的情形,使得程式執行速度變得緩慢而沒效率 (例:費氏數列),因此使用一種技巧,將做過的小問題的答案儲存起來,以便下次再使用這個小問題時,能夠直接回傳答案而不浪費執行時間 (記憶化求解),這種技巧我們稱為動態規劃。
\paragraph{}以下將會介紹更多動態規劃的方法以及幾個經典的 DP 問題,至於能夠以 DP 解決的題目眾多,族繁不及被宰,更多的問題請自行在 Online Judge 中尋找並思考,這種題型是高中乃至大學競賽最愛考的題型,幾乎每場比賽都會出的必考題,請各位讀者好好把握。

\subsection{特性 (Property)}

\paragraph{}使用動態規劃的時機,取決於有沒有以下特性:
\begin{itemize}
\item 重疊子問題 (Overlapping Subproblems)
\item 最佳子結構 (Optimal Substructure)
\item 無後效性
\end{itemize}

\subsection{名詞定義 (Terminology)}

\begin{itemize}
\item 狀態
\item 決策
\item 狀態轉移方程 (State Transferring Formula)
  \begin{itemize}
  \item 向前遞推法
  \item 向後遞推法
  \end{itemize}
\end{itemize}

\subsection{動態規劃標記 (Dynamic Programming Notation)}

\paragraph{}目前最方便分析 DP 的方法即是 xD/yD 標記法,x 和 y 皆為非負整數,兩者均表示多少維度的概念,例如:1D 代表一個維度、2D 代表兩個維度等等。
\paragraph{}xD 代表 DP 時狀態的維度,簡單來說就像是你 DP 時的陣列開了幾維,而 yD 代表每次狀態可能的轉移維度,通常都是這些轉移的可能中取極值或者求和。
\paragraph{}例如費氏數列的狀態數就是一個一維陣列,因此費氏數列的 $x=1$;而費氏數列的狀態轉移方程中 $F[i]=F[i-1]+F[i-2]$,決定 $F[i]$ 的轉移數只有一種 $O(1)$,因此維度為 $y=0$。總和而言,費氏數列為 1D/0D 的 DP。
\paragraph{}又例如 LIS (之後會介紹) 的狀態是一維陣列,$x=1$;轉移方程式為
\begin{align*}
LIS[i] = \max_{j<i}{\{LIS[j]+1\}}\text{ \textbf{if} }a[j]<a[i]
\end{align*}
,它是取第 $i$ 個元素前,所有符合條件的最小值,此時轉移數有 $O(n)$,因此維度為 $y=1$。總和而言,LIS 為 1D/1D 的 DP。
\paragraph{}這樣子的表示法使我們可以容易算出時間複雜度以及空間複雜度。根據剛剛的推論,時間複雜度為 $O(n^{x+y})$,空間複雜度為 $O(n^{x})$,不難。

\subsection{與遞迴的關係}

\section{線型動態規劃 (Linear DP)}

\subsection{費氏數列 (Fibonacci Numbers)}

\paragraph{}目前大家最熟知的動態規劃,即是費氏數列。以下是一道例題:
\paragraph{}有些讀者應該可以不難發現其中的關係,那麼沒發現的讀者也不要灰心,讓我們用上面的思路來解決此道問題。
\paragraph{}第一、決策,首先我們要找出爬樓梯所能做的決策,找出決策有兩派做法,這兩派就是我們上面所述的向前遞推法和向後遞推法。
\paragraph{}如果我們從向前遞推法的觀點來看,當你在第 i 階的時候,你可以做兩件事:

\begin{itemize}
\item 走到第 $i+1$ 階 (一次爬一階)
\item 走到第 $i+2$ 階 (一次爬兩階)
\end{itemize}

\paragraph{}找到決策之後,等於是我們也找到了狀態轉移方程,而向前遞推法,必須從已知的狀態類推到未知的狀態,因此我們還要找到第一個已知的狀態 (也就是遞迴中的中止條件)。
\paragraph{}第一個初始條件我們通常找你所設「狀態」的極值,此時的狀態該設為什麼呢?其實,你在不知不覺間已經設定好了狀態--你在第幾階就是你的狀態,因為他狀態只有一個,所以你所需要記錄的陣列只要開一維,也代表說動態規劃的方向是一維的。

\paragraph{}此時階數的範圍介於 $0<n<100$ 之間,我們可以找 $n=1$ 或是 $n=99$ 階的情況,明顯地我們找 $n=1$ 階可以很容易知道方法數只有 1 種,於是我們就找到第一個已知狀態。於是我們可以產生一段虛擬碼如下:

\begin{algorithm}
  \centering
  \begin{algorithmic}[1]
  \Procedure{$DP$}{}
    \State $stair[1]\gets{1}$
    \For{$i=1$ to $n-2$}
      \State $stair[i+1]\gets{stair[i+1]+stair[i]}$
      \State $stair[i+2]\gets{stair[i+2]+stair[i]}$
    \EndFor
    \State $stair[n]\gets{stair[n]+stair[n-1]}$
  \EndProcedure
  \end{algorithmic}
\end{algorithm}

\subsection{最長遞增子序列 (Longest Increasing Subsequence, LIS)}
\subsection{最長共同子序列 (Longest Common Subsequence, LCS)}
\subsection{LCS、LIS 相互轉換}

\section{背包問題 (Knapsack Problems)}

\subsection{01 背包問題 (01-knapsack)}
\subsection{子集合求和問題 (Subset Sum Problem)}
\subsection{完全背包問題--找零問題 (Coin Change)}
\subsection{多重背包問題 (Bounded Knapsack Problem)}
\subsection{分組背包問題}

\section{其他動態規劃}

\subsection{區間動態規劃}
\subsection{環形動態規劃}
\subsection{集合動態規劃 (狀態壓縮)}
\subsection{最大值最小化問題 (Min-max Problem)}
\subsection{區間最大和問題 (maximum consecutive subsequence)}
\subsection{最大矩形面積 (Largest Rectangle Area)}

\section{動態規劃優化 (Dynamic Programming Optimizations)}

\subsection{預處理}
\subsection{改變狀態}
\subsection{滾動數組}
\subsection{決策單調性}
\subsection{單調佇列優化}
\subsection{四邊形不等式優化}
\subsection{斜率優化}

\section{樹型動態規劃}

\subsection{樹形依賴背包}

\ifx \allfiles \undefined

\printindex[noun]
\clearpage

\end{CJK}
\end{document}

\fi
\ifx \allfiles \undefined
\documentclass[12pt,a4paper,oneside]{report}

%% === CJK 套件 ===
\usepackage{CJKutf8,CJKnumb}                 % 中文套件
%% === AMS 標準套件 ===
\usepackage{amsmath,amsfonts,amssymb,amsthm} % 數學符號
%% === ===
\usepackage{algorithm}
\usepackage{listings}                        % 程式碼
%% === TikZ 套件 ===
\usepackage{tikz,tkz-graph,tkz-berge}        % 繪圖
\usepackage{multicol}
%% == ==
\usepackage[unicode]{hyperref}
\usepackage{xcolor}
\hypersetup{
    colorlinks,
    linkcolor={blue!100!black},
    citecolor={blue!75!black},
    urlcolor={blue!50!black}
}
%% == 調整設定 ==
\usepackage{enumitem}                           % 修改 enumerate, item
\usepackage{bbding}
\usepackage{titletoc,titlesec,imakeidx}
%% == ==
\usepackage{newfloat}
\usepackage{caption,subcaption}
\usepackage{xkeyval,xargs}
\usepackage{ulem}
\usepackage{import}

%% === 設定 C++ 格式 ===
\lstset{%
  language=C++,             % 設定語言
  %% === 空白, tab 相關 ===
  tabsize=2,                % 設定 tab = 多少空白
  %showspaces=true,          % 設定是否標示空白
  %showtabs=true,            % 設定是否標示 tab
  %tab=\rightarrowfill,      % 設定 tab 樣式
  %% === 行數相關 ===
  numbers=left,             % 行數標示位置
  stepnumber=1,             % 每隔幾行標示行數
  numberstyle=\tiny,
  %breaklines=true,          % 設定斷行
  %% === 顏色設定 ===
  basicstyle=\ttfamily,
  keywordstyle=\color{blue}\ttfamily,
  stringstyle=\color{red!50!brown}\ttfamily,
  commentstyle=\color{green!50!black}\ttfamily,
  %identifierstyle=\color{black}\ttfamily,
  emphstyle=\color{purple}\ttfamily,
  extendedchars=false,
  texcl=true,
  moredelim=[l][\color{magenta}]{\#},
  captionpos=b,
  %% === 其他 ===
  %frame=single
}

% ===============================================
%
%  設定頁面格式
%
% ===============================================
%% === 設定頁面格式 ===
%\hoffset         = 10pt                      % 水平位移,預設為 0pt
\voffset         = -15pt                     % 垂直位移,預設為 0pt
\oddsidemargin   = 0pt                       % 預設為 31pt
%\topmargin       = 20pt                      % 預設為 20pt
%\headheight      = 12pt                      % header 的高度,預設為 12pt
%\headsep         = 25pt                      % header 和 body 的距離,預設為 25pt
\textheight      = 620pt                     % body 內文部分的高度,預設為 592pt
\textwidth       = 450pt                     % body 內文部分的寬度,預設為 390pt
%\marginparsep    = 10pt                      % margin note 和 body 的距離,預設為 10pt
%\marginparwidth  = 35pt                      % margin note 的寬度,預設為 35pt
%\footskip        = 30pt                      % footer 高度 + footer 和 body 的距離,預設為 30pt

%% ==  ==
\DeclareFloatingEnvironment[fileext=frm,placement={!ht},name=Frame]{code}
\captionsetup[code]{labelfont=bf}

\makeindex

\linespread{1.14}

\begin{document}
\begin{CJK}{UTF8}{bkai}

\subimport{../config/}{document-config.tex}
\setcounter{chapter}{5}

\fi

\chapter{動態規劃 (Dynamic Programming, DP)}

\section{概論 (Introduction)}

\paragraph{}\textbf{動態規劃 (常稱 DP)} 是一種解題常用的手段,它的想法與\textbf{遞迴與分治法}相近,將大問題分割成小問題來解決,但是在切割小問題的過程中,常常會遭遇到重複切割小問題的情形,使得程式執行速度變得緩慢而沒效率 (例:費氏數列),因此使用一種技巧,將做過的\textbf{小問題的答案儲存起來},以便下次再使用這個小問題時,能夠直接回傳答案而不浪費執行時間 (記憶化求解),這種技巧我們稱為動態規劃。
\paragraph{}以下將會介紹更多動態規劃的方法以及幾個經典的 DP 問題,至於能夠以 DP 解決的題目眾多,\sout{族繁不及被宰},更多的問題請自行在 Online Judge 中尋找並思考,這種題型是高中乃至大學競賽最愛考的題型,幾乎每場比賽都會出的必考題,請各位讀者好好把握。

\subsection{特性 (Property)}

\paragraph{}使用動態規劃的時機,取決於有沒有以下特性:
\begin{itemize}
\item \textbf{重疊子問題} (Overlapping Subproblems)
\paragraph{}這是使用動態規劃最根本的原因,因為我們在分割小問題時有重複子問題,才會有使用動態規劃,將重複問題的解答儲存的需要。

\begin{figure}
  \caption{$f_5$ 的遞迴}
  \label{fig:fibo-recursive}
\end{figure}

\paragraph{}最明顯的例子就是費氏數列,如果我們使用遞迴,畫出程式執行時的遞迴樹的話,我們很容易發現:當我們要算出費氏數列第 5 項 $f_5$ 時 (見圖 \ref{fig:fibo-recursive}),我們的程式會計算 $f_5$ 1 次、$f_4$ 1 次、$f_3$ 2 次、$f_2$ 3 次、$f_1$ 5 次、$f_0$ 3 次 (共計 15 次)!
\paragraph{}更明顯的例子,當我們要呼叫 f7 時,我們總共會計算 f7 1 次,f6 1 次,f5 2 次,f4 3 次,f3 5 次,f2 8 次,f1 13 次,f0 8 次!這樣子遞迴去計算的時間複雜度為 O(2n)!然而實際上,呼叫 8 次所得的 f0、13 次的 f1、8 次的 f2、 ...... 等等,這些答案都是相同的,因此我們在第一次呼叫 f0 時將其答案算出,儲存在陣列中,等待下次被呼叫時,再取先前算出的值,如圖 6.2:
\paragraph{}我們將重複的答案合併計算並儲存後,實際上每個函式只被計算 1 次,因此時間複雜度將降為 O(n)!這也是為什麼動態規劃存在的理由之一!
\paragraph{}

\item \textbf{最佳子結構} (Optimal Substructure)
\paragraph{}這是動態規劃可以使用的基礎。最佳子結構指的是,當我們將大問題切割為小問題時,小問題得到最佳解,遞推到大問題也可以得到大問題的最佳解。這個特性常常在求極值、以及求方法數的情況下非常顯著。
\paragraph{}如果沒有這個特性,則動態規劃不可使用,換句話說,如果從小問題得到的最佳解,並不能推出原問題的最佳解,那麼我們將原問題切割為小問題沒有太大意義。
\paragraph{}

\item \textbf{無後效性}
\paragraph{}這個特性跟最佳子結構是相似的意義,無後效性強調:當前所做的決策只跟當前的狀態有關,不會影響之後的狀態。動態規劃的方向常常由小問題開始往大問題合併、解決,換句話說,當前狀態 (小問題) 的決定並不會影響之後的決定 (大問題)。

\end{itemize}

\subsection{名詞定義 (Terminology)}

\begin{itemize}
\item 狀態
\paragraph{}描述切割子問題的性質,具體來說,狀態在動態規劃中表現為多維陣列的指標上,例如費氏數列的遞迴關係式中 F[i] = F[i-1] + F[i-2],此時費氏數列的「項數」即為「狀態」。狀態是我們自己去定義的,借由巧妙的定義,我們可以讓一個問題能夠符合動態規劃的原則 (指重疊子問題、最佳子結構、無後效性三個原則)。
\paragraph{}另外一種對於狀態的解釋:動態規劃用作搜索整個答案可能性的時候,我們會將原來問題的所有情形 (如棋盤上棋子的位置等) 儲存起來,利用每個情形之間的關係來做動態規劃,因為在第 4.2 節提到狀態空間搜索,這時動態規劃就可以套用當時的思路,進而降低複雜度。
\paragraph{}

\item 決策
\paragraph{}我們要從當前狀態遞推到其他狀態,我們可以列舉出一系列轉移的步驟,這些步驟統稱為決策,這也是找出狀態轉移方程的一種思路。
\paragraph{}

\item \textbf{狀態轉移方程 (State Transferring Formula)}
\paragraph{}當我們找出所有決策後,我們可以構造出狀態轉移方程,通常以遞迴式表示。找出狀態轉移方程有以下兩種方法:
  \begin{itemize}
  \item 向前遞推法
  \paragraph{}向前遞推法就是從一個已知的狀態中,遞推至未知的狀態。以下圖 6.3 為例,淡黃色的部分為我們已知的數值,當我們跑到已知數值 F[i] 時,他可以往下遞推到未知的區域 F[i + 1] 和 F[i + 2]。又稱正向法或順推法。
  \paragraph{}
  \item 向後遞推法
  \paragraph{}與前面的方法相反,向後遞推主要是從未知的區域往前求得與已知數值的關係,以下圖 6.4 為例,當我們遞推到未知數值 F[i],我們可以從已知數值 F[i-1] 和 F[i-2] 推得。又稱逆向法或逆推法。
  \paragraph{}無論是向前遞推法或是向後遞推,狀態轉移方程所表達是事情都是一樣的,因為狀態轉移方程可以利用數學形式來簡單描述 DP,所以接下來的討論多會有狀態轉移方程,而在開始討論 DP 前,我們還得先講 DP 的標記。
  \end{itemize}
\end{itemize}

\subsection{動態規劃標記 (Dynamic Programming Notation)}

\paragraph{}目前最方便分析 DP 的方法即是 xD/yD 標記法,x 和 y 皆為非負整數,兩者均表示多少\textbf{維度}的概念,例如:1D 代表一個維度、2D 代表兩個維度等等。
\paragraph{}xD 代表 DP 時狀態的維度,簡單來說就像是你 DP 時的陣列開了幾維,而 yD 代表每次狀態可能的轉移維度,通常都是這些轉移的可能中取極值或者求和。
\paragraph{}例如費氏數列的狀態數就是一個一維陣列,因此費氏數列的 $x=1$;而費氏數列的狀態轉移方程中 $F[i]=F[i-1]+F[i-2]$,決定 $F[i]$ 的轉移數只有一種 $O(1)$,因此維度為 $y=0$。總和而言,費氏數列為 1D/0D 的 DP。
\paragraph{}又例如 LIS (之後會介紹) 的狀態是一維陣列,$x=1$;轉移方程式為
\begin{align*}
LIS[i] = \max_{j<i}{\{LIS[j]+1\}}\text{ \textbf{if} }a[j]<a[i]
\end{align*}
,它是取第 $i$ 個元素前,所有符合條件的最小值,此時轉移數有 $O(n)$,因此維度為 $y=1$。總和而言,LIS 為 1D/1D 的 DP。
\paragraph{}這樣子的表示法使我們可以容易算出時間複雜度以及空間複雜度。根據剛剛的推論,時間複雜度為 $O(n^{x+y})$,空間複雜度為 $O(n^{x})$,不難。

\subsection{與遞迴的關係}

\paragraph{}在第二章我們學過遞迴,當時說過遞迴有兩個條件:第一、終止條件,第二、遞推條件。其實遞迴切割得好的話,那麼這個定義也可以利用動態規劃有效減少狀態數,此時終止條件和遞推條件各自對應:

\begin{enumerate}
\item 初始狀態 (終止條件)
\item 狀態轉移方程 (遞推條件)
\end{enumerate}

\paragraph{}有些時候的遞迴題目不太能用動態規劃解決 (例如回溯法系列的問題),不過這些題目大多的思想都是暴力搜尋,可以利用第五章學到的技巧去優化。

\section{線型動態規劃 (Linear DP)}

\subsection{費氏數列 (Fibonacci Numbers)}

\paragraph{}目前大家最熟知的動態規劃,即是費氏數列。以下是一道例題:
\paragraph{}有些讀者應該可以不難發現其中的關係,那麼沒發現的讀者也不要灰心,讓我們用上面的思路來解決此道問題。
\paragraph{}第一、決策,首先我們要找出爬樓梯所能做的決策,找出決策有兩派做法,這兩派就是我們上面所述的向前遞推法和向後遞推法。
\paragraph{}如果我們從向前遞推法的觀點來看,當你在第 i 階的時候,你可以做兩件事:

\begin{itemize}
\item 走到第 $i+1$ 階 (一次爬一階)
\item 走到第 $i+2$ 階 (一次爬兩階)
\end{itemize}

\paragraph{}找到決策之後,等於是我們也找到了狀態轉移方程,而向前遞推法,必須從已知的狀態類推到未知的狀態,因此我們還要找到第一個已知的狀態 (也就是遞迴中的中止條件)。
\paragraph{}第一個初始條件我們通常找你所設「狀態」的極值,此時的狀態該設為什麼呢?其實,你在不知不覺間已經設定好了狀態--你在第幾階就是你的狀態,因為他狀態只有一個,所以你所需要記錄的陣列只要開一維,也代表說動態規劃的方向是一維的。

\paragraph{}此時階數的範圍介於 $0<n<100$ 之間,我們可以找 $n=1$ 或是 $n=99$ 階的情況,明顯地我們找 $n=1$ 階可以很容易知道方法數只有 1 種,於是我們就找到第一個已知狀態。於是我們可以產生一段虛擬碼如下:

\begin{algorithm}
  \centering
  \begin{algorithmic}[1]
  \Procedure{$DP$}{}
    \State $stair[1]\gets{1}$
    \For{$i=1$ to $n-2$}
      \State $stair[i+1]\gets{stair[i+1]+stair[i]}$
      \State $stair[i+2]\gets{stair[i+2]+stair[i]}$
    \EndFor
    \State $stair[n]\gets{stair[n]+stair[n-1]}$
  \EndProcedure
  \end{algorithmic}
\end{algorithm}

\paragraph{}上述虛擬碼我們可以很容易發現:當我們從第 1 階走到第 2 階、第 3 階,第 2 階走到第 3 階、第 4 階, ... ,以此類推的時候,我們發現到最後時,第 n - 2 階走到第 n - 1 階、第 n 階,但是動態規劃還並未結束,如下圖所示。

\paragraph{}我們做完第 n - 2 階後,我們發現一項事實:第 n - 1 階可以走到第 n 階呀!但是你的程式中的迴圈設定中,第 n - 1 階不只會走到第 n 階,更會走到第 n + 1 階!超出範圍有可能會使你的程式發生錯誤,因此有兩種做法可以避免這種錯誤:第一種就將虛擬碼 6.6 第 7 行做特殊處理;第二種方法,就是加大陣列的長度,如下圖所示,我們額外加大陣列來讓最後第 n - 1 階執行一般式的時候不會出現錯誤。

\paragraph{}如果我們用向後遞推法來看的話,我們會看到這個問題:我們要怎樣才能走到第 i 階?這時我們可以從題目找出決策:

\begin{enumerate}
\item 從第 i - 1 階抵達第 i 階 (一次爬一階)
\item 從第 i - 2 接抵達第 i 階 (一次爬兩階)
\end{enumerate}

\paragraph{}向後遞推法通常可以很容易用數學來表示狀態轉移方程 stair[i] = stair[i-1] + stair[i-2]。這時我們也是以「階數」來做為狀態,接著我們要找他的初始狀態,由前面可以知道,已知狀態找 n = 1 是最方便的選擇。但是!讀者有沒有發現,如果在 i = 2 的時候,由狀態轉移方程我們得到關係式:stair[2] = stair[1] + stair[0],我們發現到在 i = 2 的時候也會產生類似的錯誤:在題目中 i = 0 沒有定義,因此狀態轉移方程在 i = 2 時不適用。

\paragraph{}那麼是遞推方程不正確嗎?不是,我們觀察一下 stair[i] = stair[i-1] + stair[i-2],因為我們的階數範圍一定在 0 和 100 之間,因此 i > 0 、 i-1 > 0 、 i-2 > 0,所以這個狀態轉移方程一定要在 i > 2 才能使用,而 i = 1 、 i = 2 時必須由我們自己找出它的值,才能讓動態規劃繼續下去 (初始條件很重要!!)。

\paragraph{}剛剛我們得到 i = 1 階時很明顯答案是 1 種,接著我們再來求 i = 2 階時的情形,如果我們要踏上第 2 階,我們可以從第 1 階爬上來、也可以從原點一次爬兩階上去,因此共有 2 種方法。

\paragraph{}初始條件有了,狀態轉移方程也有了,所以我們就可以找出動態規劃的虛擬碼如下:

\paragraph{}虛擬碼 6.10 可以發現,迴圈執行到最後並不用進行特殊處理,因為向後遞推法是以已知的條件推至未知的條件,在迴圈的使用上可以知道最後第 n 階 = 第 n-1 階 + 第 n-2 階,沒有邊界的問題。

\paragraph{}無論是向前遞推法或是向後遞推法都是很好的方法,而在此例中,在邊界條件的處理以及遞推關係式而言,筆者都覺得向後遞推法在此例較為方便。當然,讀者可以隨意利用此兩種方法來推導動態規劃的方程式,沒有固定的解答。

\subimport{../problem/}{06-dynamic-programming-fibonacci.tex}

\subsection{最長遞增子序列 (Longest Increasing Subsequence, LIS)}

\subimport{../problem/}{06-dynamic-programming-lis.tex}

\subsection{最長共同子序列 (Longest Common Subsequence, LCS)}

\subsection{LCS、LIS 相互轉換}

\subimport{../problem/}{06-dynamic-programming-lcs.tex}

\section{背包問題 (Knapsack Problems)}

\subsection{01 背包問題 (01-knapsack)}

\subimport{../problem/}{06-dynamic-programming-01-knapsack.tex}

\subsection{子集合求和問題 (Subset Sum Problem)}

\subimport{../problem/}{06-dynamic-programming-subset-sum.tex}

\subsection{完全背包問題--找零問題 (Coin Change)}

\subimport{../problem/}{06-dynamic-programming-coin-change.tex}

\subsection{多重背包問題 (Bounded Knapsack Problem)}
\subsection{分組背包問題}

\section{其他動態規劃}

\subsection{區間動態規劃}
\subsection{環形動態規劃}
\subsection{集合動態規劃 (狀態壓縮)}
\subsection{最大值最小化問題 (Min-max Problem)}
\subsection{區間最大和問題 (maximum consecutive subsequence)}
\subsection{最大矩形面積 (Largest Rectangle Area)}

\section{動態規劃優化 (Dynamic Programming Optimizations)}

\subsection{預處理}
\subsection{改變狀態}
\subsection{滾動數組}
\subsection{決策單調性}
\subsection{單調佇列優化}
\subsection{四邊形不等式優化}
\subsection{斜率優化}

\section{樹型動態規劃}

\subsection{樹形依賴背包}

\ifx \allfiles \undefined

\printindex[noun]
\clearpage

\end{CJK}
\end{document}

\fi
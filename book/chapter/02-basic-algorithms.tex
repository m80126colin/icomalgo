\ifx \allfiles \undefined
\documentclass[12pt,a4paper,oneside]{report}

%% === CJK 套件 ===
\usepackage{CJKutf8,CJKnumb}                 % 中文套件
%% === AMS 標準套件 ===
\usepackage{amsmath,amsfonts,amssymb,amsthm} % 數學符號
%% === ===
\usepackage{algorithm}
\usepackage{listings}                        % 程式碼
%% === TikZ 套件 ===
\usepackage{tikz,tkz-graph,tkz-berge}        % 繪圖
\usepackage{multicol}
%% == ==
\usepackage[unicode]{hyperref}
\usepackage{xcolor}
\hypersetup{
    colorlinks,
    linkcolor={blue!100!black},
    citecolor={blue!75!black},
    urlcolor={blue!50!black}
}
%% == 調整設定 ==
\usepackage{enumitem}                           % 修改 enumerate, item
\usepackage{bbding}
\usepackage{titletoc,titlesec,imakeidx}
%% == ==
\usepackage{newfloat}
\usepackage{caption,subcaption}
\usepackage{xkeyval,xargs}
\usepackage{ulem}
\usepackage{import}

%% === 設定 C++ 格式 ===
\lstset{%
  language=C++,             % 設定語言
  %% === 空白, tab 相關 ===
  tabsize=2,                % 設定 tab = 多少空白
  %showspaces=true,          % 設定是否標示空白
  %showtabs=true,            % 設定是否標示 tab
  %tab=\rightarrowfill,      % 設定 tab 樣式
  %% === 行數相關 ===
  numbers=left,             % 行數標示位置
  stepnumber=1,             % 每隔幾行標示行數
  numberstyle=\tiny,
  %breaklines=true,          % 設定斷行
  %% === 顏色設定 ===
  basicstyle=\ttfamily,
  keywordstyle=\color{blue}\ttfamily,
  stringstyle=\color{red!50!brown}\ttfamily,
  commentstyle=\color{green!50!black}\ttfamily,
  %identifierstyle=\color{black}\ttfamily,
  emphstyle=\color{purple}\ttfamily,
  extendedchars=false,
  texcl=true,
  moredelim=[l][\color{magenta}]{\#},
  captionpos=b,
  %% === 其他 ===
  %frame=single
}

% ===============================================
%
%  設定頁面格式
%
% ===============================================
%% === 設定頁面格式 ===
%\hoffset         = 10pt                      % 水平位移,預設為 0pt
\voffset         = -15pt                     % 垂直位移,預設為 0pt
\oddsidemargin   = 0pt                       % 預設為 31pt
%\topmargin       = 20pt                      % 預設為 20pt
%\headheight      = 12pt                      % header 的高度,預設為 12pt
%\headsep         = 25pt                      % header 和 body 的距離,預設為 25pt
\textheight      = 620pt                     % body 內文部分的高度,預設為 592pt
\textwidth       = 450pt                     % body 內文部分的寬度,預設為 390pt
%\marginparsep    = 10pt                      % margin note 和 body 的距離,預設為 10pt
%\marginparwidth  = 35pt                      % margin note 的寬度,預設為 35pt
%\footskip        = 30pt                      % footer 高度 + footer 和 body 的距離,預設為 30pt

%% ==  ==
\DeclareFloatingEnvironment[fileext=frm,placement={!ht},name=Frame]{code}
\captionsetup[code]{labelfont=bf}

\makeindex

\linespread{1.14}

\begin{document}
\begin{CJK}{UTF8}{bkai}

\subimport{../config/}{document-config.tex}
\setcounter{chapter}{1}

\fi

\chapter{基礎演算法}

\paragraph{}首先我們要介紹的第一個演算法是窮舉法。

\paragraph{}顧名思義,窮舉法將\textbf{所有可能的答案}都列舉出來,然後驗證這些答案的正確性。窮舉法有時又被稱為暴力搜尋法 (暴搜, Brute force),當測試\textbf{資料很小}時,窮舉法通常是一個很好的選項。

\paragraph{}因為這題的點數少,因此我們採用窮舉法:找出任意三個點,用行列式算出三角形的 2 倍面積,取出最大值後再除以 2,得到最大的三角形面積。

\paragraph{}讀者可能會有疑問,為什麼我們不直接算出三角形的面積來比大小呢?這是因為在電腦的世界中,浮點數存在\textbf{浮點數誤差},當我們的行列式算出面積時,有可能讓 7.5 誤差變成 7.499999999,因此有時候為了減少這種誤差擴大,我們會去減少浮點數的運算,避免誤差。

\paragraph{}當然,從此題的運算來看,我們沒有避免浮點數誤差的必要 (因為不管怎樣算最後都要除以 2,因此先除後除是沒有差別),只是習慣上當我們遇到浮點數運算時,會去注意到此類問題。 (詳見冼(ㄕㄣˇ,音省)鏡光〈使用浮點數最最基本的觀念〉)

\section{窮舉法 (Complete Search)}

\section{遞迴 (Recursion) 與分治法 (Divide \& Conquer)}

\subsection{簡介}

\subsection{河內塔 (Hanoi Tower)}
\subsection{約瑟夫問題 (Josephus Problem)}
\subsection{回溯法 (Backtracking)}
\subsection{八皇后 (8 queens)}
\subsection{數獨 (Sudoku)}
\subsection{快速乘方法 (Fast Exponential Algorithm)}
\subsection{排列組合問題}

\section{貪婪法 (Greedy)}

%\subsection{常見做法}
%\subsection{找零問題 (Coin Changing Problem)}
%\subsection{工作排程 (Activity Selection Problem)}
%\subsection{誰先晚餐}
%\subsection{分數背包 (Fractional Knapsack Problem)}
%\subsection{過橋問題 (Bridge and Torch Problem)}
%\subsection{看動畫}
%\subsection{其他應用}

\ifx \allfiles \undefined

\printindex[noun]
\clearpage

\end{CJK}
\end{document}

\fi
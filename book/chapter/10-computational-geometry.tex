\ifx \allfiles \undefined
\documentclass[12pt,a4paper,oneside]{report}

%% === CJK 套件 ===
\usepackage{CJKutf8,CJKnumb}                 % 中文套件
%% === AMS 標準套件 ===
\usepackage{amsmath,amsfonts,amssymb,amsthm} % 數學符號
%% === ===
\usepackage{algorithm}
\usepackage{listings}                        % 程式碼
%% === TikZ 套件 ===
\usepackage{tikz,tkz-graph,tkz-berge}        % 繪圖
\usepackage{multicol}
%% == ==
\usepackage[unicode]{hyperref}
\usepackage{xcolor}
\hypersetup{
    colorlinks,
    linkcolor={blue!100!black},
    citecolor={blue!75!black},
    urlcolor={blue!50!black}
}
%% == 調整設定 ==
\usepackage{enumitem}                           % 修改 enumerate, item
\usepackage{bbding}
\usepackage{titletoc,titlesec,imakeidx}
%% == ==
\usepackage{newfloat}
\usepackage{caption,subcaption}
\usepackage{xkeyval,xargs}
\usepackage{ulem}
\usepackage{import}

%% === 設定 C++ 格式 ===
\lstset{%
  language=C++,             % 設定語言
  %% === 空白, tab 相關 ===
  tabsize=2,                % 設定 tab = 多少空白
  %showspaces=true,          % 設定是否標示空白
  %showtabs=true,            % 設定是否標示 tab
  %tab=\rightarrowfill,      % 設定 tab 樣式
  %% === 行數相關 ===
  numbers=left,             % 行數標示位置
  stepnumber=1,             % 每隔幾行標示行數
  numberstyle=\tiny,
  %breaklines=true,          % 設定斷行
  %% === 顏色設定 ===
  basicstyle=\ttfamily,
  keywordstyle=\color{blue}\ttfamily,
  stringstyle=\color{red!50!brown}\ttfamily,
  commentstyle=\color{green!50!black}\ttfamily,
  %identifierstyle=\color{black}\ttfamily,
  emphstyle=\color{purple}\ttfamily,
  extendedchars=false,
  texcl=true,
  moredelim=[l][\color{magenta}]{\#},
  captionpos=b,
  %% === 其他 ===
  %frame=single
}

% ===============================================
%
%  設定頁面格式
%
% ===============================================
%% === 設定頁面格式 ===
%\hoffset         = 10pt                      % 水平位移,預設為 0pt
\voffset         = -15pt                     % 垂直位移,預設為 0pt
\oddsidemargin   = 0pt                       % 預設為 31pt
%\topmargin       = 20pt                      % 預設為 20pt
%\headheight      = 12pt                      % header 的高度,預設為 12pt
%\headsep         = 25pt                      % header 和 body 的距離,預設為 25pt
\textheight      = 620pt                     % body 內文部分的高度,預設為 592pt
\textwidth       = 450pt                     % body 內文部分的寬度,預設為 390pt
%\marginparsep    = 10pt                      % margin note 和 body 的距離,預設為 10pt
%\marginparwidth  = 35pt                      % margin note 的寬度,預設為 35pt
%\footskip        = 30pt                      % footer 高度 + footer 和 body 的距離,預設為 30pt

%% ==  ==
\DeclareFloatingEnvironment[fileext=frm,placement={!ht},name=Frame]{code}
\captionsetup[code]{labelfont=bf}

\makeindex

\linespread{1.14}

\begin{document}
\begin{CJK}{UTF8}{bkai}

\subimport{../config/}{document-config.tex}

\fi

%%
\newcommand{\True}{\textsc{True}}
\newcommand{\False}{\textsc{False}}

\chapter{計算幾何}

\section{向量}
\subsection{向量運算}

\paragraph{}向量是數學上常用的一種思維,它包含了大小及方向兩種概念,無論是在二維平面、或是三維空間中,我們通常以一個箭頭來表示一個向量。舉例來說,小明今天往北走了 5 公里,那麼「往北走了 5 公里」這句話就代表著一個向量--大小是 5 公里而方向是北方;另一個例子,假設小明往東北走了 10 公里,那麼這時大小是 10 公里、方向是東北方。
\paragraph{}為了以數學來描述向量,我們利用座標平面,來描述二維平面上向量所擁有的特性。假設 x-y 平面上,以 x 軸正向為東邊,y 軸正向為北邊,以第一個例子來說,假設小明一開始在原點 (0,0) 的位置,往北走 5 公里後,小明的位置會在 (0,5) 的地方,這時 (0,0) 和 (0,5) 間的直線距離就會「產生」一條向量,方向是從 (0,0) 指向 (0,5),如圖 10.1。
\paragraph{}讀者可能會以為,向量就是小明所走過的路徑。其實並不盡然,假設小明的好朋友小華,他昨天從原點往東邊走了 4 公里到達 (4,0),休息一下過後,接著再往北走 3 公里抵達 (4,3),依照上面的法則,我們可以像圖 10.2 一樣畫出兩條藍色的,分別是 (0,0) 指向 (4,0),和 (4,0) 指向 (4,3) 的兩條向量。
\paragraph{}但是總和來說,我們可以看做是起點是 (0,0),終點是 (4,3) 的一條向量,如圖 10.3 的紅色向量。我們把兩條藍色向量「合併」成一條紅色向量,這就是向量的「加法」。向量加法可以看做是兩段路程相加所造成的結果。
\paragraph{}為了精準描述向量,我們定義向量--意即箭頭--的兩端,三角形的箭頭表示「終點」的概念,而另外一端是「起點」的概念,如圖 10.4。
\paragraph{}那麼我們用一組數字 (x,y) 來描述一個向量:定義一個向量的起點在 (0,0) 的位置,終點在 (x,y) 的位置,那麼這個向量的值為 (x,y)。這麼定義的好處是:我們可以很方便作向量上的一些運算,與一開始提到向量是有「大小」和「方向」的概念有些微不同,但是經由一些運算,我們也是能求出向量所擁有的大小和方向。
\paragraph{}我們回頭看小明,小明原本起點在 (0,0) 的位置,終點在 (0,5),因此小明距離原點的向量為 (0,5);而小華一開始在原點 (0,0) 的位置,經過 (4,0) 後,最後「終點」在 (4,3),注意!向量只看「起點」和「終點」的直線距離,不管路程為何,因此小華總和的向量為 (4,3),即是圖 10.3 的紅色向量。
\paragraph{}我們再來仔細分析小華的例子:我們剛剛從圖 10.3 得到的結論--圖中紅色向量是兩條藍色向量「相加」的結果。那我們要怎樣知道這兩條藍色向量「相加」會得到後來的紅色向量呢?我們先從那兩條藍色向量開始觀察。第一條藍色向量,是從原點走到 (4,0),因此第一段向量的值毫無意外是 (4,0)。
\paragraph{}第二段向量就麻煩了,它的起點是 (4,0),終點是 (4,3),跟我們原來向量起點在 (0,0) 的定義並不相同,這時我們如果想要知道它的向量,我們就得「假裝」把它的起點搬回原點,如圖 10.5,我們把整段向量搬回起點時,終點位置的座標也會跟著改變,不難想像整段向量是「平移」回原點的,因此起點座標從 (4,0) 變成 (0,0),那終點座標如何改變呢?
\paragraph{}我們觀察起點座標的數值變化,我們發現:當起點從 (4,0) 搬回 (0,0),x 座標減了 4。因此,終點座標在平移的過程中,x 座標也減去 4,最後終點座標變成了 (0,3),由此可知第二段向量的值就是 (0,3)。
\paragraph{}我們知道兩段藍色向量的值 (4,0) 和 (0,3),我們可以發現,這兩段向量的 x 值相加,並且 y 值相加,就會是最後的紅色向量 (4,3)。其實,向量加法也可以從原本的敘述中找出:小華原本在原點,一開始先「向東走 4 公里」(第一段藍色向量),接著「往北走 3 公里」(第二段藍色向量),因此最後結果就會變成--
\paragraph{}(4,0) + (0,3) = (4+0, 0+3) = (4,3)
\paragraph{}更廣義來說,如果兩個向量 (x1,y1) 和 (x2,y2) 相加,他的結果為:
\paragraph{}(x1 + x2, y1 + y2)
\paragraph{}此外,剛剛我們還運用到了「向量減法」的概念,同樣的道理,如果有兩個點 (x2, y2) 和 (x1, y1) 相減,他的結果會是:
\paragraph{}(x2 - x1, y2 - y1)
\paragraph{}向量減法可以看做是以 (x1,y1) 做為起點、(x2,y2) 做為終點時向量的值。

\subsection{內積}

\paragraph{}是向量的一種乘法,常常被稱為內積 (Inner Product)。
\begin{algorithm}
\caption{內積}
\label{algo:dot_product}
\begin{algorithmic}
\Procedure{Dot}{$O,A,B$}\Comment{$O,A,B$ 是點座標}
  \State \Return $(A.x-O.x)\times{(B.x-O.x)}+(A.y-O.y)\times{(B.y-O.y)}$
\EndProcedure
\end{algorithmic}
\end{algorithm}

\begin{algorithm}
\caption{向量內積}
\label{algo:dot_product_vector}
\begin{algorithmic}
\Procedure{DotVector}{$A,B$}\Comment{$A,B$ 是向量}
  \State \Return $A.x\times{B.x}+A.y\times{B.y}$
\EndProcedure
\end{algorithmic}
\end{algorithm}

\subsection{外積}
\paragraph{}常常被稱為外積 (Outer Product)。

\begin{algorithm}
\caption{外積}
\label{algo:cross_product}
\begin{algorithmic}
\Procedure{Cross}{$O,A,B$}\Comment{$O,A,B$ 是點座標}
  \State \Return $(A.x-O.x)\times{(B.y-O.y)}-(A.y-O.y)\times{(B.x-O.x)}$
\EndProcedure
\end{algorithmic}
\end{algorithm}
\begin{algorithm}
\caption{向量外積}
\label{algo:cross_product_vector}
\begin{algorithmic}
\Procedure{CrossVector}{$A,B$}\Comment{$A,B$ 是向量}
  \State \Return $A.x\times{B.y}-A.y\times{B.x}$
\EndProcedure
\end{algorithmic}
\end{algorithm}

\section{計算幾何初步}

\subsection{基本操作}
\paragraph{}計算幾何中,通常要注意的有兩件事:第一、就是浮點數誤差;而第二、就是我們如何減少計算量來提高求出答案的效率。
\paragraph{}首先,我們會遇到的問題是:怎樣判斷一個浮點數為零?

\begin{algorithm}
\caption{判斷浮點數為零}
\begin{algorithmic}
\Procedure{IsZero}{$x$}\Comment{$x$ 是一個浮點數}
  \If{$-\epsilon<x<\epsilon$}\Comment{$\epsilon$ 是你「定義」的一個很小的常數}
    \State \Return \True
  \Else
    \State \Return \False
  \EndIf
\EndProcedure
\end{algorithmic}
\end{algorithm}

\ifx \allfiles \undefined
\printindex[noun]

\clearpage
\end{CJK}
\end{document}
\fi
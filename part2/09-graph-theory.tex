\chapter{圖論 (Graph Theory)}

\section{緒論 (Introduction)}
\subsection{頂點和邊 (Vertex and Edge)}
\subsection{圖與分支度 (Graph and Degree)}
\subsection{子圖 (Subgraph)}
\subsection{路徑與環 (Path and Cycle)}
\subsection{二分圖與完全圖 (Bipartite and Complete Graph)}
\subsection{連通 (Connected)}
\subsection{權重 (Weight)}

\section{圖的表示法 (Representation)}
\subsection{相鄰矩陣 (Adjacency Matrix)}
\subsection{相鄰串列 (Adjacency Lists)}
\subsection{邊列表 (Edge List)}
\subsection{前向星法 (Forward Star)}
\subsection{鏈式前向星}
\subsection{綜合比較}

\section{遍歷與排序}
\subsection{圖的遍歷 (Traversal of Graph)}
\subsection{時間戳記法 (Time Stamp)}
\subsection{有向無環圖 (Directed Acyclic Graph, DAG)}
\subsection{拓樸排序法 (Topological Sort)}

\section{最小生成樹 (Minimum Spanning Trees)}
\subsection{問題定義}
\subsection{Kruskal's 演算法}
\subsection{Prim's 演算法}

\section{最短路徑問題 (Shortest Path Problem)}
\subsection{問題定義}
\subsection{Bellman-Ford 演算法}

\begin{algorithm}
\label{algo-bellman-ford}
\caption{Bellman-Ford 演算法}
\begin{algorithmic}[1]
\Procedure{BellmanFord}{$s, G,\omega$}
  \State\Comment{$path$ 紀錄所有點到 $s$ 當前的最短路徑}
  \For {$v\in{V(G)}$}\Comment{初始化 $path$}
    \State $path[v]\gets\infty$
  \EndFor
  \State $path[s]\gets{0}$
  \For {$i=1\cdots{}|V(G)|-1$}\Comment{執行 $|V(G)|-1$ 次迴圈}
    \For {$e=\langle{u,v}\rangle\in{E(G)}$}\Comment{每次對所有邊擴張看看}
      \If {$path[u]+\omega{(u,v)}<path[v]$}
        \State $path[v]\gets{path[u]+\omega{(u,v)}}$
      \EndIf
    \EndFor
  \EndFor
  \For {$e=\langle{u,v}\rangle\in{E(G)}$}\Comment{檢查有無負環}
    \If {$path[u]+\omega{(u,v)}<path[v]$}\Comment{還能擴張代表不合理}
      \State \Return ``存在負環''
    \EndIf
  \EndFor
  \State \Return $path$
\EndProcedure
\end{algorithmic}
\end{algorithm}

\subsection{SPFA (Shortest Path Faster Algorithm)}

\begin{algorithm}
\label{algo-spfa}
\caption{Shortest Path Faster Algorithm (SPFA)}
\begin{algorithmic}[1]
\Procedure{ShortestPathFasterAlgorithm}{$s, G,\omega$}
  \State\Comment{$path$ 紀錄所有點到 $s$ 當前的最短路徑}
  \For{$v\in{V(G)}$}\Comment{初始化 $path$}
    \State $path[v]\gets\infty$
  \EndFor
  \State $path[s]\gets{0}$
  \State $Q.push(s)$\Comment{$Q$ 是 queue}
  \While{$Q.size()\geq{1}$}
    \State $u\gets{Q.front()}$
    \State $Q.pop()$
    \For{$v\in{V(G)}$ where $(u,v)\in{E(G)}$}\Comment{所有和 $u$ 相鄰的點}
      \If{$path[u]+\omega{(u,v)}<path[v]$}
        \State $path[v]\gets{path[u]+\omega{(u,v)}}$
        \If{$v\notin{Q}$}
          \State $Q.push(v)$
        \EndIf
      \EndIf
    \EndFor
  \EndWhile
  \State \Return $path$
\EndProcedure
\end{algorithmic}
\end{algorithm}

\subsection{Dijkstra's 演算法}

\subsection{Floyd-Warshall 演算法}

\begin{algorithm}
\label{algo-floyd-warshall}
\caption{Floyd Warshall 演算法}
\begin{algorithmic}[1]
\Procedure{Floyd-Warshall-Algorithm}{$G,\omega$}
  \For {$k\in{V(G)}$}
    \For {$i\in{V(G)}$}
      \For {$j\in{V(G)}$}
        \If {$\omega{(i,j)}<\omega{(i,k)}+\omega{(k,j)}$}
          \State $\omega{(i,j)}\gets{\omega{(i,k)}+\omega{(k,j)}}$
        \EndIf
      \EndFor
    \EndFor
  \EndFor
\EndProcedure
\end{algorithmic}
\end{algorithm}

\section{圖的連通性 (Connectivity)}
\subsection{問題定義}
\subsection{關節點 (Articulation Point)}
\subsection{橋 (Bridge)}
\subsection{強連通分量 (Strong Connected Component, SCC)}
\subsection{雙連通分量 (Bi-Connected Component, BCC)}

\section{匹配問題 (Matching)}
\subsection{問題定義}
\subsection{交錯軌道 (Alternating Path) 和增廣路徑 (Augmenting Path)}
\subsection{最大二分匹配 (Maximum Bipartite Matching)}
\subsection{最大權二分匹配--匈牙利演算法 (Hungary Algorithm)}

\section{網路流 (Network Flow)}
\subsection{問題定義}
\subsection{最大流量問題 (Maximum Flow)}
\subsection{Ford-Fulkerson 演算法}
\subsection{Edmond-Karp 演算法}
\subsection{網路流建模}
\subsection{最小切割 (Minimum Cut)}
\subsection{最大流最小切割 (Max-flow Min-cut)}
\subsection{最小成本最大流 (Min-cost Max-flow)}